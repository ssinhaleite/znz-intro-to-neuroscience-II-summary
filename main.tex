\documentclass[12pt,article,oneside,a4paper]{memoir}

%% Packages
%% ========
\usepackage{graphicx}
\usepackage{titlesec}
\usepackage{wrapfig}
	
\setcounter{secnumdepth}{4}

\titleformat{\paragraph}
{\normalfont\normalsize\bfseries}{\theparagraph}{1em}{}
\titlespacing*{\paragraph}
{0pt}{3.25ex plus 1ex minus .2ex}{1.5ex plus .2ex}

%% many common packages
\input{commonpackages}

%% Some more packages that you may want to use.  Have a look at the
%% file, and consult the package docs for each.
\input{extrapackages}

%% Our layout configuration.
\input{layoutsetup}

%% Theorem environments.  You will have to adapt this for a German
%% thesis.
\input{theoremsetup}

%% Helpful macros.
\input{macrosetup}

%%page layout settings and listing templates etc.
\input{settings}

\title{\textbf{ZNZ Introduction to Neuroscience II} \\
       Spring 2017\\\normalsize version 1.0}

\author{
	Vanessa Leite
	\vspace{2em}
	\\Repository page: \url{https://github.com/ssinhaleite/znz-intro-to-neuroscience-II-summary}\\
	Contact \href{mailto:vrcleite@gmail.com}{vrcleite@gmail.com} if you have any questions.}
	\thesistype{The Summary of the lectures in 2017}
	\department{ZNZ - Institute of Neuroinformatics, ETH}
	\date{\today}

\begin{document}
\frontmatter


%% DO NOT CHANGE.
\begin{titlingpage}
  \calccentering{\unitlength}
  \begin{adjustwidth*}{\unitlength-24pt}{-\unitlength-24pt}
    \maketitle
  \end{adjustwidth*}
\end{titlingpage}

\mainmatter

%% This change is needed if the article option for the memoir document class
%% is used, in order to count sections (article) as if they were chapters (memoir)
\counterwithout{section}{chapter}

%% Our content

\newpage
\clearpage
\pagenumbering{roman}
\setcounter{tocdepth}{3}
\setcounter{secnumdepth}{2}
\tableofcontents

\clearpage
\pagenumbering{arabic}

\section{Cognitive Neuroscience}
%%-----------------------------------------------------------------------------------------%%
\subsection{Methods}

\subsection{Perception and Attention - prof. D. Kiper - 27.02.2017}
\paragraph{Slide \#1:} Perception is not passive, it is a process. Description of elements of perceptual process and factors that influence what individuos perceive.
\paragraph{Slide \#2:} within them: particular perception (proprioception); meaningful experiences: we are not passive analyzers
\paragraph{Slide \#3:} When the senses are activated, starts the perceptual selection
\paragraph{Slide \#4:} enviromental stimulli: sensory depravation tank: if you do not receive stimulus, your brain creates it.
\paragraph{Slide \#5:} Selective screening: -our system eliminates some factors because they are not important for us to be aware of. -internal and external factors influencing in the perception. - cronic depression: slow/fast twich (?)
\paragraph{Slide \#6:} internal factors: - state of adaptation - constancies: color constancy: your brain starts to iluminate the enviroment making you think the color is the same in different enviroments.
\paragraph{Slide \#7:} perceptual organization

\paragraph{Slides \#8-11:} Notes: -one hand: not think perception as a passive process (sensation is the passive process) - perception is noisy and makes mistakes - very complex system.

\paragraph{Slides \#12-14:} Types of attention: 1. overt attention, 2. covert attention: related with spotlight of attention (slide \#15), you can't pay attention to many things. When you pay attention in one thing, your capacity to pay attention in others decay. 3. feature attention: shadowing tasks (slide \#16)

\paragraph{Slide \#17:} frontal lobe $\rightarrow$ saccade eyes movements (overt); parieral lobe $\rightarrow$ covert attention
\paragraph{Slide \#18:} theory about how attention works
\paragraph{Slide \#19:} one "phrase(?)" in each ear.
\paragraph{Slides \#20-21:} a lot of simple experiments can be made to evaluate attention, for instance, pop out. It is to "find" elements based on one feature, more features (conjunction) $\rightarrow$ takes longer to find it. $\rightarrow$ pop out experiments can prove sinestesia
\paragraph{Slides \#21-22:} Desimone experiments. MTA (medial temporal area) $\rightarrow$ prefered direction of motion $\uparrow$. Attention is like a weight to fire particular neurons.
\paragraph{Slides \#23-?:} a lot of slides were ignored for later reading
\paragraph{Gorila video:} innatentional blindness

%%-----------------------------------------------------------------------------------------%%
\subsection{Decision Making - 06.03.2017}

\paragraph{Appetitive and aversive value}
\paragraph{Multivoxel pattern analysis} PFC, LPFC, Parietal region
\paragraph{Deciding for self vs others}
\paragraph{Executed:} decision for others
\paragraph{Model:} I'm not choosing for me but I'm still modeling. What I would choose for me?
\paragraph{vmPFC} = ventral medium pre frontal cortex
\paragraph{Areas on the brain that are not core for the type of the test but still may be part of the decision}
\paragraph{comparison:} costs x bemefits
\subparagraph{delay costs}
\subparagraph{effort costs}
\paragraph{OFC} = orbital frontal cortex
\paragraph{ACC} = anterior cingulate cortex

\paragraph{Decision} related with learning: depends of what you have learned
\paragraph{Value (as in food) is not purely motor or sensory}
\subparagraph{how much people like chocolate after each square}
\paragraph{All regions (on slide) receive input from VTA - dopaminergic neurons}

\paragraph{long term memory}
\paragraph{conditioning is impaired in amygdala but not in hippocampus (can not hold facts)}
\paragraph{fear impairement (patient SM)}
\paragraph{Similarity = shape, etc}
\paragraph{positive and negative prediction errors}
\paragraph{learning is proportional to prediction errors}

\paragraph{dopamine neurons:} action potentials fires when unpredicted reward occurs; fewer action potentials than "normal" when no reward occurs.

\paragraph{learning from others}
\paragraph{selfish = other = dorsal region}

\paragraph{Representation $\rightarrow$ move from objective to subjective}
\paragraph{dopamine neurons may represent values in a objective way}
\subparagraph{the same value has lower reward value if the subject needs to wait more for it}

%%-----------------------------------------------------------------------------------------%%
\subsection{Emotion - 13.03.2017}
\paragraph{Find methods to measure emotions}
\paragraph{What is an emotion?} emotions are short lasting x mood are long lasting
\paragraph{emotion as everyday feeling}
\subparagraph{from psycology}
\subparagraph{from neuroscience:} neurons controls behaviours
\paragraph{emotion as feelings:} qualia $\rightarrow$ non communicative; verbal expression
\paragraph{comparative:} emotion is shared between humans and animals
\paragraph{fear conditionining x fear expression x fear prosody}

\paragraph{common sense:} we do things because we feel
\paragraph{Physiological theory:} we feel because we sense things (attribution: think about the senses)
\paragraph{Facial feedback:} active some muscles leads to feeling
\paragraph{Appraisal theory:} automated no-consciouness that leads sensation to feeling
\paragraph{Affect theory:} framework to talk about subject feelings

\subsubsection{ Decision Theoretical View}
\paragraph{emotions are actions:} adaptive to achieve goals.
\paragraph{Attention on two things:} goals and controls algorithms (that decide the actions)

\paragraph{rats cannot learn shock from taste, but they learn from be sick}

\paragraph{many "algorithms" for many behaviours}

\paragraph{game:} go pick a "piece" fast enough to not be hitted by "rock"

\paragraph{hippocampus lesion:} difficult approach
\paragraph{amygdala lesion:} return approach

\paragraph{Approximating algorithm: } optimal is too much complicated; why more time to go when is more difficult?

\paragraph{meta control:} how we decide what we decide?

\paragraph{Why study humans?} communication of emotions

\subsubsection{Clinical application}
\paragraph{fear extinction:} you can train the extinction of fear. However, the complete memory is not "removed".
\paragraph{memory reconsolidation:} memory consolidation needs proteins.
\paragraph{fear erasure:} where in the brain the fear seats? amygdala?

\paragraph{Multivariate fMRI analysis}
\subparagraph{benzodiapezines:} ansiolitics
\subparagraph{lesion on hippocampus make rats stay longer on open arm}
\paragraph{slide: benzo:} red line; placebo: black line (human analogon)

%%-----------------------------------------------------------------------------------------%%
\subsection{Memory - Prof. Katharina Henke - 20.03.2017}
\subsubsection{Methods and Memory}
\paragraph{fMRI}
\paragraph{what structures support learning / tasks activities / retrieval activity?}
\paragraph{Discover silent process (animals and humans)}
\paragraph{Read the unconscionous mind using fMRI}
\paragraph{Identify people with real memory problem}

\paragraph{To identify areas on the brain essential for a task is necessary to analyze people with brain damage (in a specific area)}

\paragraph{PET - Tomography:} used on study of learning and memory.
\paragraph{Patient:} 23 years old - psycological trauma
\paragraph{No brain damage - no structural damage}
\paragraph{Glucose pet indicates if area is working - functional amnesia}

\subsection{Learning during sleep - EEG}
\paragraph{slowing sleep - delta waves}

\subsection{Multistore model}
\paragraph{Sensory memory (up to 1s) $\rightarrow$ STM (15-20s to 1 min) $\rightarrow$ LTM (minutes and longer)}
\paragraph{Hippocampus:} episodic memory $\rightarrow$ memory for personal episodies - autobiographical, what, where, when
\paragraph{Hippocampus - very vulnerable $\rightarrow$ lots of diseases by damage on hippocampus}
\paragraph{ovulation invreases hyppocampus activity as in menstruation (?)}
\subsection{Declarative (explicit) memory }
\paragraph{Semantic memory:} independent of hippocampus
\subparagraph{through repetition is possible to learning new things using the neocortex and not passing throught the hippocampus}
\subsection{Nondeclarative (implicit) memory - unconsciouness}
\paragraph{Procedural learning:} HM patient still able to ride a bike.
\paragraph{Priming:} tendency to process some perception in the same way. When you see a complex picture you need a longer time to identify things then in a second/third time.

\paragraph{1990-2000}
\paragraph{Healthy patient do not show hippocampus activity during learning / memory}
\paragraph{Hippocampus is speciallized in associate memory}
\subparagraph{anterior part:} semantic and temporal association
\subparagraph{posterior part:} sensory and spatial
\paragraph{Parahippocampus gyrus:} individual objects - important for priming

\subsubsection{Memory $\rightarrow$ new division $\rightarrow$ Model of Katharina (2010)}
\paragraph{Encoding:} rapid / slow
\paragraph{Association:} flexible / rigid / no association (unique item)
\paragraph{Learning in sleep when the second word (?) is received in a moment where the neuron is polarized}

%%-----------------------------------------------------------------------------------------%%
\subsection{Body Perception - 27.03.2017}
\subsubsection{Phantomology}
\paragraph{hemiphantom:} one side
\paragraph{phenomenon:} studied since 1510 (Paré) but term "phantom limb" is from 1829 (Mitchell).
\paragraph{to read:} Finger and thustwitt, 2003, Neurosurgery 52.
\paragraph{Sensetion is in the brain, not in the limb - Descartes}
\paragraph{experiment with adult monkeys:} lose a finger and the neighbors fingers took the "empty" area of the finger on the brain - cortical plasticity
\paragraph{Maladaptive model:} pain is the price for plasticity
\paragraph{It is not very clear what is guided by phantom limb and why some people feel phantom limb pain}
\paragraph{Negative phantom limb:} there is a limb but no "connections", also called xenomelia
\paragraph{Phantom limb:} approximately 50\% believes the phantom limb is still there even when a pile of books is placed at the same location of the limb. 50\% perceive that the phantom limb is in his/her head.
\paragraph{research relevance:} visual-somesthetic interactions.
\paragraph{clinical relevance:} protesis.

\paragraph{You do not need to physically lose a limb to experience phantom limb. The visual system can stop the feeling of phantom limb.}

\paragraph{supernumary (?)}
\paragraph{Even if the phantom limb sensations disappears the phantom limb pain can still holds}
\paragraph{Phantom limb can be feel even in congenital limb experiencies. Scientific fact: does not mean remembering is not important, also does not mean "body schema" is "inate"}
\paragraph{Phantom moving is like imagine moving a limb - bilateral activation on brain}

\section{Clinical Neuroscience}

\subsection{Neurology: Ophthalmology, Otology, Epileptology and Parkinson - 03.04.2017}
\subsubsection{Neuro ophtamology}
\paragraph{Clinical eye testing}  
\begin{itemize}
\item moviments
\item smooth pursuit
\item saccadic eye moviemnt - fast eye moviment
\subitem disorders: velocity, metrics (can "pass" the point), latency
\item nystagmus
\subitem primary gaze: indicative of cerebellum loss $\rightarrow$ gaze holding and rebound
\subitem vestibulo occular reflex: the head moves and the eyes move together and then, after a while the eyes turn to the object-goal position
\end{itemize}

\paragraph{Clinical balance testing}
\begin{itemize}
\item spontaneous nystagmus: eyes drift to the side of the loss.
\item head impulse test: negative $\rightarrow$ eye stays in position; positive: $\rightarrow$ eyes follow head and then turn.
\item vertical: "close" one eye and the other stays in the same position
\item dynamic: see letters moving the head
\end{itemize}

\paragraph{vertical occular deviation}
\paragraph{dynamic visual acuity}

\paragraph{sensorymotor balance:} romberg test: close eyes and balance is lost.
\paragraph{caloric testing}
\paragraph{bimallolar (?) vibration sense}

\subsubsection{Epilepsy}
\paragraph{What is epilepsy?} brain disorder, chronic condition
\paragraph{What is a seizure?} temporary disruption of normal brain function.
\begin{itemize}
\item eye moviment, body tension
\item depending on the anatomical location of the seizure, can be all body or only an arm for instance
\end{itemize}
\paragraph{EEG} pyramidal cells in the cortex: sleep and close eyes produce high variations.
\paragraph{Seizures does not mean epilepsy.} Some seizures can be provocated.

\paragraph{Classification of seizures:}
\begin{itemize}
\item focal: part of the brain
\item generalized: whole brain; always without consciouness.
\end{itemize}

\subsection{Neurology: Multiple Sclerosis, Neuromuscular, Stroke and Neuropsychology}
\subsection{Spinal Cord Injury}

\subsection{Epilepsy}
 
\subsection{Depression}

\subsection{Schizophrenia}

\subsection{Addiction Clinics}

\subsection{Addiction in Society}

\subsection{Neurosurgery}
some text here and one citation: \cite{Example}

%%%%%%%%%%%%%%%%%%%%%%%%%%%%%%%%%%%%%%%%%%%%%%%%%%%%%%%%%%%%%%%%%%%%%%%%%%%%%%%%%%%%%%%%%%%%%%%%%%%
\newpage


\section{Previous exams}

\subsubsection{2013}
\begin{enumerate}
\item \paragraph{FMRI}
\item \paragraph{Microglia}
\item \paragraph{Circadian Rhythms}
\item \paragraph{Motor Learning}
\item \paragraph{Parkinson}
\item \paragraph{Multiple Sclerosis}
\end{enumerate}

\subsubsection{2011}
\begin{enumerate}
\item \paragraph{ fMRI is routinely used to study the neural processes underlying behavior. Please describe all the procedures necessary for conducting and correctly interpreting an fMRI study, covering the following areas:}
\subparagraph{Data acquisition (what signals are measured in fMRI, and how?)}
\subparagraph{Data analysis (which sequential routines are necessary to detect signal changes in fMRI images, using software packages such as SPM?)}
\subparagraph{Results interpretation (what inferences about neuronal activity can be drawn from fMRI results?}

answer here.

\item \paragraph{Please list the different types of long-term memory you know of. Describe their properties in humans, group them according to involved brain structures and give examples of behavioral tests that allow to model these memory types in rodents}

\item \paragraph{Sleep regulation in physiological short and long sleepers: Explain the most important principles how sleep and wakefulness are physiologically regulated and how sleep-wake regulation may differ between habitual short and long sleepers.}

\item \paragraph{Robotic tools have played a significant role in the investigation of human motor
learning.} \subparagraph{Describe the role of internal models in human motor control and how such
models are acquired} \subparagraph{Identify three unique features of robotic systems that make them valuable tools to investigate human motor learning.} \subparagraph{Discuss how these unique features could be applied to clinical assessment and therapy of sensorimotor impairments} answer here.

\item \paragraph{The term "frontotemporal dementias" subsumes a heterogeneous group of disorders:}
\subparagraph{Please describe the clinical presentations of patients with frontotemporal dementia (major clinical syndromes, and characteristic features).} \subparagraph{Which genes/gene loci have been associated with frontotemporal dementia?} \subparagraph{Please summarize which major molecular subgroups of frontotemporal dementias can be defined and briefly discuss current knowledge and/or hypotheses on underlying pathomechanisms in the two most common subgroups.} answer here.

\item \paragraph{The maintenance of central and peripheral tolerance is the reason that autoimmune diseases are relatively rare. Please answer the following questions:} \subparagraph{How does central T cell tolerance work (which organ performs T cell education, what is negative and positive selection)?} \subparagraph{What are the mechanisms of peripheral tolerance? Remember, we discussed four of them. Please shortly recapitulate.} \subparagraph{Why does the inflammation in an MS-lesion subside after a while? What mechanisms can dampen an ongoing immune response (if you do not know,
speculate!)?} answer here.

\end{enumerate}

\subsection{All Question - topics}

\subsubsection{To organize}
\begin{enumerate}
\item \paragraph{Characteristics of sleep in mammals: Do they apply to invertebrates?}
behavioural:
Sleeping site 
Quiescence
Body posture
Elevated arousal threshold
Rapid state reversibility
Physiological:
Altered EEG
Reduced muscle tone
Reduced heart rate
Reduced respiration
Reduced body temperature
Regulatory:
Compensatory response to sleep deficit or excess sleep

\item \paragraph{Non-REM-REM sleep}

REM: rapid eye movement. EEG low amplitude, mixed frequency (more similar to wake than to deep sleep EEG). Most prominent in the morning hours.
non-REM: is subdivided into four substages 1-4 in human, deep sleep consists of stages 3 and 4. In deep sleep, he EEG contains prominent slow waves (0.5-4.5 Hz, high amplitude).
cycles: REM sleep occurs every 90-100 mins during sleep (ultradian oscillator origins in the Pons). General term to describe cyclic alternation between REM and non-REM sleep. Healthy people usually start with stage 1, then 2, 3, 4, 2, REM, 2, 3, 4, REM etc.

\item \paragraph{Sleep homeostasis and marker of sleep homeostasis on the sleep EEG}

homeostasis has been defined as the coordinated physiological processes wich maintain most of the steady states in the organism; sleep homeostasis refers to the sleep need in dependance of the time spent awake. Sleep need rises exponentially during wake and declines exponentially during sleep. According to 2-process model of sleep regulation, sleep need is additionally dependant on circadian time.
	NREM-sleep is controlled thalamocortically.
        	Marker of sleep homeostasis: slow-wave activity (power of slow waves rises in recovery sleep after sleep deprivation according to the 2-process model)
        	
\item \paragraph{Endogenous sleep-promoting components: comments}

SCN (superchiasmatic nucleus of hypothalamus)
	clock genes: transcriptional/translational process
	melatonin: built during sleep
	Thalamus: control of NREM-sleep
	Pons: Regulation of REM-sleep
	Potential homeostatic sleep-promoting agents (Experiment: if CSF from a sleep-deprived animal is transferred to a rested animal, the rested animal becomes tired $\rightarrow$ there must be an agent in the CSF that accumulates during wakefulness and makes tired): adenosine, Interleukin-1b, TNFa, GHRH, prostaglandin.

\item \paragraph{Role of thalamus-correlated rhythm in sleep: comments}

Thalamus controls the NREM sleep rhythmus $\rightarrow$ EEG activation / desactivation

\item \paragraph{Circadian pacemakers, entrainment, Zeitgeber, phas-response curve}

Pacemaker: SCN (superchiasmatic nucleus of hypothalamus)
	entrainment means that the ‘inner clock’, located in the SCN, is flexible in the way that it can adapt the phase (example: time-zone flights) and the frequency (example: bunker experiments, where one ‘day’ lasts 25 hours) of the circadian clock.
	Entrainment: via light, signalling from the eye to the SCN (possible photoreceptor: Melanopsin?). In the SCN, per transcription is activated upon light signal.
Phase-response curve: depending on the circadian time, when a light pulse is presented, the phase of the circadian clock is shifted forward or backward. If the light pulse is presented shortly before the active period has started, then the phase is advanced and if the pulse is given shortly after the active period has ended, the phase is delayed (in humans). There is one time point during night when the phase shift swiches from delayed to advanced.

\item \paragraph{Which physiological and endocrine variables in human are frequently used a phase-marker of circadian rhythm}

endocrine: melatonin, adrenal gland (adrenalin, cortison); GHRH (Growth hormone releasing hormon)
	Physiological: body temperature; activity (via activity monitor); alpha-activity in the waking EEG

\item \paragraph{What is the evidence that SCN is a circadian pacemaker}

lesion method $\rightarrow$ arythmicity
         in vitro culture of a single SCN neuron    
         SCN transplant reserves the rhythm
         in vitro SCN

\item \paragraph{Which genes (gene?) are (is) involvedin generation of circadian rhythm}

in mammals: Bmal1 is rhythmically expressed by the SCN. Clock and bmal1(basic helix-loop-helix transcription factor family) build heterodimers. These heterodimers bind to E-boxes of enhancers of the per and cry gene. Per and Cry proteins dimerize outside the nucleus and are phosphorylated. The dimers re-enter the nucleus and downregulate transcription of  clock and bmal1 (negative feedback-loop). Situation is even more complex, also containing positive feedback-loops…)
        in fruit fly Cyc/Clk heterodimers activate per/tim gene transcription. Negative feedback-loop: Per/Tim heterodimers inhibit activation of their own genes via Cyc/Clk.

%%%%%%%%
\item \paragraph{Please describe briefly:} \subparagraph{Some conceptional problems in studying depression from a neuroscientific point of view} 

Depression is a disorder of subjective feeling (translation from first person perspective to third person perspective: epistemic problem). It is very difficult to evaluate the disease scientifically (animal models that are really adequate to depression, they can’t talk to tell their feelings). The definition of psychiatry depends on social values and personal evaluation of suffering, but not depends on the organic disorder. No reliable bjective markers like genetic defects or metabolic disfunctions.

\subparagraph{Some changes of the neurobiology system in a depression state}
Change of HPA-axis (hypothalamus-pituitary-adrenal)
	The prominent mechanism by which the brain reacts to acute and chronic stress is the activation of HPA-axis $\rightarrow$ cortisol levels rise.
		Hypothalamus secretes CRH (corticotropin-releasing hormon) $\rightarrow$ pituitary (hypophysis) secrets adrenalcorticotropin (ACTH) $\rightarrow$ adrenal gland secrets cortisol

Growth hormone is reduced.
	Sleep disorders (EEG!), disturbances of appetite regulation.

different activation of brain areas (activation of medioorbital cortex and ventral anterior cingulate $\rightarrow$ limbic system activated)

%%%%%%%%%
\item \paragraph{Roles of automate as models for computation}

The computational process in neurons can be investigated in neuroinformatics via   automate models (as   compared to the structural process via neuroscience)

\item \paragraph{Automate suitable models for describing the operation of neurons and networks of neurons}

Models for automates are transferable to neurons / neuronal networks:

	Feedforward processor: input $\rightarrow$ blackbox $\rightarrow$ output (without memory)
	In neuron: input correlates to the dentritic input (sum of input signals x weights), output correlates to the axonal output (fire or not fire)

	Finite State Machine: input $\rightarrow$ black-box (which remembers the state in which it is, memory!) $\rightarrow$ output
	In neuron: neurons also can feed back information to build up memory (this model accounts only for short-time memory (seconds)…). Feedback occurs, when axonal output is networked to dendrites of the very same neuron.

	Turing Machine: input $\rightarrow$ black-box containing unbounded memory $\rightarrow$ output
	Church-Turing-Thesis: this machine is able to compute all possible computations. Philosophic question: is the brain’s memory unbounded???
	
	Universal Turing Machine: can simulate the computational process of any Turing Machine, when it knows the protocol of this machine, thus it can also simulate the computational process of a neuron… But the protocol is not known (e.g.: a bee can do computations leading to very various and complex behavior, there is no computational model that could do that with the limited recourses of a few thousand neurons that a bee needs to accomplish it)

	But: Neuronal networks have different weights for the 10 exp 14 axons/dentrite connections. This is not possible to be determined genetically (not enough resources), but dependant on the microenvironment of each neuron in the developmental process. Moreover, they can adapt to the environment by changing those weights or even establishing new connections between axons and dentrites.
	Synaptic release is additionally very versatile, can be modulated chemically, and be inhibitory or excitatory etc.
	
%%%%%%%%
\item \paragraph{Animal models of behaviour allow us to investigate the symptoms of psychiatric disorders such as depression and schizophrenia. Discuss statements, giving examples of some specific model}

You always have to ensure that the animal model is valid. There are different aspects of validity that have to be guaranteed:
Face validity: quantifiable behavior and physiology in the animal model have to be similar to the symptoms in the investigated human illness.
Construct validity: the quantifiable behavior and physiology in the animal model must be a result of the same central state as in the human patient. Theoretical rationale.
Predictive validity: close correspondence between drug actions on behavior and physiology of the animal model and the human patient.
Inter-laboratory validity
Inter-species validity

 Schizophrenia:
Impairment of working memory leads to symptoms like halluzinations: lack of references against associative memories.
	Selective attention is impaired, leading to delusions (misinterpretations), confusion of external and internal stimuli and retreatment to safety (neg. symptoms). 
Specific test: Latent Inhibition (LI) test. LI paradigm: repeated non-reinforced pre-exposure to a stimulus retards subsequent conditioning to that stimulus. This reflects the ability of ‘learning not to attend’. In animals: rats reduce LI when given amphetamine $\rightarrow$ animal model for schizophrenia. Rats that get amphetamine, avoid the box where they got shocked previously in the CAR (conditioned avoidance response) and reduce licking water in the CER (conditioned emotional response) tests compared to control animals due to their impaired LI.


	Depression:
	Animal model of learned helplessness: animals are exposed to negative stimuli and don’t get the possibility to escape. This leads to the ‘learned helplessness’ symptom, especially, if the animals are very young, which means, they give up very quickly and are not able to escape unwanted situations. Learned helplessness can be measured by the escape behavior in a two-way avoidance test. In this test, animals are placed in a shuttle box and exposed to a foot shock. They are allowed to escape to the save compartment of the shuttle box. If they get conditioned for the shock with a tone, starting shortly before the shock, animals learn to escape already at the presentation of the tone. ‘Helpless’ animals are not good in escaping compared to controls.
	Chronic mild stress: Animals are chronically exposed to mild stress like food / water deprivation for some hours, not enough space, over night illumination etc. The loss of pleasure (anhedonia) is measured with the ICSS (intra-cranial self-stimulation), the PRS (progressive reward schedule) or the sucrose preference test.
	Early life stress: Pups are stressed by separating them from mother for several hours per day etc.$\rightarrow$ anhedonia
	
%%%%%%%%
\item \paragraph{What are the physiological correlations of fMRI signal and how does the fMRI signals correlate with neuronal activities
}

The physical correlation of fMRI is neural activity, resulting in an initial (about 0.5-2s) ‘undershoot’ of the proportion of oxygenated hemoglobin, due to the consumption of oxygen for the neural activity. This leads to a reduction of the BOLD (blood oxygen level dependant) signal. Neural activity seems to mediate vasodilation (maybe through to release of NO), leading to an increase of blood flow (after 2-10s), resulting in an increase of the BOLD signal due to the better blood supply. Studies comparing BOLD data with EEG data have shown that the BOLD signal rather reflects the information uptake and processing by neurons than their spiking output measured by EEG.

\item \paragraph{What is the impact of W.I. and what information do we get from it?}

\item \paragraph{What's the significance of blood-brain-barrier in infectious disease of the CNS}

\item \paragraph{ (a) Describe the properties of a finite state machine. (b) Is a single neuron kind of  a finite state machine? Explain your answer. (c) What kinds of model (artificial neurons) do you know about?}

\item \paragraph{Please give a detailed account of the process by which prions upon entering the body reach the CNS.}

\end{enumerate}

\subsubsection{Methods}
\begin{enumerate}
\item \paragraph{Describe methods for measuring motivation, attention and memory in rodents and/or primates. In which neuropsychiatric disease are these beavioural processes disrupted?}

\item \paragraph{ fMRI is routinely used to study the neural processes underlying behavior. Please describe all the procedures necessary for conducting and correctly interpreting an fMRI study, covering the following areas:}
\subparagraph{Data acquisition (what signals are measured in fMRI, and how?)}
\subparagraph{Data analysis (which sequential routines are necessary to detect signal changes in fMRI images, using software packages such as SPM?)}
\subparagraph{Results interpretation (what inferences about neuronal activity can be drawn from fMRI results?} answer.

\end{enumerate}

\subsubsection{Memory}
\begin{enumerate}
\item \paragraph{Please list the different types of long-term memory you know of. Describe their properties in humans, group them according to involved brain structures and give examples of behavioral tests that allow to model these memory types in rodents}
\end{enumerate}

\subsubsection{Sleep}
\begin{enumerate}
\item \paragraph{Sleep regulation in physiological short and long sleepers: Explain the most important principles how sleep and wakefulness are physiologically regulated and how sleep-wake regulation may differ between habitual short and long sleepers.}

\end{enumerate}

\subsubsection{Schizophrenia}
\begin{enumerate}
\item \paragraph{Please describe shortly a neurobiological model of schizophrenia}
\end{enumerate}

	





%%%%%%%%%%%%%%%%%%%%%%%%%%%%%%%%%%%%%%%%%%%%%%%%%%%%%%%%%%%%%%%%%%%%%%%%%%%%%%%%%%%%%%%%%%%%%%%%%%%
\newpage

\section{References}
The pictures used in this summary are from the class slide sets or internet, and belong to their respective owners. In the context of the summary they are used for educational purposes only.

\subsection{Cognitive Neuroscience}
\begin{itemize}
\item Christian C. Ruff and Scott A. Huettel, Chapter 6 - Experimental Methods in Cognitive Neuroscience, In Neuroeconomics (Second Edition), edited by Paul W. Glimcher and Ernst Fehr, Academic Press, San Diego, 2014, Pages 77-108, ISBN 9780124160088, \url{http://dx.doi.org/10.1016/B978-0-12-416008-8.00006-1}
\end{itemize}

\end{document}
