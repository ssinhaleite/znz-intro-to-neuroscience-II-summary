\documentclass[12pt,article,oneside,a4paper]{memoir}

%% Packages
%% ========
\usepackage{graphicx}
\usepackage{titlesec}
\usepackage{wrapfig}

\setcounter{secnumdepth}{4}

\titleformat{\paragraph}
{\normalfont\normalsize\bfseries}{\theparagraph}{1em}{}
\titlespacing*{\paragraph}
{0pt}{3.25ex plus 1ex minus .2ex}{1.5ex plus .2ex}

%% many common packages
\input{commonpackages}

%% Some more packages that you may want to use.  Have a look at the
%% file, and consult the package docs for each.
\input{extrapackages}

%% Our layout configuration.
\input{layoutsetup}

%% Theorem environments.  You will have to adapt this for a German
%% thesis.
\input{theoremsetup}

%% Helpful macros.
\input{macrosetup}

%%page layout settings and listing templates etc.
\input{settings}

\title{\textbf{ZNZ Introduction to Neuroscience II} \\
       Spring 2017\\\normalsize version 1.0}

\author{
	Vanessa Leite
	\vspace{2em}
	\\Repository page: \url{https://github.com/ssinhaleite/znz-intro-to-neuroscience-II-summary}\\
	Contact \href{mailto:vrcleite@gmail.com}{vrcleite@gmail.com} if you have any questions.}
	\thesistype{The Summary of the lectures in 2017}
	\department{ZNZ - Institute of Neuroinformatics, ETH}
	\date{\today}

\begin{document}
\frontmatter


%% DO NOT CHANGE.
\begin{titlingpage}
  \calccentering{\unitlength}
  \begin{adjustwidth*}{\unitlength-24pt}{-\unitlength-24pt}
    \maketitle
  \end{adjustwidth*}
\end{titlingpage}

\mainmatter

%% This change is needed if the article option for the memoir document class
%% is used, in order to count sections (article) as if they were chapters (memoir)
\counterwithout{section}{chapter}

%% Our content

\newpage
\clearpage
\pagenumbering{roman}
\setcounter{tocdepth}{3}
\setcounter{secnumdepth}{2}
\tableofcontents

\clearpage
\pagenumbering{arabic}

\section{Cognitive Neuroscience}
%%-----------------------------------------------------------------------------------------%%
\subsection{Research Methods in Cognitive Neuroscience - prof. Christian Ruff - 20.02.2017}

There are various types of methods to acquire cognitive information. Those methods have different temporal resolution (can be acquired in milisseconds, seconds, hours, day, etc) and spatial resolution (can get information of the brain, maps, columns, layers, cels, synapses, molecules). Also, they have a 3rd component that is correlation x interference.

The main techniques that will be considered in this lecture are \textbf{correlative} and \textbf{causal}.
\subsubsection{Correlative techniques}

Measurements techniques measures changes in brain function while a research participant (human or animal) engages in some cognitive activity. Measuarements techniques are often described as being "correlational" because they can show that signals from a brain region co-occur with a function of interest, but they cannot show that a region is necessary for that function.

\paragraph{(f)MRI} (functional) Magnetic Resonance Imaging

MRI is a non-invasive imagin technique that employs principles of magnetic resonance to visualize different tissue types. Usually, one image is acquired. It is slow (minutes) but accurate (sub-mm spatial resolution).

The fMRI employs special sequences that are i) sensitive to blood oxygenation and ii) fast to acquire (whole brain in ~2-3 seconds). Numerous images are recorded and represent timecourse of blood oxygenation durig experimental task.

How the magnetic resonance works:
\begin{itemize}
\item Place an object (brain) in a strong magnetic field
\subitem protons in the body have spins with a specific orientation and frequency, when the body is inside an MRI scanner, the protons align with the direction of the magnetic field.
\item Apply radio waves
\subitem radio frequence pulses with the appropriate frequency (\textbf{Larmor frequency}) change the orientation of the spins as the protons absorb the energy. When the pulse is turned off, the protons return to their original orientations, this process is called "relaxation", and during the (longitudinal $\rightarrow$ T1 and transverse $\rightarrow$ T2 \ref{fig:orientation-relaxation}) relaxation, the protons emit energy in the form of radio waves.
\item Measure radio waves emitted by object
\subitem T1 is time constante of how quickly the protons realign with the magnetic field, for instance, CSF has low signal (dark) and fat has high signal (bright).
\subitem T2 is time constante of how quickly the protons emit energy when recovering to equilibrium, for instance, fat has low signal (dark) and CSF has high sinal (bright).
\end{itemize}

\begin{figure}
  \centering
  \includegraphics[width=0.3\textwidth]{imgs/mri-longitudinal-transverse.png}
  \caption{Orientation of the relaxation}
  \label{fig:orientation-relaxation}
\end{figure}

\textbf{The human scanners have a strong static magnetic field (around 1.5-7 Tesla)}

The \textbf{T1 fMRI} images are structural images with high spatial resolution ( less than 1 mm) and accurately distinguish different types of tissue. The \textbf{T2 fMRI} images have lower spatial resolution (2-3 mm) and relate changes in MR-signal to an experimental manipulation. Timeseries represents a large number of signals that are acquired in temporal order at a specific rate.

Some terminology:
\begin{itemize}
\item subjects: the item that will be scanned
\item sessions: each time that the subject is inside of the scanner
\item runs: all the images generated in one section for the whole subject. One complete scan of the subject is obtained in one single run.
\item volume: the 3d images generated from one single run
\item slices: each section of the volume is called slice.
\item voxel: each single unit information in a slice
\end{itemize}

\paragraph{the BOLD (Blood Oxygenation Level Dependent) contrast} measures inhomogeneities in the magnetic fiels (T2) due to changes in the level of O\textsubscript{2} in the blood. This way, fMRI measures neural activity indirectly via BOLD signal.

The oxygenated hemoglobine is diamagnetic (non magnetic) and produce no signal loss, however, the deoxygenated hemoglobine is paramagnetic (magnetic) and then produce a signal loss. When a specific region of the cortex increases its activity in response to a task, the extraction fraction of oxygen from the local capillaries leads to an initial drop in oxygenated hemoglobine (oxyHb) and an increase in local carbon dioxide (CO\textsubscript{2}) and deoxygenated hemoglobine (deoxyHb). 

Some properties of the BOLD signal:
\begin{itemize}
\item peaks 4-6 seconds after neural activity (delay)
\item back to baseline after approx. 30 secs
\item can vary in precise shape between regions and subjects
\item often shows undershoot and sometimes shows initial undershoot
\end{itemize}

Due to an over-compensatory increase of rCBF( regional Cerebral Blood Flow), increased neural activity can decreases the relative amount of deoxyHB. This is called neuro-vascular coupling and it is an active area of research.

At present, the safest assumption is that BOLD relates to both spiking output and excitatory postsynaptic activity in neurons. Inhibitory activity is not assumed to lead to BOLD increases.

BOLD signal is not an absolute measure, but differs from session to session due to differences in scanner sensitivity, subject, etc. This way, BOLD signal needs to be compared between different conditions within the same experiment to infer BOLD changes (increase or decrease) due to neural process of interest P. \textbf{[Task with P] - [control task without P] = P}

For this "subtraction approach", there are assumptions of "pure insertion": i) cognitive (and neural) processes can be added to others without changing them and ii) changed behavior (and brain activity) reflects only added process.

\paragraph{Design of fMRI experiments} they can be block-designs or event-related designs. In the block-designs, we measure constant BOLD response to a \textbf{series of stimuli}. In the event-related designs, we measure BOLD response to \textbf{each stimulus}.

\begin{itemize}
\item Block-designs:
\subitem higher statistical sensitivity for detecting effects.
\subitem some psychological process have to/may be better in blocks, for instance, if there is difficult to switch between states or to reduce surprise effects.
\item Event-related designs:
\subitem randomised trial order
\subitem some events can not be blocked due to stimulus context.
\end{itemize}

In the fMRI designs, the predictions for BOLD signal can be categorical (identify classes), parametric (stimuli rotating, expanding) or model-based (check correlation between some model and BOLD signals). One can use a factorial desing and combine different factors (categorical, parametric and model-based) within one study, allowing study of context-dependent neural responses (can show a failures of pure insertion).

Sometimes, the resolution of the experiment is smaller than the MR image resolution, for this, we can consider the MVPA (multivariate activity pattern) or repetition suppression instead of the univariate signal in each voxel. The MVPA assumes that the signal in each voxel represents mixture of neuronal populations specialised for different features. Note that the pattern of increases and decreases may hence reliably differentiate different stimuli, even if each voxel by itself does not. The repetition suppression... ?

\paragraph{ Analysis of fMRI experiments: SPM (Statistical Parametric Mapping) } it is a statistical approach instantiated in the most widely used software package for fMRI anaysis, it is implemented in MATLAB and it is open source. Allows standardised detection of \textbf{regional activity changes} in each voxel, associated with task parameters.

\begin{itemize}
\item Preprocessing
\subitem Realignment (= registration): fix small head movements, assumes that the shape of the brain does not change.
\subitem Spatial Normalisation: increase sensitivity with more subjects, extrapolate findings to the whole population and make results from different studies comparable (all in the same 'coordinate system')
\subitem Smoothing: increase signal to noise, improve inter-subject averaging. In SPM, smoothing is a convolution with a Gaussian kernel. After smoothing, each voxel becomes the result of applying a weighted region of interest.
\item Model estimation
\subitem parameters estimation from GLM of voxel timeseries
\item Contrasts and SPMs
\subitem statistical inference
\end{itemize}


\subsubsection{Causal Techniques}
Manipulation techniques examine how perturbations of the brain’s function - either by
transiently changing neuronal firing rates or neurotransmitter levels or by permanently damaging tissue - change cognitive functions or behavior, and they are called causal approaches.

\paragraph{History of causal techniques: lesion studies } Investigates causal brain-behavior relations (consequence of focal head wounds). They are measure in hypothesis-guided fashion cognitive and behavioral deficits of brain-lesioned patients.

\begin{itemize}
\item Advantages:
\subitem truly causal
\subitem can revel unexpected findings
\subitem medical relevance
\item Limitations:
\subitem lesions often diffuse
\subitem reorganization?
\subitem medication/treatments?
\subitem population inferences?
\end{itemize}

\paragraph{History of causal techniques: invasive stimulation} Fritsch \& Hitzig in 1870 electrically stimulated an awake dog's brain via inserted wires and caused involuntary movements. The experiments were done in Frisch's home as the University would not allow the experiments. It was the first study to show that externally supplied electricity triggers neural function.

\begin{itemize}
\item Stimulation pre-surgery to determine tissue function
\item Systematic "cartography" of brain-behavior (humunsculus)
\item Only possible in patients $\rightarrow$ invasive!
\end{itemize}

\paragraph{Causal Methods: Non-invasive stimulation} overcome need for invasive pre-surgical diagnosis, allow systematic testing of excitability and integrity of motor tracts, modulate functio of the cortex for clinical purposes. Examples: TMS (transcranial magnetic stimulation) and tES (transcranial electric stimulation).

\paragraph{TMS: Biophysics} From Faraday's Law: a time-varying magnetic field induces an electric field in a conducting material. The induced electric field results in a measurable voltage and current flow. For TMS, the conducting material is the brain and the induced current activates neurons \ref{fig:tms}.

\begin{figure}
  \centering
  \includegraphics[width=\textwidth]{imgs/tms.png}
  \caption{Basic principle of TMS}
  \label{fig:tms}
\end{figure}

Brain is not homogenous conductor, but mixture of different materials (skull, liquor,
gray and white matter) that have different conductivities. So, how does the electric field affect neurons?

\begin{itemize}
\item Activation of nerve fibre determined by the spatial derivative of the field component parallel to the fibre (the activating function)
\item Nerve bends are low-threshold points and therefore easiest to stimulate; the stronger the field, the stronger the stimulation.
\item Cortical neurons have numerous bends, terminals and branches; these will all be affected most at the location where the induced fiels is maximal.
\item The likely stimulation point in the cortex for random orientation of bends etc, is the field maximum.
\end{itemize}

\paragraph{TMS: Neurophysiology and types of stimulation protocols} TMS pulses of hand representation in M1 (motor cortex area 1) cause measurable twitches in hand muscles. Non-motor cortical areas require different behavioral indices.

\begin{itemize}
\item repetitive TMS protocols apply series of TMS pulses to reduce or increase cortical excitability in the aftermath of stimulation.
\item The most established and widely used protocol consists of several hundred pulses at 1Hz, which reduces Motor Evoked Potentials (MEP) size for roughly the same duration as the length of rTMS application.
\item High-frequency rTMS (5-20Hz) has been claimed to enhance MEP size but effects
vary strongly between and within individuals.
\item Theta-burst TMS mimicks the theta rhythm that is expressed during memory
storage.
\subitem Theta-burst Stimulation (TBS) has been shown to lead to reductions (for continuous TBS) or enhancements (for intermittent TBS) of MEP size lasting more than 30 minutes.
\item Need to be sure that the brain area is "at rest" during the stimulation (voluntary movements/contraction can reverse/abolishes the effects.
\end{itemize}

\paragraph{tES: Biophysics} Two poles with electric potential difference (charge) connected through a conductive medium. The connection leads to discharge by electric current: negatively charged ions (anions) flow to anode and positively charged ions (cathions) flow to cathode \ref{fig:tes}. 

\begin{itemize}
\item About 50\% of the applied current reaches the cortex, the rest is shunted by the skull.
\end{itemize}

\begin{figure}
  \centering
  \includegraphics[width=0.3\textwidth]{imgs/tes.png}
  \caption{Basic principle of tES}
  \label{fig:tes}
\end{figure}

\begin{table}[h]
  \begin{tabular}{ l |  p{6cm} |  p{6cm} }
    \hline
     & TMS & tES \\ \hline
    current & Induction of current by magnetic field & direct application of current \\ \hline
    area & relatively focal & not very focal \\ \hline
    induce & precisely timed burst of action potentials (+ physiological effects) & does not induce time-locked neural activity but modulates natural activity \\ \hline
	threshold & suprathreshold stimulation & subthreshold stimulation \\ \hline
    effects & phenomenological & physiological \\
    \hline
  \end{tabular}
  \caption{Differences between TMS and tES}
\end{table}

%%-----------------------------------------------------------------------------------------%%

\subsection{Perception and Attention - prof. D. Kiper - 27.02.2017}

\subsubsection{Perception}
Perception is not a passive process (sensation is the passive process). Perception is the process by which people select, organize, interpret and respond to information from the world around them. It is selection and organization of environmental stimuli to provide meaningful experiences (we are not passive analyzers). The particular perception of itself is called proprioception. If you do not receive any stimulus (for instance, in a depravation tank), your brain creates it.

The perceptual process consists of six stages:\ref{fig:perception-stages}.
(1-2)People receive stimuli from the enviroment throught their senses.
(3) When the senses are activated, starts the perceptual selection. The perceptual selection is a filter, that allow us to deal with the most important matter. This is called \textbf{selective screening}: our system eliminates some factors because they are not important for us to be aware of. The "most important" is based on influencing factors that can be external or internal, as listed in Table \ref{table:factors-perception}.
(4) When the most important stimuli is identified, starts the perceptual organization process by which people group the stimuli in recognizable patterns, listed in Table  \ref{table:perceptual-organization}.
(5-6) Then, we use the information received to interpret and respond to the stimuli. Perception is noisy and makes mistakes - very complex system. The most common types of perceptual errors are accuracy in judgment (main types listed in Table \ref{table:accuracy-judgment}), perceptual defence, stereotyping, halo effect, projection, role of culture, etc.

\begin{table}
  \begin{tabular}{ p{13cm} |  p{2cm} }
    \hline
    Internal & External \\ \hline
    personality -  strong factor & size \\ \hline
    learning and perceptual sets - expectation of particular interpretation based on past experiences with the same or similar objects & intensity \\ \hline
    motivation - the needs and desires at any particular time can influence perception (when you are hungry you can perceive a food as more delicious than when you are not hungry) & contrast \\ \hline
	 & motion \\ \hline
 	 & repetition \\ \hline
 	 & novelty \\ \hline
     & familiarity \\
    \hline
  \end{tabular}
  \caption{Externals and internal influencing perception}
  \label{table:factors-perception}
\end{table}

\begin{table}
  \begin{tabular}{ l |  p{12.5cm} }
    \hline
    continuity & lines are seen as following the smoothest path \\ \hline
    closure & tendency to complete an object and perceive it as a constant \\ \hline
    color constancy & your brain starts to iluminate the enviroment making you think the color is the same in different enviroments \\
    \hline
  \end{tabular}
  \caption{Examples of perceptual organization}
  \label{table:perceptual-organization}
\end{table}

\begin{figure}
  \centering
  \includegraphics[width=0.5\textwidth]{imgs/perception-stages.png}
  \caption{Stages of perception}
  \label{fig:perception-stages}
\end{figure}

\begin{table}
  \begin{tabular}{ p{5cm} |  p{10cm} }
    \hline
    similarity error & assuming that people are similar to us and then, will behave like us \\ \hline
    contrast error & comparing people to others rather than to some absolut standard \\ \hline
    overweighting of negative information & tendency to overreact to something negative \\ \hline
    race, age, and gender bias & tendency to be more or less positive based on one's race, age or sex \\ \hline
	first impression error & forming first impressions that are resistant to change \\ 
    \hline
  \end{tabular}
  \caption{Examples of perceptual organization}
  \label{table:accuracy-judgment}
\end{table}

\subsubsection{Attention}
Attention is the taking possession of the mind, in clear and vivid form, of one out of what sem several simultaneous possible objects or trains of thought. It is the focalization, concentration of consciouness. It implies withdraw from some things in order to deal effectively with others.

Attention can change rapidly, switching from one thing to another. It can be steered by our intentions ("top-down"), as when we look for a particular face in a crowd, or it
can be steered by features of objects in the world ("bottom-up"), as when our attention is grabbed by a police car's flashing lights in our rearview mirror.

\begin{figure}[h]
  \centering
  \includegraphics[width=0.5\textwidth]{imgs/attention.png}
  \caption{A definition of attention by exclusion}
  \label{fig:attention}
\end{figure}

There are three main types od attention: overt, covert and feature attention. It one of them are explained in Table \ref{table:types-attention}

\begin{table}
  \begin{tabular}{ p{5cm} |  p{10cm} }
    \hline
    overt attention & selectively attending to an item or location over others by moving the eyes to point in that direction \\ \hline
    covert attention & related with spotlight of attention, you can't pay attention to many things, this way when you pay attention in one thing, your capacity to pay attention in others decay. \\ \hline
    feature attention & shadowing tasks, we can distinguish two mixed texts by focusing our attention on cues such as type style \\ 
    \hline
  \end{tabular}
  \caption{Types of attention}
  \label{table:types-attention}
\end{table}

Slide \#18: frontal lobe $\rightarrow$ saccade eyes movements (overt attention); parieral lobe $\rightarrow$ covert attention.

\paragraph{Broadbent's filter theory - early selection} Broadbent (1958) argued that information from all of the stimuli presented at any given time enters a sensory buffer.  One of the inputs is then selected on the basis of its physical characteristics for further processing by being allowed to pass through a filter, all the others are lost.  Broadbent assumed that the filter rejected the non-shadowed or unattended message at an early stage of processing. It takes time to shift attention.

\paragraph{Treisman's theory - attenuation} physical characteristics are used to select one information for full processing but other messages are given partial processing. 

\paragraph{Deutsch and Deutsch theory - late selection} All information get through but attention filters only act after meaning is analyzed. 
%\paragraph{Slide \#19:} one "phrase(?)" in each ear.

Early or late experiments: Treisman and Geffen (1967) set about to test whether the filter was early or late in the processing stream. They had subjects shadow a message on one ear, and tap whenever they heard a certain word in either ear. When the key word appeared in the attended ear, subjects tapped 87\% of the time, but when the key word appeared in the unattended ear, subjects tapped 8\% of the time. This indicates that early selection is occurring.

A lot of simple experiments can be made to evaluate attention, for instance, pop out. It is easy to "find" elements based on one feature, more features (conjunction) $\rightarrow$ takes longer to find it. \textbf{pop out experiments can prove sinestesia because one can use one specific feature such as color, to find elements easily}. The more popout, more later selection (selection after meaning). Popout can be trained!

\paragraph{Effects in early visual cortex} Desimone experiments. MTA (medial temporal area) $\rightarrow$ prefered direction of motion $\uparrow$. Attention is like a weight to fire particular neurons.

\paragraph{Selective and divided attention} Attention is studied by presenting participants with two or more stimuli at the same time, this is called dual-task performance. In selective (focused) attention tasks, people are instructed to respond to one input only. In divided attention tasks, people are asked to process and respond to more than one input.

\paragraph{Gorila video:} innatentional blindness

%%-----------------------------------------------------------------------------------------%%
\subsection{Decision Making - 06.03.2017}
What do we know about how the brain computes stimulus values at the time of decision-making? Recent meta-analyses: Positive effects of sv on BOLD are higher than negative effects. Also, the decision $>$ outcome.

Becker-DeGroot-Marshack auctions to measure goal values.

\paragraph{Appetitive and aversive value}
\paragraph{Multivoxel pattern analysis} PFC, LPFC, Parietal region
\paragraph{Deciding for self vs others}
\paragraph{Executed:} decision for others
\paragraph{Model:} I'm not choosing for me but I'm still modeling. What I would choose for me?
\paragraph{vmPFC} = ventral medium pre frontal cortex
\paragraph{Areas on the brain that are not core for the type of the test but still may be part of the decision}
\paragraph{comparison:} costs x bemefits
\subparagraph{delay costs}
\subparagraph{effort costs}
\paragraph{OFC} = orbital frontal cortex
\paragraph{ACC} = anterior cingulate cortex

\paragraph{Decision} related with learning: depends of what you have learned
\paragraph{Value (as in food) is not purely motor or sensory}
\subparagraph{how much people like chocolate after each square}
\paragraph{All regions (on slide) receive input from VTA - dopaminergic neurons}

\paragraph{long term memory}
\paragraph{conditioning is impaired in amygdala but not in hippocampus (can not hold facts)}
\paragraph{fear impairement (patient SM)}
\paragraph{Similarity = shape, etc}
\paragraph{positive and negative prediction errors}
\paragraph{learning is proportional to prediction errors}

\paragraph{dopamine neurons:} action potentials fires when unpredicted reward occurs; fewer action potentials than "normal" when no reward occurs.

\paragraph{learning from others}
\paragraph{selfish = other = dorsal region}

\paragraph{Representation $\rightarrow$ move from objective to subjective}
\paragraph{dopamine neurons may represent values in a objective way}
\subparagraph{the same value has lower reward value if the subject needs to wait more for it}

%%-----------------------------------------------------------------------------------------%%
\subsection{Emotion - 13.03.2017}
\paragraph{Find methods to measure emotions}
\paragraph{What is an emotion?} emotions are short lasting x mood are long lasting
\paragraph{emotion as everyday feeling}
\subparagraph{from psycology}
\subparagraph{from neuroscience:} neurons controls behaviours
\paragraph{emotion as feelings:} qualia $\rightarrow$ non communicative; verbal expression
\paragraph{comparative:} emotion is shared between humans and animals
\paragraph{fear conditionining x fear expression x fear prosody}

\paragraph{common sense:} we do things because we feel
\paragraph{Physiological theory:} we feel because we sense things (attribution: think about the senses)
\paragraph{Facial feedback:} active some muscles leads to feeling
\paragraph{Appraisal theory:} automated no-consciouness that leads sensation to feeling
\paragraph{Affect theory:} framework to talk about subject feelings

\subsubsection{ Decision Theoretical View}
\paragraph{emotions are actions:} adaptive to achieve goals.
\paragraph{Attention on two things:} goals and controls algorithms (that decide the actions)

\paragraph{rats cannot learn shock from taste, but they learn from be sick}

\paragraph{many "algorithms" for many behaviours}

\paragraph{game:} go pick a "piece" fast enough to not be hitted by "rock"

\paragraph{hippocampus lesion:} difficult approach
\paragraph{amygdala lesion:} return approach

\paragraph{Approximating algorithm: } optimal is too much complicated; why more time to go when is more difficult?

\paragraph{meta control:} how we decide what we decide?

\paragraph{Why study humans?} communication of emotions

\subsubsection{Clinical application}
\paragraph{fear extinction:} you can train the extinction of fear. However, the complete memory is not "removed".
\paragraph{memory reconsolidation:} memory consolidation needs proteins.
\paragraph{fear erasure:} where in the brain the fear seats? amygdala?

\paragraph{Multivariate fMRI analysis}
\subparagraph{benzodiapezines:} ansiolitics
\subparagraph{lesion on hippocampus make rats stay longer on open arm}
\paragraph{slide: benzo:} red line; placebo: black line (human analogon)

%%-----------------------------------------------------------------------------------------%%
\subsection{Memory - Prof. Katharina Henke - 20.03.2017}
\subsubsection{Methods and Memory}
\paragraph{fMRI}
\paragraph{what structures support learning / tasks activities / retrieval activity?}
\paragraph{Discover silent process (animals and humans)}
\paragraph{Read the unconscionous mind using fMRI}
\paragraph{Identify people with real memory problem}

\paragraph{To identify areas on the brain essential for a task is necessary to analyze people with brain damage (in a specific area)}

\paragraph{PET - Tomography:} used on study of learning and memory.
\paragraph{Patient:} 23 years old - psycological trauma
\paragraph{No brain damage - no structural damage}
\paragraph{Glucose pet indicates if area is working - functional amnesia}

\subsection{Learning during sleep - EEG}
\paragraph{slowing sleep - delta waves}

\subsection{Multistore model}
\paragraph{Sensory memory (up to 1s) $\rightarrow$ STM (15-20s to 1 min) $\rightarrow$ LTM (minutes and longer)}
\paragraph{Hippocampus:} episodic memory $\rightarrow$ memory for personal episodies - autobiographical, what, where, when
\paragraph{Hippocampus - very vulnerable $\rightarrow$ lots of diseases by damage on hippocampus}
\paragraph{ovulation invreases hyppocampus activity as in menstruation (?)}
\subsection{Declarative (explicit) memory }
\paragraph{Semantic memory:} independent of hippocampus
\subparagraph{through repetition is possible to learning new things using the neocortex and not passing throught the hippocampus}
\subsection{Nondeclarative (implicit) memory - unconsciouness}
\paragraph{Procedural learning:} HM patient still able to ride a bike.
\paragraph{Priming:} tendency to process some perception in the same way. When you see a complex picture you need a longer time to identify things then in a second/third time.

\paragraph{1990-2000}
\paragraph{Healthy patient do not show hippocampus activity during learning / memory}
\paragraph{Hippocampus is speciallized in associate memory}
\subparagraph{anterior part:} semantic and temporal association
\subparagraph{posterior part:} sensory and spatial
\paragraph{Parahippocampus gyrus:} individual objects - important for priming

\subsubsection{Memory $\rightarrow$ new division $\rightarrow$ Model of Katharina (2010)}
\paragraph{Encoding:} rapid / slow
\paragraph{Association:} flexible / rigid / no association (unique item)
\paragraph{Learning in sleep when the second word (?) is received in a moment where the neuron is polarized}

%%-----------------------------------------------------------------------------------------%%
\subsection{Body Perception - 27.03.2017}
\subsubsection{Phantomology}
\paragraph{hemiphantom:} one side
\paragraph{phenomenon:} studied since 1510 (Paré) but term "phantom limb" is from 1829 (Mitchell).
\paragraph{to read:} Finger and thustwitt, 2003, Neurosurgery 52.
\paragraph{Sensetion is in the brain, not in the limb - Descartes}
\paragraph{experiment with adult monkeys:} lose a finger and the neighbors fingers took the "empty" area of the finger on the brain - cortical plasticity
\paragraph{Maladaptive model:} pain is the price for plasticity
\paragraph{It is not very clear what is guided by phantom limb and why some people feel phantom limb pain}
\paragraph{Negative phantom limb:} there is a limb but no "connections", also called xenomelia
\paragraph{Phantom limb:} approximately 50\% believes the phantom limb is still there even when a pile of books is placed at the same location of the limb. 50\% perceive that the phantom limb is in his/her head.
\paragraph{research relevance:} visual-somesthetic interactions.
\paragraph{clinical relevance:} protesis.

\paragraph{You do not need to physically lose a limb to experience phantom limb. The visual system can stop the feeling of phantom limb.}

\paragraph{supernumary (?)}
\paragraph{Even if the phantom limb sensations disappears the phantom limb pain can still holds}
\paragraph{Phantom limb can be feel even in congenital limb experiencies. Scientific fact: does not mean remembering is not important, also does not mean "body schema" is "inate"}
\paragraph{Phantom moving is like imagine moving a limb - bilateral activation on brain}

\section{Clinical Neuroscience}

%%-----------------------------------------------------------------------------------------%%
\subsection{Neurology: Ophthalmology, Otology, Epileptology and Parkinson - 03.04.2017}
\subsubsection{Neuro ophtamology}
\paragraph{Clinical eye testing}  
\begin{itemize}
\item moviments
\item smooth pursuit
\item saccadic eye moviemnt - fast eye moviment
\subitem disorders: velocity, metrics (can "pass" the point), latency
\item nystagmus
\subitem primary gaze: indicative of cerebellum loss $\rightarrow$ gaze holding and rebound
\subitem vestibulo occular reflex: the head moves and the eyes move together and then, after a while the eyes turn to the object-goal position
\end{itemize}

\paragraph{Clinical balance testing}
\begin{itemize}
\item spontaneous nystagmus: eyes drift to the side of the loss.
\item head impulse test: negative $\rightarrow$ eye stays in position; positive: $\rightarrow$ eyes follow head and then turn.
\item vertical: "close" one eye and the other stays in the same position
\item dynamic: see letters moving the head
\end{itemize}

\paragraph{vertical occular deviation}
\paragraph{dynamic visual acuity}

\paragraph{sensorymotor balance:} romberg test: close eyes and balance is lost.
\paragraph{caloric testing}
\paragraph{bimallolar (?) vibration sense}

\subsubsection{Epilepsy}
\paragraph{What is epilepsy?} brain disorder, chronic condition
\paragraph{What is a seizure?} temporary disruption of normal brain function.
\begin{itemize}
\item eye moviment, body tension
\item depending on the anatomical location of the seizure, can be all body or only an arm for instance
\end{itemize}
\paragraph{EEG} pyramidal cells in the cortex: sleep and close eyes produce high variations.
\paragraph{Seizures does not mean epilepsy.} Some seizures can be provocated.

\paragraph{Classification of seizures:}
\begin{itemize}
\item focal: part of the brain
\item generalized: whole brain; always without consciouness.
\end{itemize}

\paragraph{EEG measures synchrony post-synaptic potential}

\paragraph{Hypersynchronization:} spikes and sharp edges (seizure)
\paragraph{generalized seizures:} neurons die

\paragraph{Treatment:}
\begin{itemize}
\item surgery: small area
\item pharmacotherapy
\item disease specific treatment
\end{itemize}

\subsubsection{Parkinson}
Parkinson is a syndrome not a disease
\paragraph{Variety of symptons:} always related with a lack of dopamine. However, too much dopamine is also a disturb but it is not parkinson.
\begin{itemize}
\item slow movements
\item tremor
\item stiffness
\item postural instability
\end{itemize}

\paragraph{Parkinson's disease:} the common reason for Parkinson's syndrome. Second most common degenerative disorder, just lose to alzheimer.

\paragraph{L-dopa is the most effective therapy to Parkinson's disease} it is like insuline for diabetes. However it is a symptomatic-therapy not a cause-therapy.
\paragraph{Manage the quantity of dopamine is not easy}

%%-----------------------------------------------------------------------------------------%%
\subsection{Neurology: Multiple Sclerosis, Neuromuscular, Stroke and Neuropsychology - 10.04.2017}

\subsubsection{Multiple Sclerosis}
\begin{itemize}
\item Most frequent CNS disease among young adults (20-40y)
\item Relapses manifestation: coming and go
\item distructive disease
\item inflamation and other shrink (?) on brain
\item enviromental effects: more common abouve equator line
\item diagnostic depend of space and time
\subitem space: not only one area of CNS lesion
\subitem time: because come and go
\item PPMS: primary progressive multiple scelerosis: CSF
\subitem punctions of CSF: if you find bacteria (?) in the CSF and serum it is a general inflamation, if you find only in the CSF is MS.
\item scale 0 to 10; 10 means death
\item MS became a treatable disease in 90's
\item de-myelination is necessary to diagnose MS: you need MRI information and additional signs.
\end{itemize}

\subsubsection{Neuromuscular disorders}
\begin{itemize}
\item diagnosis: medial (?) history, neurological examination, lab tests
\item myotonic reaction: ?
\end{itemize}

\subsubsection{Stroke} 
Any acute neurological sympton is a stroke unless proven otherwise
\paragraph{Stroke is not a seizure, and also is not a migraine}
\paragraph{two type of strokes:} ischemia and hemorrhage
\begin{itemize}
\item ischemia: (80\%) $\rightarrow$ block of a vessel - recanalization
\item hemorrhage: (20\%) blood pressure control / hematoma evacuation, reduction of intracranial pressure - cateter
\end{itemize}
\paragraph{cells in penumbra are potentially salvable} penumbra is an area around the vessel.
After 270 mins of a stroke, recanalization can lead to  hemorrhagia.
\paragraph{prognosis:} death + dependent (approx. 90\% hemorragic and approx. 40\% of ischemia)

\subsubsection{Neuropsychological disorders }
\begin{itemize}
\item aphasia (language disorders)
\subitem broca $\rightarrow$ patient do not speak but understand
\subitem wesnick $\rightarrow$ spontaneous speech is fluent but it is impossible to repeat phrases.
\item alexia $\rightarrow$ can write but can not read
\item apraxia $\rightarrow$ can use tools but can not immitate their uses
\item anosognosia $\rightarrow$ no notion of its own deficience
\end{itemize}

%%-----------------------------------------------------------------------------------------%%
\subsection{Spinal Cord Injury - 25.04.2017 - Dr. Martin Schubert and Dr. Armin Curt}

\paragraph{Clinical exam (padronized)} international grade of spinal cord injury (from A to E, where E is normal)

\subsubsection{Neuro urology}
Disfunction on urinary tract - spinal cord injury
\paragraph{paper to read:} ``Lower urinary trait disfunction''.

\paragraph{mickey mouse kidney (?)} leads to incontinency

\paragraph{disreflexia}  above T6, vaso constriction. Increase on blood pressure leads to (parada cardiaca)

\subsection{Epilepsy}
 
%%-----------------------------------------------------------------------------------------%%
\subsection{Depression}

\subsection{Schizophrenia}

%%-----------------------------------------------------------------------------------------%%
\subsection{Addiction Clinics - 15.05.2017 - Dr. Marcus}
\paragraph{Global Burden of Disease (2010)} Alcohol + ilicit drugs $\rightarrow$ increased 5.4\%. Tobacco $\rightarrow$ 3.7\%.

\paragraph{Mortality in Switzerland} 1. tobacco >>> 2. alcohol >> 3. suicid >> 4. traffic accidents

\paragraph{Drugs most harm} 1. alcohol >>> 2. heroin >>> 3. cocaine

Intoxication, dependence

\paragraph{Dependence Syndrome (six signs)} Need to present at least 3 signs during a month.
\begin{itemize}
\item Strong desire or compulsion
\item Impaired capacity to control substance taking behaviour
\item A physiological withdraw state with reduce or cease (?)
\item Evidence of tolerance
\item Preocupation with substance use by alternative pleasures (to not reduce or give up)
\item Persistent use despite harmful consequence
\end{itemize}

\paragraph{Dependence} Environmental and genetic factors. First use is not enough to become dependent

Graph: probability to get dependent vs time (nicotine, cocaine, alcohol, cannabis)

\paragraph{Stigmas of Alcohol, Depression and Schizophrenia} to read: ``Addiction is a Brain disease, and it matters''

\paragraph{Alcohol risks for health consequences} low $\rightarrow$ high
\paragraph{Treatment goals} preservation / restoration of health and social integration

Pharmacotherapy, Psycotherapy, Social Support.

+ 50\% abstinence
- 50\% reduction

\paragraph{Heroin} pain killer
open drug scene in Zurich (1980/1990)

\paragraph{Not drug free, but social acceptable}
\begin{itemize}
\item prevention
\item therapy
\item harm reduction
\item regression
\end{itemize}

33k people died of opioid in USA (when??)

\paragraph{Cocaine} issue in Zurich (top 3 cities in europe)
\paragraph{Patient interview} cocaine addicted

\paragraph{Recreational x addicted users} start of use drugs generate positive effects.

graph of positive/negative effects vs use

\paragraph{Berridge (?) et al, 2009} ``want'' vs ``like'' dissociates over time
\paragraph{want} dopamine
\paragraph{like} endogenous opioids

\paragraph{Addiction as pathological learning and memory} reward, memory, drive and control

\paragraph{Molecular target of drugs}
\begin{itemize}
\item I - binds to G-protein - metabotropic - LSD
\item II - binds to ionotropic receptors
\item III - interacts with monoamine transporters (GABA receptors)
\end{itemize}

\paragraph{Addiction increase dopamine levels} mesolimbic projection (VTA - ventral tegmental area)

\paragraph{faster uptake} faster drug effect (reward)
\paragraph{place/people as trigger} if drig do not ``come'' the brain seeks the drug $\rightarrow$ absence of dopamine

\paragraph{Serotonin and addiction} without dopamine, one can still get addicted if there is serotonin in the body.


\subsection{Addiction in Society}

%%-----------------------------------------------------------------------------------------%%
\subsection{Neurosurgery - 22.05.2017 - Jorn Fierstra}
\paragraph{Highly Specialized Medicine} most brain surgery but also spine surgery

UZH/ETH is an outstanding place

\subsubsection{Brain pathologies}
\paragraph{Aneurism} determine if must be treated. 50\% of bleeding, some people do not need surgery.
\paragraph{bypass} 80 years neurosurgery ?
\paragraph{Elana (?) laser system bypass} invented in zurich

\subsubsection{Spine Pathologies}
\paragraph{2nd image} neck
\paragraph{3rd image} tumor spinal cord

\subsubsection{Brain tumor} 30 cases per year
\paragraph{1st image} tumor from dura
\paragraph{3rd image} glioma (butterfly)
\paragraph{4th image} surgery, microscope, neural navigation
\paragraph{5th image} tumor that do not ``appear'', glowing tumor

Intraoperative MRI

\paragraph{Neurosurgery challenges} better imaging methods

BOLD fMRI
Respiract: changes in CO2 to contract/dilate veins in brain

Neurovascular uncoupling
finger tapping - image effects
sensitive to threshold


%%%%%%%%%%%%%%%%%%%%%%%%%%%%%%%%%%%%%%%%%%%%%%%%%%%%%%%%%%%%%%%%%%%%%%%%%%%%%%%%%%%%%%%%%%%%%%%%%%%
\newpage


\section{Previous exams}

\subsubsection{2013}
\begin{enumerate}
\item \paragraph{FMRI}
\item \paragraph{Microglia}
\item \paragraph{Circadian Rhythms}
\item \paragraph{Motor Learning}
\item \paragraph{Parkinson}
\item \paragraph{Multiple Sclerosis}
\end{enumerate}

\subsubsection{2011}
\begin{enumerate}
\item \paragraph{ fMRI is routinely used to study the neural processes underlying behavior. Please describe all the procedures necessary for conducting and correctly interpreting an fMRI study, covering the following areas:}
\subparagraph{Data acquisition (what signals are measured in fMRI, and how?)}
\subparagraph{Data analysis (which sequential routines are necessary to detect signal changes in fMRI images, using software packages such as SPM?)}
\subparagraph{Results interpretation (what inferences about neuronal activity can be drawn from fMRI results?}

answer here.

\item \paragraph{Please list the different types of long-term memory you know of. Describe their properties in humans, group them according to involved brain structures and give examples of behavioral tests that allow to model these memory types in rodents}

\item \paragraph{Sleep regulation in physiological short and long sleepers: Explain the most important principles how sleep and wakefulness are physiologically regulated and how sleep-wake regulation may differ between habitual short and long sleepers.}

\item \paragraph{Robotic tools have played a significant role in the investigation of human motor
learning.} \subparagraph{Describe the role of internal models in human motor control and how such
models are acquired} \subparagraph{Identify three unique features of robotic systems that make them valuable tools to investigate human motor learning.} \subparagraph{Discuss how these unique features could be applied to clinical assessment and therapy of sensorimotor impairments} answer here.

\item \paragraph{The term "frontotemporal dementias" subsumes a heterogeneous group of disorders:}
\subparagraph{Please describe the clinical presentations of patients with frontotemporal dementia (major clinical syndromes, and characteristic features).} \subparagraph{Which genes/gene loci have been associated with frontotemporal dementia?} \subparagraph{Please summarize which major molecular subgroups of frontotemporal dementias can be defined and briefly discuss current knowledge and/or hypotheses on underlying pathomechanisms in the two most common subgroups.} answer here.

\item \paragraph{The maintenance of central and peripheral tolerance is the reason that autoimmune diseases are relatively rare. Please answer the following questions:} \subparagraph{How does central T cell tolerance work (which organ performs T cell education, what is negative and positive selection)?} \subparagraph{What are the mechanisms of peripheral tolerance? Remember, we discussed four of them. Please shortly recapitulate.} \subparagraph{Why does the inflammation in an MS-lesion subside after a while? What mechanisms can dampen an ongoing immune response (if you do not know,
speculate!)?} answer here.

\end{enumerate}

\subsection{All Question - topics}

\subsubsection{Methods}
\begin{enumerate}
\item \paragraph{ fMRI is routinely used to study the neural processes underlying behavior. Please describe all the procedures necessary for conducting and correctly interpreting an fMRI study, covering the following areas:}
fMRI (functional Magnetic Resonance Imaging) is a non invasive technique that employs principles of magnetic resonance that are sensitive to blood oxygenation and are fast to acquire.

\subparagraph{Data acquisition (what signals are measured in fMRI, and how?)}
To acquire data it is necessary follow three steps: 
\begin{enumerate}
\item Place an object (brain) in a strong magnetic field: protons in the body have spins with a specific orientation and frequency. When the body is inside an MRI scanner, the protons align with the direction of the magnetic field.
\item Apply radio waves: radio frequences pulses with the appropriate frequence (Larmor frequency) change the orientation of the spins as the protons absorb the energy. When the pulse is turned off, the protons return to their original orientations (this process is called relaxation), and during relaxation the protons emit energy in the form of radio waves.
\item Measure radio waves emitted by the object (brain):  two measures can be acquired - T1 longitudinal and T2 transverse. T1 is how quicly the protons realign with the magnetic field and accurately distinguish different types of tissue, T2 is how quicly the protons emit energy when recovering equilibrium and relate changes in MR-signal to an experimental manipulation.
\end{enumerate}

One of the most commons signal used to relate changes in the MR-signal to the experimental manipulation is the BOLD (Blood Oxygenation Level Dependent) signal. This signal measures inhomogeneities in the magnetic field (T2) due to changes in the level of O\textsubscript{2} in the blood. This way, fMRI measures neural activity indirectly. The oxygenated blood is non magnetic while the deoxygenated blood is magnetic. When a specific region of the cortex increases its activity in response to a task, this region consumes the oxygen leading to an initial drop in oxygenated hemoglobine and an increase in local carbon dioxide and deoxygenated hemoglobine.

\subparagraph{Data analysis (which sequential routines are necessary to detect signal changes in fMRI images, using software packages such as SPM?)}
The data analysis using software such as SPM allows standardised detection of activity changes in each voxel, and for this three macro steps are necessary:
\begin{enumerate}
\item Preprocessing: in the preprocessing phase we can realign the images to fix small head movements, normalise the data to increase sensitivity with more subjects, to extrapolate findings to the whole population, to make the results comparable among different studies, also, we can smooth the image to increase signal to noise, impreve inter-subject average.
\item Model estimation: calculate parameter for instance, from GLM of voxel timeseries.
\item Contrasts and SPM: do the statistical inference.
\end{enumerate}

\subparagraph{Results interpretation (what inferences about neuronal activity can be drawn from fMRI results?}
The fMRI results (using BOLD signal) are not an absolut measure, the results can differ from session to session due to differences in the scanner, in the subject, etc. This way, BOLD signals need to be compared between different conditions within the same experiment to infer BOLD changes, for example, we acquire signal for the task P and also signals from a control withou the task P. The difference between the two acquisition will be the result of the task P. This is an approach that assumes a "pure insertion" theory, where cognitive (and neural) process can be added to others withou changing them and the change in behavior (and in brain activity) reflects only the added process. Also, the fMRI results are correlative results, i.e., they can show that signals from a brain region co-occur with a task of interest but cannot show that a region is necessary for that function.

\item \paragraph{What are the physiological correlations of fMRI signal? How does the fMRI signals correlate with neuronal activities?}

The physical correlation of fMRI is neural activity, resulting in an initial (about 0.5-2s) ‘undershoot’ of the proportion of oxygenated hemoglobin, due to the consumption of oxygen for the neural activity. This leads to a reduction of the BOLD (blood oxygen level dependant) signal. Neural activity seems to mediate vasodilation (maybe through to release of NO), leading to an increase of blood flow (after 2-10s), resulting in an increase of the BOLD signal due to the better blood supply. Studies comparing BOLD data with EEG data have shown that the BOLD signal rather reflects the information uptake and processing by neurons than their spiking output measured by EEG.

\end{enumerate}

\subsubsection{Memory}
\begin{enumerate}
\item \paragraph{Please list the different types of long-term memory you know of. Describe their properties in humans, group them according to involved brain structures and give examples of behavioral tests that allow to model these memory types in rodents}
\end{enumerate}

\subsubsection{Sleep}
\begin{enumerate}
\item \paragraph{Sleep regulation in physiological short and long sleepers: Explain the most important principles how sleep and wakefulness are physiologically regulated and how sleep-wake regulation may differ between habitual short and long sleepers.}

\item \paragraph{Characteristics of sleep in mammals: Do they apply to invertebrates?}
behavioural:
Sleeping site 
Quiescence
Body posture
Elevated arousal threshold
Rapid state reversibility
Physiological:
Altered EEG
Reduced muscle tone
Reduced heart rate
Reduced respiration
Reduced body temperature
Regulatory:
Compensatory response to sleep deficit or excess sleep

\item \paragraph{Non-REM-REM sleep}

REM: rapid eye movement. EEG low amplitude, mixed frequency (more similar to wake than to deep sleep EEG). Most prominent in the morning hours.
non-REM: is subdivided into four substages 1-4 in human, deep sleep consists of stages 3 and 4. In deep sleep, he EEG contains prominent slow waves (0.5-4.5 Hz, high amplitude).
cycles: REM sleep occurs every 90-100 mins during sleep (ultradian oscillator origins in the Pons). General term to describe cyclic alternation between REM and non-REM sleep. Healthy people usually start with stage 1, then 2, 3, 4, 2, REM, 2, 3, 4, REM etc.

\item \paragraph{Sleep homeostasis and marker of sleep homeostasis on the sleep EEG}

homeostasis has been defined as the coordinated physiological processes wich maintain most of the steady states in the organism; sleep homeostasis refers to the sleep need in dependance of the time spent awake. Sleep need rises exponentially during wake and declines exponentially during sleep. According to 2-process model of sleep regulation, sleep need is additionally dependant on circadian time.
	NREM-sleep is controlled thalamocortically.
        	Marker of sleep homeostasis: slow-wave activity (power of slow waves rises in recovery sleep after sleep deprivation according to the 2-process model)

\item \paragraph{Endogenous sleep-promoting components: comments}

SCN (superchiasmatic nucleus of hypothalamus)
	clock genes: transcriptional/translational process
	melatonin: built during sleep
	Thalamus: control of NREM-sleep
	Pons: Regulation of REM-sleep
	Potential homeostatic sleep-promoting agents (Experiment: if CSF from a sleep-deprived animal is transferred to a rested animal, the rested animal becomes tired $\rightarrow$ there must be an agent in the CSF that accumulates during wakefulness and makes tired): adenosine, Interleukin-1b, TNFa, GHRH, prostaglandin.

\item \paragraph{Role of thalamus-correlated rhythm in sleep: comments}

Thalamus controls the NREM sleep rhythmus $\rightarrow$ EEG activation / desactivation
\end{enumerate}

\subsubsection{Schizophrenia}
\begin{enumerate}
\item \paragraph{Please describe shortly a neurobiological model of schizophrenia}
\end{enumerate}

\subsubsection{Depression}
\begin{enumerate}
\item \paragraph{Please describe briefly:} \subparagraph{Some conceptional problems in studying depression from a neuroscientific point of view} 

Depression is a disorder of subjective feeling (translation from first person perspective to third person perspective: epistemic problem). It is very difficult to evaluate the disease scientifically (animal models that are really adequate to depression, they can’t talk to tell their feelings). The definition of psychiatry depends on social values and personal evaluation of suffering, but not depends on the organic disorder. No reliable bjective markers like genetic defects or metabolic disfunctions.

\subparagraph{Some changes of the neurobiology system in a depression state}
Change of HPA-axis (hypothalamus-pituitary-adrenal)
	The prominent mechanism by which the brain reacts to acute and chronic stress is the activation of HPA-axis $\rightarrow$ cortisol levels rise.
		Hypothalamus secretes CRH (corticotropin-releasing hormon) $\rightarrow$ pituitary (hypophysis) secrets adrenalcorticotropin (ACTH) $\rightarrow$ adrenal gland secrets cortisol

Growth hormone is reduced.
	Sleep disorders (EEG!), disturbances of appetite regulation.

different activation of brain areas (activation of medioorbital cortex and ventral anterior cingulate $\rightarrow$ limbic system activated)

\end{enumerate}

\subsubsection{Psychiatric disorders - general}
\begin{enumerate}
\item \paragraph{Animal models of behaviour allow us to investigate the symptoms of psychiatric disorders such as depression and schizophrenia. Discuss statements, giving examples of some specific model}

You always have to ensure that the animal model is valid. There are different aspects of validity that have to be guaranteed:
Face validity: quantifiable behavior and physiology in the animal model have to be similar to the symptoms in the investigated human illness.
Construct validity: the quantifiable behavior and physiology in the animal model must be a result of the same central state as in the human patient. Theoretical rationale.
Predictive validity: close correspondence between drug actions on behavior and physiology of the animal model and the human patient.
Inter-laboratory validity
Inter-species validity

 Schizophrenia:
Impairment of working memory leads to symptoms like halluzinations: lack of references against associative memories.
	Selective attention is impaired, leading to delusions (misinterpretations), confusion of external and internal stimuli and retreatment to safety (neg. symptoms). 
Specific test: Latent Inhibition (LI) test. LI paradigm: repeated non-reinforced pre-exposure to a stimulus retards subsequent conditioning to that stimulus. This reflects the ability of ‘learning not to attend’. In animals: rats reduce LI when given amphetamine $\rightarrow$ animal model for schizophrenia. Rats that get amphetamine, avoid the box where they got shocked previously in the CAR (conditioned avoidance response) and reduce licking water in the CER (conditioned emotional response) tests compared to control animals due to their impaired LI.


	Depression:
	Animal model of learned helplessness: animals are exposed to negative stimuli and don’t get the possibility to escape. This leads to the ‘learned helplessness’ symptom, especially, if the animals are very young, which means, they give up very quickly and are not able to escape unwanted situations. Learned helplessness can be measured by the escape behavior in a two-way avoidance test. In this test, animals are placed in a shuttle box and exposed to a foot shock. They are allowed to escape to the save compartment of the shuttle box. If they get conditioned for the shock with a tone, starting shortly before the shock, animals learn to escape already at the presentation of the tone. ‘Helpless’ animals are not good in escaping compared to controls.
	Chronic mild stress: Animals are chronically exposed to mild stress like food / water deprivation for some hours, not enough space, over night illumination etc. The loss of pleasure (anhedonia) is measured with the ICSS (intra-cranial self-stimulation), the PRS (progressive reward schedule) or the sucrose preference test.
	Early life stress: Pups are stressed by separating them from mother for several hours per day etc.$\rightarrow$ anhedonia

\item \paragraph{Describe methods for measuring motivation, attention and memory in rodents and/or primates. In which neuropsychiatric disease are these beavioural processes disrupted?}
\end{enumerate}

\subsubsection{Circadian Rhythm}
\begin{enumerate}
\item \paragraph{Circadian pacemakers, entrainment, Zeitgeber, phas-response curve}

Pacemaker: SCN (superchiasmatic nucleus of hypothalamus)
	entrainment means that the ‘inner clock’, located in the SCN, is flexible in the way that it can adapt the phase (example: time-zone flights) and the frequency (example: bunker experiments, where one ‘day’ lasts 25 hours) of the circadian clock.
	Entrainment: via light, signalling from the eye to the SCN (possible photoreceptor: Melanopsin?). In the SCN, per transcription is activated upon light signal.
Phase-response curve: depending on the circadian time, when a light pulse is presented, the phase of the circadian clock is shifted forward or backward. If the light pulse is presented shortly before the active period has started, then the phase is advanced and if the pulse is given shortly after the active period has ended, the phase is delayed (in humans). There is one time point during night when the phase shift swiches from delayed to advanced.

\item \paragraph{Which physiological and endocrine variables in human are frequently used a phase-marker of circadian rhythm}

endocrine: melatonin, adrenal gland (adrenalin, cortison); GHRH (Growth hormone releasing hormon)
	Physiological: body temperature; activity (via activity monitor); alpha-activity in the waking EEG

\item \paragraph{What is the evidence that SCN is a circadian pacemaker}

lesion method $\rightarrow$ arythmicity
         in vitro culture of a single SCN neuron    
         SCN transplant reserves the rhythm
         in vitro SCN

\item \paragraph{Which genes (gene?) are (is) involvedin generation of circadian rhythm}

in mammals: Bmal1 is rhythmically expressed by the SCN. Clock and bmal1(basic helix-loop-helix transcription factor family) build heterodimers. These heterodimers bind to E-boxes of enhancers of the per and cry gene. Per and Cry proteins dimerize outside the nucleus and are phosphorylated. The dimers re-enter the nucleus and downregulate transcription of  clock and bmal1 (negative feedback-loop). Situation is even more complex, also containing positive feedback-loops…)
        in fruit fly Cyc/Clk heterodimers activate per/tim gene transcription. Negative feedback-loop: Per/Tim heterodimers inhibit activation of their own genes via Cyc/Clk.
\end{enumerate}

\subsubsection{Models of Computation (?) $\rightarrow$ this topic was not covered in 2017}
\begin{enumerate}
\item \paragraph{Roles of automate as models for computation}

The computational process in neurons can be investigated in neuroinformatics via   automate models (as   compared to the structural process via neuroscience)

\item \paragraph{Automate suitable models for describing the operation of neurons and networks of neurons}

Models for automates are transferable to neurons / neuronal networks:

	Feedforward processor: input $\rightarrow$ blackbox $\rightarrow$ output (without memory)
	In neuron: input correlates to the dentritic input (sum of input signals x weights), output correlates to the axonal output (fire or not fire)

	Finite State Machine: input $\rightarrow$ black-box (which remembers the state in which it is, memory!) $\rightarrow$ output
	In neuron: neurons also can feed back information to build up memory (this model accounts only for short-time memory (seconds)…). Feedback occurs, when axonal output is networked to dendrites of the very same neuron.

	Turing Machine: input $\rightarrow$ black-box containing unbounded memory $\rightarrow$ output
	Church-Turing-Thesis: this machine is able to compute all possible computations. Philosophic question: is the brain’s memory unbounded???
	
	Universal Turing Machine: can simulate the computational process of any Turing Machine, when it knows the protocol of this machine, thus it can also simulate the computational process of a neuron… But the protocol is not known (e.g.: a bee can do computations leading to very various and complex behavior, there is no computational model that could do that with the limited recourses of a few thousand neurons that a bee needs to accomplish it)

	But: Neuronal networks have different weights for the 10 exp 14 axons/dentrite connections. This is not possible to be determined genetically (not enough resources), but dependant on the microenvironment of each neuron in the developmental process. Moreover, they can adapt to the environment by changing those weights or even establishing new connections between axons and dentrites.
	Synaptic release is additionally very versatile, can be modulated chemically, and be inhibitory or excitatory etc.

\item \paragraph{What is the impact of W.I. and what information do we get from it?}

\item \paragraph{ (a) Describe the properties of a finite state machine. (b) Is a single neuron kind of  a finite state machine? Explain your answer. (c) What kinds of model (artificial neurons) do you know about?}
\end{enumerate}




%%%%%%%%%%%%%%%%%%%%%%%%%%%%%%%%%%%%%%%%%%%%%%%%%%%%%%%%%%%%%%%%%%%%%%%%%%%%%%%%%%%%%%%%%%%%%%%%%%%
\newpage

\section{References}
The pictures used in this summary are from the class slide sets or internet, and belong to their respective owners. In the context of this summary they are used for educational purposes only.

\subsection{Cognitive Neuroscience}
\begin{itemize}
\item Christian C. Ruff and Scott A. Huettel, Chapter 6 - Experimental Methods in Cognitive Neuroscience, In Neuroeconomics (Second Edition), edited by Paul W. Glimcher and Ernst Fehr, Academic Press, San Diego, 2014, Pages 77-108, ISBN 9780124160088, \url{http://dx.doi.org/10.1016/B978-0-12-416008-8.00006-1}
\end{itemize}

\end{document}
