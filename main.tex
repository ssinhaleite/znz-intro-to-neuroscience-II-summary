\documentclass[12pt,article,oneside,a4paper]{memoir}

%% Packages
%% ========
\usepackage{graphicx}
\usepackage{titlesec}
\usepackage{wrapfig}

\setcounter{secnumdepth}{4}

\titleformat{\paragraph}
{\normalfont\normalsize\bfseries}{\theparagraph}{1em}{}
\titlespacing*{\paragraph}
{0pt}{3.25ex plus 1ex minus .2ex}{1.5ex plus .2ex}

%% many common packages
\input{commonpackages}

%% Some more packages that you may want to use.  Have a look at the
%% file, and consult the package docs for each.
\input{extrapackages}

%% Our layout configuration.
\input{layoutsetup}

%% Theorem environments.  You will have to adapt this for a German
%% thesis.
\input{theoremsetup}

%% Helpful macros.
\input{macrosetup}

%%page layout settings and listing templates etc.
\input{settings}

\title{\textbf{ZNZ Introduction to Neuroscience II} \\
       Spring 2017\\\normalsize version 1.0}

\author{
	Vanessa Leite
	\vspace{2em}
	\\Repository page: \url{https://github.com/ssinhaleite/znz-intro-to-neuroscience-II-summary}\\
	Contact \href{mailto:vrcleite@gmail.com}{vrcleite@gmail.com} if you have any questions.}
	\thesistype{The Summary of the lectures in 2017}
	\department{ZNZ - Institute of Neuroinformatics, ETH}
	\date{\today}

\begin{document}
\frontmatter


%% DO NOT CHANGE.
\begin{titlingpage}
  \calccentering{\unitlength}
  \begin{adjustwidth*}{\unitlength-24pt}{-\unitlength-24pt}
    \maketitle
  \end{adjustwidth*}
\end{titlingpage}

\mainmatter

%% This change is needed if the article option for the memoir document class
%% is used, in order to count sections (article) as if they were chapters (memoir)
\counterwithout{section}{chapter}

%% Our content

\newpage
\clearpage
\pagenumbering{roman}
\setcounter{tocdepth}{3}
\setcounter{secnumdepth}{2}
\tableofcontents

\clearpage
\pagenumbering{arabic}

\section{Cognitive Neuroscience}

%%---------------------------------------------------------------------------%%
\subsection{Methods in Cognitive Neuroscience - prof. Christian Ruff - 20.02.2017}

There are various types of methods to acquire cognitive information. Those
methods have different temporal resolution (can be acquired in milisseconds,
seconds, hours, day, etc), spatial resolution (can get information of the
brain, maps, columns, layers, cels, synapses, molecules) and invasiveness.

Techniques that record neuronal activity directly through electrophysiological
means tend to have very good temporal resolution and techniques that manipulate
brain function through drug effects or brain lesions tend to have the poorest
temporal resolution.

Techniques that position electrode sensors directly within the brain have the
highest spatial resolution and techniques that measure electrical signals that
spread diffusely tend to have the lowest spatial resolution.

Non-invasive techniques record endogenous brain signals using sensors outside
the body, thus they have almost no risk and can be conducted repeatedly in
human volunteer participants. In the other hand, invasive techniques introduce
a chemical or recording device into the body. Some invasive techniques can be
used in humam volunteers (with significant attention) but other can be used
only in human patients and/or non-human animals.

The main techniques that will be considered here are \textbf{correlative or
measurement} and \textbf{causal or manipulation}. Both types of techniques
provide distinct and complementary information about brain function.

Usually, the progress of cognitive neuroscience research happens quickly when
measurement (correlative) techniques establish links between brain structure
and cognitive function, and then manipulation (causal) techniques probe that
relationship.

\subsubsection{Correlative techniques}

Measurements techniques measure changes in brain function (information
transmited by neurons) while a research participant (human or animal) engages
in some cognitive activity. They are often described as being ``correlational''
because they can show that signals from a brain region co-occur with a function
of interest, but they cannot show that a region is necessary for that function.

\paragraph{(f)MRI} (functional) Magnetic Resonance Imaging

MRI is a non-invasive imagin technique that employs principles of magnetic
resonance to visualize different tissue types. Usually, one image is acquired.
It is slow (minutes) but accurate (sub-mm spatial resolution).

The fMRI employs special sequences that are i) sensitive to blood oxygenation
and ii) fast to acquire (whole brain in ~2-3 seconds). Numerous images are
recorded and represent timecourse of blood oxygenation durig experimental task.

How the magnetic resonance works:
\begin{itemize}
\item Place an object (brain) in a strong \textbf{magnetic} field
\subitem protons in the body have spins with a specific orientation and
frequency, when the body is inside an MRI scanner, the protons align with the
direction of the magnetic field.
\item Deliver energy in form of radio waves
\subitem pulses of radio frequence with the appropriate frequency (that depends
on the atomic nucleus being imaged, usually \textbf{Larmor frequency}) change
the orientation of the spins as the protons absorb the energy. The frequence
used is called \textbf{resonant frequency}. When the pulse is turned off, the
protons return to their original orientations, this process is called
``relaxation'', and during the (longitudinal $\rightarrow$ T1 and transverse
$\rightarrow$ T2) relaxation, the protons emit energy in the form of radio
waves (Figure~\ref{fig:orientation-relaxation}).
\item Measure radio waves emitted by the object
\subitem T1 (longitunial) is time constant of how quickly the protons realign
with the magnetic field, for instance, CSF has low signal (dark) and fat has
high signal (bright).
\subitem T2 (tranverse) is time constant of how quickly the protons emit
energy when recovering to equilibrium, for instance, fat has low signal (dark)
and CSF has high sinal (bright).
\end{itemize}

\begin{figure}
  \centering
  \includegraphics[width=0.3\textwidth]{imgs/mri-longitudinal-transverse.png}
  \caption{Orientation of the relaxation}
  \label{fig:orientation-relaxation}
\end{figure}

\textbf{The human scanners have a strong static magnetic field (around 1.5-7
Tesla). For reference, the earth's magnetic field is approximately 0.5 Gauss or
50-millionths of a Tesla.}

The \textbf{T1 fMRI} images are structural images with high spatial resolution
(less than 1 mm) and accurately distinguish different types of tissue. The
\textbf{T2 fMRI} images have lower spatial resolution (2-3 mm) and relate
changes in MR-signal  to an experimental manipulation.
Timeseries represents a large number of signals that are acquired in temporal
order at a specific rate.

Some terminology:
\begin{itemize}
\item subjects: the item that will be scanned
\item sessions: each time that the subject is inside of the scanner
\item runs: all the images generated in one section for the whole subject. One
complete scan of the subject is obtained in one single run.
\item volume: the 3d images generated from one single run
\item slices: each section of the volume is called slice.
\item voxel: each single unit information in a slice
\end{itemize}

\paragraph{the BOLD (Blood Oxygenation Level Dependent) contrast}
\label{bold-fmri}

BOLD measures inhomogeneities in the magnetic fiels (T2) due to changes in the
level of O\textsubscript{2} in the blood. This way, fMRI measures neural
activity indirectly via BOLD signal.

The oxygenated hemoglobine is diamagnetic (non magnetic) and produce no signal
loss, however, the deoxygenated hemoglobine is paramagnetic (magnetic) and then
produce a signal loss. When a specific region of the cortex increases its
activity in response to a task, the extraction of oxygen from the local
capillaries leads to an initial drop in oxygenated hemoglobine~(oxyHb) and an
increase in local carbon dioxide~(CO\textsubscript{2}) and deoxygenated
hemoglobine~(deoxyHb).

Some temporal properties of the BOLD signal:
\begin{itemize}
\item peaks 4-6 seconds after neural activity (delay)
\item back to baseline after approx. 30 secs
\item can vary in precise shape between regions and subjects
\item often shows undershoot and sometimes shows initial undershoot
\end{itemize}

Due to an over-compensatory increase of rCBF~(regional Cerebral Blood Flow),
increased neural activity can decreases the relative amount of deoxyHB.
This is called neurovascular coupling and it is an active area of research.

At present, the safest assumption is that BOLD relates to both spiking output
and excitatory postsynaptic activity in neurons. Inhibitory activity is not
assumed to lead to BOLD increases.

BOLD signal is not an absolute measure, but differs from session to session due
to differences in scanner sensitivity, subject, etc. This way, BOLD signal
needs to be compared between different conditions within the same experiment to
infer BOLD changes (increase or decrease) due to neural process of interest P.
In other words: \textbf{[Task with P] - [control task without P] = P}.
For this ``subtraction approach'', there are assumptions of ``pure insertion'':
i) cognitive (and neural) processes can be added to others without changing
them and ii) changed behavior (and brain activity) reflects only added process.

\paragraph{Design of fMRI experiments}
The advantages of fMRI are evident in its widespread acceptance among
researchers and its visibility among the general public. fMRI allows us to
\textbf{map complex cognitive functions in the brain of human volunteer
participants with a good combination of spatial and temporal resolution}.
However, fMRI has some disadvantages: it remains expensive, the scanner
typically costs \$500-\$1000 per hour; also, some participants will be excluded
based on issues related to safety (e.g., implanted devices) or comfort (e.g.,
claustrophobia). Moreover, even very small physiological variation (like head
movements of only a few milimeters, breathing, or heartbeats) introduces noise
into the BOLD signal.

The design of fMRI experiments can be block-designs or event-related designs.
In the \textbf{block-designs} we measure constant BOLD response to a
\textbf{series of stimuli}. In the \textbf{event-related designs} we measure
BOLD response to \textbf{each stimulus}.

Usually, fMRI experiments present an experimental stimuli displayed via
MR-compatible monitor, head mounted display, or projection system, and the
participant indicates his responses by moving a joystick or pressing a button.

\begin{itemize}
\item Block-designs:
\subitem higher statistical sensitivity for detecting effects.
\subitem some psychological process have to/may be better in blocks, for
instance, if there is a difficult to switch between states or to reduce surprise
effects.
\item Event-related designs:
\subitem randomised trial order
\subitem some events can not be blocked due to stimulus context.
\end{itemize}

In the fMRI designs, the predictions for BOLD signal can be categorical
(identify classes), parametric (stimuli rotating, expanding) or model-based
(check correlation between some model and BOLD signals). One can use a
factorial desing and combine different factors (categorical, parametric and
model-based) within one study, allowing study of context-dependent neural
responses (can show failures of pure insertion).

Sometimes, the resolution of the experiment is smaller than the MR image
resolution, for this, we can consider the MVPA~(multivariate activity pattern)
or repetition suppression instead of the univariate signal in each voxel. The
MVPA assumes that the signal in each voxel represents mixture of neuronal
populations specialised for different features. Note that the pattern of
increases and decreases may hence reliably differentiate different stimuli,
even if each voxel by itself does not. The repetition suppression design use
decreases in BOLD signal for 2nd vs 1st presentation of stimulus as index for
functional specialization of neurons within the voxel.

\paragraph{Analysis of fMRI experiments: SPM~(Statistical Parametric Mapping)}
SPM is a statistical approach instantiated in the most widely used software
package for fMRI anaysis, it is implemented in MATLAB and it is open source.
Allows standardised detection of \textbf{regional activity changes} in each
voxel, associated with task parameters.

\begin{itemize}
\item Preprocessing
\subitem Realignment (= registration): fix small head movements, assumes that
the shape of the brain does not change.
\subitem Spatial Normalisation: increase sensitivity with more subjects,
extrapolate findings to the whole population and make results from different
studies comparable (all in the same 'coordinate system')
\subitem Smoothing: increase signal to noise, improve inter-subject averaging.
In SPM, smoothing is a convolution with a Gaussian kernel. After smoothing,
each voxel becomes the result of applying a weighted region of interest.
\item Model estimation
\subitem parameters estimation from GLM of voxel timeseries
\item Contrasts and SPMs
\subitem statistical inference
\end{itemize}

The results from fMRI are presented usually in three diferent ways, as shown in
Figure~\ref{fig:fMRI-results}. More specifically:
\begin{figure}
  \centering
  \includegraphics[width=\textwidth]{imgs/fMRI-results.png}
  \caption{Common ways to present fMRI results. A) maps of activation, B) time
  courses of BOLD fMRI signal and C) parameter estimates of fMRI activation.}
  \label{fig:fMRI-results}
\end{figure}

\begin{itemize}
\item Maps of activation
\subitem the image is not a snapshot of the brain activity or a map of brain
function.
It simply indicates the results of a particular set of statistical tests, and
the threshold for significance is (usually) corrected by the number of tests
conducted.
\item Time course of activation
\subitem it shows the BOLD contrast MR signal changes over the duration of the
experiment. The pattern of changes in BOLD signal over time is called a
\textbf{hemodynamic response}.
\item Parameter estimates
\subitem This involves creating a hypothesized model for the changes in brain
activation that would be observed if there was an effect of the experimental
condition. Its main advantage is that it provides a hypothesis-based
statistical framework that can be adapted to any experimental design.
\end{itemize}

\paragraph{Advances in fMRI}
Contrary to popular conception, advances in MRI technology have not been
through stronger scanners, instead, the advances are changes in the hardware
and procedures for collecting fast and high-signal images. Rather than only
recording signals from a single sensor around the sample object, new
multi-channel scanners record MR signals from a large number of sensors at
different points in space.

\subsubsection{Causal Techniques}
In order to address the impact of neural processes on behavior, neuroscients
have developed several research techniques to experimentally manipulate neural
processing in specific brain areas. Causal or manipulation techniques examine
how perturbations of the brain's function change cognitive functions or
behavior. Pertubations on brain can be achieved either by transiently changing
neuronal firing rates or neurotransmitter levels (\textbf{brain stimulation
techniques}) or by permanently damaging tissue (\textbf{techniques that study
the consequences of brain lesions}). There is also a third technique:
\textbf{neurophamacological intervetion}.

\paragraph{Brain Stimulation Techniques}
Communication between connected neurons depends on the flow of electric
charges. Neurons maintain an electric potential of about -70mV and when this
potential rises above a fixed threshold voltage-gated ion channels open and
trigger action potentials. This variation on membrane potential is usually
caused by synaptic input from other neurons, but an external electrical current
can also affect membrane voltages and thus generate or inhibit action
potentials. Brain Stimulation Techniques produce electrical currents in the
brain in a controlled and locally specific fashion.

\paragraph{History of causal techniques: invasive stimulation}
Fritsch \& Hitzig in 1870 electrically stimulated an awake dog's brain via
inserted wires and caused involuntary movements. The experiments were done in
Frisch's home as the University would not allow the experiments. It was the
first study to show that externally supplied electricity triggers neural
function. Penfield and Rasmussen in 1950 attached electrodes to the cortical
surface of human patients who were about to undergo neurosurgery and applied
electrical current at various parts of the cerebral cortex. The behaviors and
sensations elicited by stimulation of each area were documented in one of the
first empirical maps of various motor, sensory and cognitive functions in the
human cortex \textbf{Systematic ``cartography'' of brain-behavior
(homunculus)}.

Nowadays, direct electrical stimulation of neurons via intracranial electrodes
remains a routine technique in animal research, but most neuroscients use
non-invasive brain stimulation techniques in human research as these techniques
do not require surgery and can thus be employed routinely in health
participants.

\paragraph{Causal Methods: Non-invasive stimulation}
Overcome need for invasive pre-surgical diagnosis, allow systematic testing of
excitability and integrity of motor tracts, modulate function of the cortex for
clinical purposes. Examples: TMS~(transcranial magnetic stimulation) and
tES~(transcranial electric stimulation).

\paragraph{TMS}

\paragraph{TMS: Biophysics}
From Faraday's Law: a time-varying magnetic field induces an electric field in
a conducting material. The induced electric field results in a measurable
voltage and current flow. For TMS, the conducting material is the brain and the
induced current activates neurons, as shown in Figure~\ref{fig:tms}.

\begin{figure}
  \centering
  \includegraphics[width=\textwidth]{imgs/tms.png}
  \caption{Basic principle of TMS}
  \label{fig:tms}
\end{figure}

Brain is not a homogenous conductor, but mixture of different materials (skull,
liquor, gray and white matter) that have different conductivities. So, how does
the electric field affect neurons?

\begin{itemize}
\item Activation of nerve fibre determined by the spatial derivative of the
field component parallel to the fibre (the activating function)
\item Nerve bends are low-threshold points and therefore easiest to stimulate;
the stronger the field, the stronger the stimulation.
\item Cortical neurons have numerous bends, terminals and branches; these will
all be affected most at the location where the induced fiels is maximal.
\item The likely stimulation point in the cortex for random orientation of
bends etc, is the field maximum.
\end{itemize}

TMS stimulates neurons by means of electromagnetic induction. A
looped copper coil is placed against the part of the scalp overlying the site
to be stimulated. By running a strong, rapidly changing electrical current
through the coil we generate a magnetic pulse perpendicular to the coil. This
pulse permeates the skull and brain tissue.

The rapid change of the magnetic pulse generates a complementary electric field
in any conductive material (in this case, the neural tissue). \textbf{In other
words, TMS uses a magnetic field, which can pass easily through the skull, to
generate an electrical field inside  the skull.}
The likelihood that an action potential will be generated at any location
depends on the orientation of these neurons with regard to the induced
electrial field, this means, \textbf{some locations in the cortex are easier to
stimulate than others using this technique}.

The two most common coil shapes are circular and figure-eight-shaped. Circular
coils generate powerful but more difuse fields, whereas figure-eight coils
result in more focal fields that produce the maximum current at the
intersection of the two windings.

\paragraph{TMS: Neurophysiology and types of stimulation protocols}
TMS pulses of hand representation in M1~(motor cortex area 1) cause measurable
twitches in hand muscles. Non-motor cortical areas require different behavioral
indices.

\textbf{Finding the right area} The first step of any TMS experiment involves
localizing the scalp area overlying the cortical area that is to be stimulated.
The experimenter needs to estimate where on the scalp the TMS coil needs to be
placed in order to induce currents in the target area. The stimulation area can
also be identified as the site at which TMS has maximal behavioral effects in a
separated task performed before the actual experiment begins.

\textbf{Finding the optimal TMS intensity} The optimal TMS intensity is usually
determined for each participant individually as a fixed percentage of the motor
threshold (MT), that is, the minimum intensity at which TMS applied over the
motor cortex elicits hand twiches.

\textbf{Influencing brain activity} There are at least two different ways to
influence brain activity.
\begin{itemize}
\item repeated TMS (rTMS) pulses can be applied \textit{online} during task
performance at a temportal frequency (5-20 Hz).
\subitem The rTMS pulses elicit unspecific neural activity in the targeted area
that disrupts cortical computations at that location.
\item rTMS can also be applied just prior to the task (\textit{offline}). The
offline TMS generates after-effects offering a window in which the normal
functional contribution of the stimulated area and possibly interconected areas
are markedly reduced.
\subitem rTMS can be applied for several minutes at low temporal
frequency~(1Hz) $\rightarrow$ can reduce Motor Evoked Potentials for roughly
the same duration as the length of rTMS application
\subitem rTMS can be applied for less than a minute in a \textit{theta burst
pattern}, tipically 3-5 pulses at 100 Hz repeated at 5Hz. Theta-burst TMS~(TBS)
mimicks the theta rhythm that is expressed during memory storage. Also, TBS has
been shown to lead to reductions (for continuous TBS) or enhancements (for
intermittent TBS) of Motor Evoked Potential size lasting more than 30 minutes.
\end{itemize}

\paragraph{From slides notes:} It is necessary to be sure that the brain area
is ``at rest'' during the stimulation (voluntary movements/contraction can
reverse/abolishes the effects.

\paragraph{Advantages and Limitations of TMS}
TMS allows non-invasive manipulation of neural processing with high spatial
resolution (about one centimeter) and exceptional temporal resolution
(milliseconds). However, nowadays it is only possible to target brain areas on
the cortical surface. Also, for offline studies, there is some uncertainty about
the precise duration of time window of TMS after-effects during which behavioral
tests can be conducted.

\paragraph{tES: Biophysics}
Two poles with electric potential difference (charge) connected through a
conductive medium. The connection leads to discharge by electric current:
negatively charged ions~(anions) flow to anode, and positively charged
ions~(cations) flow to cathode, as shown in Figure~\ref{fig:tes}.

\begin{figure}
  \centering
  \includegraphics[width=0.3\textwidth]{imgs/tes.png}
  \caption{Basic principle of tES}
  \label{fig:tes}
\end{figure}

tES envolves attaching two electrodes to the scalp and applying a constant
electric potential difference, thus running a weak but constant electrical
current between them. This affects the neurons along the path of the current,
slightly changing their membrane voltages and thus their spontaneous firing.
These effects are stronger directly beneath the electrodes, where the current
density is highest.

\textbf{About 50\% of the applied current reaches the cortex, the rest is 
shunted by the skull.}

\begin{table}[h]
  \begin{tabular}{ l |  p{6cm} |  p{6cm} }
    \hline
     & TMS & tES \\ \hline
    current & Induction of current by magnetic field & direct application of
    current \\ \hline
    area & relatively focal & not very focal \\ \hline
    induce & precisely timed burst of action potentials (+ physiological
    effects) & it does not induce time-locked neural activity but modulates
    natural activity \\ \hline
	threshold & suprathreshold stimulation & subthreshold stimulation \\ \hline
    effects & phenomenological & physiological \\
    \hline
  \end{tabular}
  \caption{Differences between TMS and tES}
\end{table}

\paragraph{History of causal techniques: lesion studies }
Lesion studies are used to investigate causal brain-behavior relations
(consequence of focal head wounds). They are measure in a hypothesis-guided
fashion, where cognitive and behavioral deficits of brain-lesioned patients
are studied.

Brain lesions in animals can be experimentally induced in the laboratory, which
enables scientists to test anatomically specific hypotheses about the relevance
of the brain areas for specific behaviors.

\paragraph{Lesion studies in humans}
The study of behavioral deficits in patients with brain damage (refered to as \textbf{neuropsychology}) originated in the neurological clinic.

One of the greatest challenges in neurological research is to determine the
exact scope and extent of the neural damage associated with the given
condition.

To test a hypothesis about the functional role of a given brain area using the
lesion approach, researchers first identify a group of patients with more or
less selective damage to the brain area. It is necessary, also, to identify a
suitable control group for behavioral comparison (the control participants need
to be closely matched to the patients with respect to behavior relevant factors
such as age, intelligence, socioeconomic status, cultural background, etc).

\paragraph{Advantages}
The brain-behavior relationship is truly causal. Behavior deficits due to brain
lesion can be very profound and can be evident to untrained observers.
Moreover, the behavioral deficits resulting from naturally occurring brain
damage can be very unexpected, leading to entirely new hypothesis. The
knowledge gained from lesion studies is always relevant for medical care as it
specifies behavioral deficits in patientes with specific types of brain damage,
which may help the diagnosis and treatment of these disorders.

\paragraph{Limitations}
Brain damage is often spatially diffuse, this can make it very difficult to
find patients with overlapping damage in the structures of interest. Often
litle is known about patients' behavior prior to the accident or illness. Brain
injuries and ilnesses and their treatment can have nonspecific sequelae that
may affect behavior, such as brain reorganization, medication effects, or an
altered life situation.

\paragraph{Lesion studies in animals}
In animals lesions are generated in clearly defined brain regions by various
means so that therapeutic measures and the time course of recovery can be
studied. A surgery is performed to produce a lesion at the designated site,
usually the damage is irreversible.

Lesion experiments in animals usually involve an experimental and a control
group of animals that undergo matched procedures to rule out any unspecific
effects of training, surgery, etc. The control group also undergoes surgery,
but the procedures do not involve harm to the brain. At the end of testing,
the extent of the lesion is documented by detailed \textit{post morten}
neuroanatomical and neurochemical examination of the brain tissue.

\paragraph{Advantages}
Full control over many variables that vary randomly in the context of
pathological brain lesion in humans. Animals can be randomly assigned to either
lesion or control group and can be perfectly matched in terms of experience,
life situation, etc.

\paragraph{Limitations}
Difficult to conduct: the training and keeping of experimental animals can be
very labor-intensive and costly. Surgery and behavioral testing require
considerable infrastructure. It is generally difficult to compare behavior
across species.

%%---------------------------------------------------------------------------%%
\newpage
\subsection{Perception and Attention - prof. D. Kiper - 27.02.2017}

\subsubsection{Perception}
Perception is \textbf{not a passive process}, sensation is the passive process.
Perception is \textbf{the process by which people select, organize, interpret
and respond to information from the world around them}. It is selection and
organization of environmental stimuli to provide meaningful experiences (we are
not passive analyzers). The particular perception of itself is called
proprioception. If you do not receive any stimulus (for instance, in a
depravation tank), your brain creates it.

\textbf{Perception is essentially the interface between the outer and inner
worlds.} The outer environment creates signals that can be sensed and the
perceiver receives these signals and converts them into psychologically
meaningful representations that define our inner experience of the world.

The perceptual process consists of six stages
(Figure~\ref{fig:perception-stages}):

\begin{figure}
  \centering
  \includegraphics[width=0.5\textwidth]{imgs/perception-stages.png}
  \caption{Stages of perception}
  \label{fig:perception-stages}
\end{figure}

\begin{itemize}
\item (1-2) People receive stimuli from the enviroment throught their senses.
\item (3) When the senses are activated, starts the perceptual selection. The
perceptual selection is a filter, that allow us to deal with the most important
matter. This is called \textbf{selective screening}: our system eliminates some
factors because they are not important for us to be aware of. The ``most
important'' is based on influencing factors that can be external or internal,
as listed in Table \ref{table:factors-perception}.
\item (4) When the most important stimuli is identified, starts the perceptual
organization process by which people group the stimuli in recognizable
patterns, listed in Table  \ref{table:perceptual-organization}.
\item (5-6) Then, we use the information received to interpret and respond to
the stimuli. Perception is noisy and makes mistakes, it is a very complex
system.
\end{itemize}

\begin{table}
  \begin{tabular}{ p{13cm} |  p{2cm} }
    \hline
    Internal & External \\ \hline
    personality -  strong factor & size \\ \hline
    learning and perceptual sets - expectation of particular interpretation
    based on past experiences with the same or similar objects & intensity
    \\ \hline
    motivation - the needs and desires at any particular time can influence
    perception (when you are hungry you can perceive a food as more delicious
    than when you are not hungry) & contrast \\ \hline
	 & motion \\ \hline
 	 & repetition \\ \hline
 	 & novelty \\ \hline
     & familiarity \\
    \hline
  \end{tabular}
  \caption{Externals and internal influencing perception}
  \label{table:factors-perception}
\end{table}

\begin{table}
  \begin{tabular}{ l |  p{10cm} }
    \hline
    perceptual grouping or continuity & lines are seen as following the
    smoothest path \\ \hline
    closure & tendency to complete an object and perceive it as a constant \\
    \hline
    color constancy & your brain starts to iluminate the enviroment making you
    think the color is the same in different enviroments \\
    \hline
  \end{tabular}
  \caption{Examples of perceptual organization}
  \label{table:perceptual-organization}
\end{table}

The most common types of perceptual errors are:
\begin{itemize}
\item accuracy in judgment (main types listed in
Table~\ref{table:accuracy-judgment}).
\item perceptual defence: the tendency for people to protect themselves against
ideas, objects or situations thar are threatening.
\item stereotyping: the belief that all members of a specific group share
similar traits and behaviors.
\item halo effect: tendency to color everthing we know about a person because
of one recognizable (un)favourable trait.
\item projection: tendenct to see one's trait in others.
\item role of culture: culture influence our perception in selecting
information and exhibiting a behavioral pattern in situations.
\end{itemize}

\begin{table}
  \begin{tabular}{ p{5cm} |  p{10cm} }
    \hline
    similarity error & assuming that people are similar to us and then, will
    behave like us \\ \hline
    contrast error & comparing people to others rather than to some absolut
    standard \\ \hline
    overweighting of negative information & tendency to overreact to something
    negative \\ \hline
    race, age, and gender bias & tendency to be more or less positive based on
    one's race, age or sex \\ \hline
	first impression error & forming first impressions that are resistant to
	change \\ 
    \hline
  \end{tabular}
  \caption{Examples of error in accuracy of judgment}
  \label{table:accuracy-judgment}
\end{table}

This simple perception-appraisal-response sequence is implemented innumerable
times in the course of any given day and, to a great extent, is completed
without effort.

\subsubsection{Attention}
Attention is taking possession of the mind, in clear and vivid form, of one out
of what seem several simultaneous possible objects or trains of thought. It is
the focalization, concentration of consciouness. It implies withdraw from some
things in order to deal effectively with others.

Attention can change rapidly, switching from one thing to another. It can be
steered by our intentions (``top-down''), as when we look for a particular face
in a crowd, or it can be steered by features of objects in the world
(``bottom-up''), as when our attention is grabbed by a police car's flashing
lights in our rearview mirror.

The bottom-up stimulus usually seems to require immediate behavioral responses
(e.g, a rock flying in your direction). It seems quite functional that the
attentional system is tuned to attend quickly to moving and looming stimuli as
those stimuli will commonly have more immediate survival implications than the
more stable ones.

Not all stimuli in the perceptual fiel receive equal attention; instead, some
stimuli are selected for relatively intense scrutinity, making them more likely
to reach the threshold of awareness. When an individual encounters stereotype/
inconsistent information, attention is drawn, and the perceptual details of an
event are encoded.

We can enhance perception, if we focus our attention on a location in the visual
field, this is called \textbf{spotlight of attention}. However, enhancing
perception in one part of the visual field takes place at the expense of other
areas.

\textbf{Shadowing task} is an experiment where two ``signs'' are presented to
the participant and one of the signs will ``hide'' the other.

There are three main types of attention: overt, covert and feature attention.
Each one of them are explained in Table \ref{table:types-attention}. The common
area for saccade eye movements (overt attention) and covert attention is the
frontal lobe; the area preferentially activated by covert attention is the
parieral lobe.

\begin{table}
  \begin{tabular}{ p{5cm} |  p{10cm} }
    \hline
    overt attention & selectively attending to an item or location over others
    by moving the eyes to point in that direction \\ \hline
    covert attention & related with spotlight of attention, you can't pay
    attention to many things, this way when you pay attention in one thing,
    your capacity to pay attention in others decay. \\ \hline
    feature attention & shadowing tasks, we can distinguish two mixed texts by
    focusing our attention on cues such as type style \\ 
    \hline
  \end{tabular}
  \caption{Types of attention}
  \label{table:types-attention}
\end{table}

There are two orienting mechanisms: \textbf{exogenous or reflexive} and
\textbf{endogenous or voluntary}. Table \ref{table:orienting-mechanisms} list
the main differences between them.

\begin{table}
  \begin{tabular}{ p{8cm} |  p{8cm} }
    \hline
    exogenous & endogenous \\ \hline
	\hline
    peripheral cues & central cues (e.g, arrows) \\ \hline
    fast (~ 100 ms) & slow (~ 300 ms) \\ \hline 
    occurs even with uninformative cues & occurs only with informative cues \\
    \hline
  \end{tabular}
  \caption{Two orienting mechanisms}
  \label{table:orienting-mechanisms}
\end{table}

\paragraph{Broadbent's filter theory - early selection}
Broadbent~(1958) argued that information from all of the stimuli presented at
any given time enters a sensory buffer. One of the inputs is then selected on
the basis of its physical characteristics for further processing by being
allowed to pass through a filter, all the others are lost. Broadbent assumed
that the filter rejects the shadowed or unattended message at an early stage of
processing. It takes time to shift attention.

\paragraph{Deutsch and Deutsch theory - late selection}
All information get through but attention filters only act after meaning is
analysed, for example, each ear receives a different message and it is asked to
the participant to say the message. The message can be recovered if it is the
only thing with meaning.

\paragraph{Treisman's theory - attenuation}
Physical characteristics are used to select one information for full processing
but other messages are given partial processing. 

Early or late experiments: Treisman and Geffen~(1967) tested whether the filter
was early or late in the processing stream. They had subjects shadowing a
message on one ear, and tap whenever they heard a certain word in either ear.
When the key word appeared in the attended ear, subjects tapped 87\% of the
time, and when the key word appeared in the unattended ear, subjects tapped 8\%
of the time. This indicates that early selection is occurring.

A lot of simple experiments can be made to evaluate attention, for instance,
pop out. It is easy to ``find'' elements based on one feature. Adding more
features (conjunction) will make it longer to find elements.
\textbf{Pop out experiments can prove sinestesia because one can use one
specific feature such as color, to find elements easily}. The more popout, more
late selection (selection after meaning). Popout can be trained!

\begin{figure}[h]
  \centering
  \includegraphics[width=0.5\textwidth]{imgs/early-late-selection.png}
  \caption{Different models of selection in attention}
  \label{fig:early-late-selection}
\end{figure}

\paragraph{Another definition of attention}
Attention is what enables us to process information about the world around us.
We can only be aware of things around us if we pay attention to them.
\textbf{Preattentive} processes help us to decide what to pay attention to and
what to filter out and ignore.

\begin{figure}[h]
  \centering
  \includegraphics[width=0.5\textwidth]{imgs/attention.png}
  \caption{A definition of attention by exclusion}
  \label{fig:attention}
\end{figure}

\paragraph{Effects in early visual cortex}
Desimone experiments. MTA~(medial temporal area) $\rightarrow$ prefered
direction of motion $\uparrow$. Attention is like a weight to fire particular
neurons.

\paragraph{Selective and divided attention}
Attention is studied by presenting participants with two or more stimuli at the
same time, this is called dual-task performance. In selective (focused)
attention tasks, people are instructed to respond to one input only. In divided
attention tasks, people are asked to process and respond to more than one input.

\paragraph{Gorila video:} innatentional blindness

%%---------------------------------------------------------------------------%%
\newpage
\subsection{Decision Making - 06.03.2017}
Two main activities for decision making: \textbf{valuation} and \textbf{action
selection}.
Valuation: what is the value of each action. Action selection: choose action
based on valuations.

What do we know about how the brain computes stimulus values at the time of
decision-making? Recent meta-analyses: Positive effects of stimulus values on
BOLD are higher than negative effects. Also, decision $>$ outcome.

What brain areas encode the stimulus value computation in simple choices between
primary rewards (e.g., foods)? Use of Becker-DeGroot-Marshack~(BDM) auctions to
measure goal values. Critical to have a real-time and incentive compatible
measure of goal values.

\paragraph{Experimental design}

\begin{figure}[h]
  \centering
  \includegraphics[width=0.7\textwidth]{imgs/experimental-design.png}
  \caption{Experimental design - free bid and forced bid trials}
  \label{fig:experimental-design}
\end{figure}

Subjects were instructed to eat at most 2h before the experiment to decrease
the attractiveness of the foods. During the informed consent period, subjects
agreed to eat whatever food item they would eventually have to eat as
determined by the auction mechanism described below.

Subjects were instructed that they would have to remain in the laboratory for
30 min at the conclusion of the experiment and that they will have to eat
whatever food was shown on a randomly selected trial unless they purchased the
right not to do so. In addition to a \$50/h participation fee, each subject
received three \$1 bills in ``spending money'' to place bids to purchase the
right to avoid eating disliked food items.
Whatever money they did not spend was theirs to keep. Subjects placed bids for
the right to avoid eating particular foods in 100 different bidding trials.
In each trial, they were allowed to bid \$0, \$1, \$2, or \$3. At the end of
the experiment, one of those trials was randomly selected, by drawing a ball
from an urn, and only the outcome of that trial was implemented.
As a result, subjects did not have to worry about spreading their \$3 budget
over the different items and they could treat each trial as if it were the only
decision that counted.

The rules of the auction were as follows. Let \textit{b} denote the bid made by
the subject for a particular item. After the bid is made, a random number
\textit{n} was drawn from a known distribution (in our case, \$0, \$1, \$2, and
\$3 were chosen with equal probability). If $b \geq n$, the subject did not
have to eat the item and paid a price equal to \textit{n}. In contrast, if $b <
n$, the subject had to eat the entire portion of the food shown but also did
not have to pay anything.

There is no incentive to bid less than the wilingness-to-pay~(WTP) since the
price paid is determined by the random number \textit{n}, and thus the bids do
not affect the price paid.
There is also no incentive to increase the bid above the WTP since this may
lead to a situation in which the subject avoids having to eat the item but ends
up paying a price larger than his WTP (e.g., consider the case of $WTP = \$1$,
$b = \$3$, and $n = \$2$). Since bidding the exact WTP is the best strategy,
the bids provide a good measure of the aversive goal value computed by the
brain for every subject and item at the time of decision making. Thus, we could
use the bids as a behavioral measure of aversive goal value to correlate with
the blood oxygen level-dependent (BOLD) signal on a trial-by-trial basis. As a
result, any area that encodes for goal values in free, but not in forced
trials, is a good candidate for the goal value computation.

\subparagraph{Free bid trials vs forced bid trials}
A key difference between the two types of trials is that, whereas subjects were
free to select the amount of their bid in the free trials, they were told how
much to bid in the forced trials. The forced bids were drawn uniformly and
independently from \$0, \$1, \$2, \$3 on each trial. As a result, subjects need
to make an aversive goal value computation in free trials to decide how much to
bid, but they do not necessarily need to do so in forced trials.

vmPFC and dlPFC encode for WTP in free trials, but not in forced trials.

\subsubsection{Valuation}
\paragraph{Appetitive and aversive value}
Which brain areas encode for aversive goal values?
Are these the same areas that encode for appetitive goal values?
Robustness to non-primary rewards and more complex decision problems: Is the
ventral medium pre frontal cortex~(vmPFC) a common area of stimulus valuation
across decision problems?

\paragraph{Multivoxel pattern analysis} PFC, LPFC, Parietal region

\paragraph{Deciding for self vs others}
vmPFC reflects the values you would assign when choosing for yourself.
vmPFC reflects the values you have learned, another area would assign the
values when choosing for others. The values computed are those that are
appropriate for the situation. dlPFC encoding reflects the values you assign
when choosing for others.

\paragraph{Model}
I'm not choosing for me but I'm still modeling. What I would choose for me?
There are some areas on the brain that are not core for the type of the test but
still may be part of the decision.

\subparagraph{How much people like chocolate after each square?}
The value decreases.

vmPFC and vStr are among the most consistently found regions in human
neuroimaging studies, however, value representations are not limited to these
regions.

\subsubsection{Action selection}
All costs are not created equal.

\paragraph{comparison:} costs x bemefits
\subparagraph{delay costs}
\subparagraph{effort costs}

Orbital Frontal Cortex~(OFC) lesions decrease willingness-to-wait. Anterior
Cingulate Cortex~(ACC) lesions decrease willingness to expend physical effort.

These experiments highlight both the power and complexity of using the lesion
method to examine brain function. Clear dissociation between ACC and OFC
lesions. Interactions between lesions and experience.

dmPFC/ACC, dlPFC, and IPS activity is consistent with regions implementing a
value comparison process. These regions are functionally connected with value
computation areas such as vmPFC and these regions modulate motor output in
favor of the selected action.

There is growing evidence that the brain makes goal-directed choices using a
process of valuation-action selection (Figure~\ref{fig:decision-making}).

\begin{figure}
  \centering
  \includegraphics[width=0.5\textwidth]{imgs/value-based-decision-making.png}
  \caption{Value based decision making process}
  \label{fig:decision-making}
\end{figure}

\subsubsection{Learning and perception in decision making}
We can think in many types of decisions: figuring out the number of an
approaching tram, or deciding what to eat in a new restaurant, or trading in
the stock market, or even playing poker.

\paragraph{Decision}
Make a decision is related with learning: depends of what you have learned.
The value of an decision (as in food) is not purely sensory or motor.

Reward system is divided in three parts:
\begin{itemize}
\item planning, judgment: prefrontal cortex and frontal cortex
\item reward: nucleus accumbens, medial forebrain bundle and ventral tegmental
area.
\item emotions, conditioned effects: amygdala
\end{itemize}

\paragraph{What is learned and how?}
What: Focus on value representations (aversive, appetitive)
How: Learning rules (rescorla-wagner, temporal difference learning rule)

\paragraph{Consequence of learning: long term memory}
Implicit memory as associative learning: classical and operant conditioning
with emotional responses in amygdala.
Classical conditioning: Before conditioning the dog only
salivates~(unconditioned response) in the presence of the food~(unconditioned stimulus).
The same way, with the presence of a sound~(neutral stimulus), there is no
salivation~(no conditioned response). During conditioning, the sound is
presented togheter with the food, which makes the dog salivates. After
conditioning, only the presence of the sound~(conditioned stimulus) is enough
to produce salivation~(conditioned response).

\paragraph{Memory functions of the temporal lobe}
Patient H.M.: ablation of temporal lobe, particularly hippocampus caused strong
anterograde amnesia, but intact working memory and procedural memory.

\textbf{Conditioning is impaired in amygdala but not hippocampus. The inverse
holds for declarative knowledge acquisition.}

\paragraph{Fear impairement (patient SM)}
Lesion in amygdala generates inability to recognize fearful faces but intact
recognition of different emotions. Also, impaired attention to eye region and
an inability to experience fear (except if it is elicited ``internally''
through CO\textsubscript{2}.

The amygdala is involved in value learning (particularly fear conditioning).

\paragraph{Learning mechanisms}
Hume: temporal contiguity, spatial contiguity and similarity (shape, etc).
Rescorla: contigency.
Kamin: prediction error (blocking paradigm).

\subparagraph{Learning by prediction errors}
In the learning phase, it is learned that with the sound, we will receive one
euro. In a next phase, if when we heard the sound, we gain 10 cents, the
prediction error is negative, if we gain one euron, the predicion error is
zero, and if we gain more than one euro, the prediction error is positive.
Prediction error response is sensitive to predicted reward magnitude.

This way, $prediction error = outcome - prediction$, and $new prediction = 
previous prediction + prediction error$.

\textbf{Learning is proportional to prediction errors}.
$\Delta V = \alpha \beta (\lambda - \Sigma V)$
Where, 
\begin{itemize}
\item $\Delta V$ is the change in associative strength of conditioned stimulus
(learning)
\item $\alpha$, $\beta$ learning rate
\item $\lambda$ is the amount of processing of unpredicted unconditioned
stimulus 
\item $\Sigma V$ is the summed current associative strength of conditioned
stimuli presented (amount learned)
\end{itemize}

\subparagraph{Temporal difference learning}
It is the difference in expected and actual outcome: prediction error signal in
relation to time $t, t+1...$.

Predicted value: $V(t) = \Sigma w_{i} x_{i}(t)$, where $w_{i}$ is the weight
and $x_{i}(t)$ is the stimulus representation (presence of CS).
Prediction error: $\delta(t) = (r(t) + \gamma V(t+1)) - V(t)$, where $r(t)$ is
the reward at time t and $\gamma$ is the discount factor.
$\Delta w_{i} = \alpha \Sigma x_{i}(t) \delta(t)$ it is the correspondence to
recorla-wagner rule $\Delta V = \alpha \beta (\lambda - \Sigma V)$.

\paragraph{Dopaminergic neurons} 
Dopaminergic neurons appear to code a (signed) reward prediction error. Action
potentials fires when unpredicted reward occurs; fewer action potentials than
``normal'' when no reward occurs. Increasing reward probability leads to
increasing CS response and decreasing reward response.

Reward prediction carries temporal information: delayed and early reward
activates dopaminergic neurons thus, prediction error response is computed
moment by moment.

L-Dopa enhances prediction error responses in ventral striatum, haloperidol
reduces them, dopamine release depends on prediction.

Activity of dopaminergic neurons, dopamine concentrations in target structures
and BOLD-activity in target structures such as the striatum process prediction
errors as described by temporal difference learning rules.

\paragraph{Altruistic learning}
Nicer people process value for others more than the value for themselves. The
spatial gradient (on ventral-dorsal regions) of altruistic value representation
is more pronounced for selfish than for altruistic participants.

\paragraph{selfish = other = dorsal region}
ventral PFC = decision for myself\\
dorsal PFC = decision for others
\\
\\
\\
Reinforcement learning is driven by prediction errors, as captured by formal
learning models.

\subsubsection{Representation}
What do we mean with representation?
What is represented? Sensory/percetual features, blood sugar, objective choice
option paramteters, pre-decision.
What if subjective processes are represented or impact representations in
absence of choice?

\paragraph{Representation $\rightarrow$ move from objective to subjective}
\paragraph{dopamine neurons may represent values in a objective way}
\subparagraph{the same value has lower reward value if the subject needs to
wait more for it}

Representations occur at various processing stages. The orbitofrontal cortex,
the hypothalamus and other regions (e.g. the insula) represent internal states.
Single neurons represent magnitude and probability of rewards and punishments.
Representations can be influenced by learning.

%%---------------------------------------------------------------------------%%
\newpage
\subsection{Emotion - Prof. Dominik Bach - 13.03.2017}

\paragraph{What is an emotion?}
Emotions are the subjective feelings that are associated with physiological
states. Emotions are short lasting and mood are long lasting. Some of the most
devastating psychiatric problems involve emotional (affective) disorders.

\subparagraph{What is a feeling?}
qualia (non communicative) + verbal expression.

Emotion as everyday feeling, different theories: neuroscience, comparative and
psychological. From neuroscience, neurons control behaviors; from comparative, 
emotion is shared between humans and animals.

\paragraph{Physiological changes associated with emotion}
All emotions are expressed through both visceral motor changes and stereotyped
somatic motor responses. Examples of changes in the activity of the visceral
motor are increase or decrease in heart rate, cutaneous blood flow (blushing or
turning pale), sweating, and gastrointestinal motility.
Activation of the visceral motor system, particularly the sympathetic division,
was long considered an all-or-nothing process. More recent studies have shown
that the responses of the autonomic nervous system are actually quite specific,
with different patterns of activation characterizing different situation and
emotional states. For example, if subjects are given muscle-by-muscle
instructions that result in facial expressions recognizable as an emotion
without being told which emotion they are simulating, each pattern of facial
muscle activity is accompanied by specific and reproducible differences in
visceral motor activity (as measured by indices such as heart rate, skin
conductance, and skin temperature). 

In 1928, Phillip Bard, removed both cerebral hemispheres (including the cortex,
underlying white matter and basal ganglia) in a series of cats.
When the anesthesia had worn off, the animals behaved as if they were enraged.
This behavior was called \textbf{sham rage}, because it had no obvious target.
Bard suggested that whereas the subjective experience of emotion might depend
on an intact cerebral cortex, the expression of coordinated emotional behaviors
does not necessarily entail cortical process. Conclusion: basic circuits for
organized behaviors accompanied by emotion are in the diencephalon and the
brainstem structures connected to it.
In short, the somatic and visceral activities associated with unified emotional
behavior are mediated by the activity of both the somatic and visceral motor
neurons, which integrate parallel, descending inputs from a constellation of 
forebrain sources. 

The left hemisphere is more importantly involved with what can be thought of as
positive emotions, whereas the right hemisphere is more involved with negative
ones. Also, the right hemisphere is more intimately concerned with both the 
perception and expression of emotions than is the left hemisphere.

\paragraph{Papez circuit}
It is a circuit suggested by Papez. He believed that these pathways provided
the conenctions necessary for cortical control of emotional expressions.
Figure~\ref{fig:papez-circuit} shows the Papez circuit: the cingulate cortex
and hypothalamus are interconnected via projections from the mammillary bodies
(part of the posterior hypothalamus) to the anterior nucleus of the dorsal
thalamus, which projects in turn to the cingulate gyrus.
The cingulate gyrus (and many other cortical regions as well) projects to the
hippocampus. Finally, the hippocampus projects via the fornix (a large fiber
bundle) back to the hypothalamus. Over time, some of the structures that Papez
originally described (for instance, hippocampus) now appear have little to do 
with emotional behavior, whereas the amygdala (that he hardly mentioned)
clearly plays a major role in the experience and expression of emotion.

\begin{figure}
  \centering
  \includegraphics[width=0.7\textwidth]{imgs/papez-circuit.png}
  \caption{Papez circuit}
  \label{fig:papez-circuit}
\end{figure}

\paragraph{fear conditionining x fear expression x fear prosody}
\textbf{Fear conditioning}: conditioning stimuli and unconditioning stimuli
inputs converge in amygdala. Amygdala neurons show long term potential (LTP),
amygdala LTP facilitates CS conditioning. Amygdala LTP blocking blocks
conditioning. Amygdala neuron firing relates to freezing behaviour. Amygdala
lesions disrupt fear acquisition and lesions post training erase fear memory.

\textbf{Fear expression}: detection of fear expression is impaired in bilateral
amygdala lesions.

\textbf{Fear prosody}: is unimpaired in bilateral amygdala lesions.

\paragraph{Find methods to measure emotions}
What criteria should we use to decide if a particular behavior is emotional or
not?

\subsubsection{Comparative theories}
Basic emotion theory: facial expression of basic emotions, and their
recognition, are organised in categories of basic emotions and they are mostly
culture-unspecific.

\paragraph{Common sense}
We do things (behavior) because we feel.
(sensation $\rightarrow$ subjective feeling $\rightarrow$ behavior)
\paragraph{Physiological theory}
we feel because we sense things. Attribution: think about the senses.
(sensation $\rightarrow$ body response $\rightarrow$ subjective feeling)
\subparagraph{James-Lange} event $\rightarrow$ body reaction $\rightarrow$
feeling
\subparagraph{Schachter-Singer} event $\rightarrow$ body reaction $\rightarrow$
attribution $\rightarrow$ feeling
\paragraph{Facial feedback} active some muscles leads to feeling. sensation
$\rightarrow$ expression (face, posture, prosody) $\rightarrow$ subjective
feeling
\paragraph{Appraisal theory:} automated no-consciouness that leads sensation to
feeling: ``I felt fear because I evaluated a situation as dangerous''.
(sensation $\rightarrow$ automated appraisal $\rightarrow$ subjective feeling
/body response)
\subparagraph{Richard Lazarus} 1. Is the event relevant for the agent? 2.
Can the agent cope with the event?
\subparagraph{Klaus Scherer} Sequential Stimulus Evaluation Checks (SECs):
1. Relevance (novelty, (un)pleasantness); 2. Consequence (causes of the event,
probability of consequences, urgency); 3. Coping potential (control, power);
4. Normative significance.
\paragraph{Affect theory:} meta-cognitive representation of inner states,
including emotions, framework to talk about subject feelings (inner states,
including emotions $\rightarrow$ subjective feeling)

\subsubsection{Decision Theoretical View}
What we can observe in emotions are internal and external actions. How are they
decided upon and coordinated?

\paragraph{Emotions are actions}
As actions, emotions are adaptive, and can be optimised to achieve goals.
\textbf{Attention on two things:} goals and controls algorithms (that decide
the actions).

For non-discrete problems, computations are often analytically intractable. The
(numerical) computations are often extremely time-consuming and many variables
(including possible actions) are unknown. Use of approximating algorithms,
optimal is too much complicated; why more time to go when is more difficult?
The approximating algorithms are listed in
Figure~\ref{fig:approximating-algorithms}.

\begin{figure}
  \centering
  \includegraphics[width=0.5\textwidth]{imgs/approximating-algorithms.png}
  \caption{Types of approximating algorithms}
  \label{fig:approximating-algorithms}
\end{figure}

\paragraph{Approximating algorithm:} 
Requirements: 
\begin{itemize}
\item Need for speed: high-dimensional distributions problematic
\item Learning from errors often impossible: reinforcement learning problematic
\item Lots of information missing: need to integrate over many variables.
\end{itemize}

Some possible solutions for this approximating algorithms are pre-programming
algorithms to 1. operate on selected sensory input; 2. consider only a selected 
action; 3. remove/reduce plasticity. That means: we need multiple algorithms.

\begin{itemize}
\item Preprogramming sensory input: rats cannot learn to freeze from taste
(but they learn to freeze from shock). They cannot be conditioned to avoid
audiovisual CS, but they can learn from be sick.
\item Preprogramming action menu: nose-poking cannot be conditioned to avoid
threat (shock), but nose-poking can be conditioned to avoid starvation.
\item Limited plasticity: startle eye blink cannot be conditioned to CS, but
airpuff eye blink can be conditioned to CS.
\end{itemize}

\paragraph{Discrete algorithms}
Which sensory input? Which actions? Action-outcome relation fixed
(pavlovian vs. instrumental)? What kind of algorithm (model-free vs
model-based)?

Should we restric ``emotions'' to humans?
Rodent anxiety models: elevated plus maze, open field test, novelty, suppressed
feeding test, successive alleys test, etc.

\paragraph{many ``algorithms'' for many behaviours}
\paragraph{game:} go pick a ``piece'' fast enough to not be hitted by ``rock''
\paragraph{hippocampus lesion:} difficult approach
\paragraph{amygdala lesion:} return approach

\paragraph{meta control:} how we decide what we decide?

\paragraph{Why study humans?}
Communication of emotions (facial expression, prosody, body posture),
attentional capture, induction of feelings, influence of 'emotional stimuli' on
deliberate decision-making, and clinical application (fear conditioning,
anxiety-like behavior).

\subsubsection{Clinical application}
\paragraph{Fear extinction}
You can train the extinction of fear. However, the complete memory is not
``removed''. Extinction builds a competing memory.
\paragraph{Memory reconsolidation}
Memory consolidation needs proteins, with this we can ``unlearn''responses to
CS.

\paragraph{Fear erasure}
Memory erasure goal: specific memory re-activation in psychotherapy context;
drug intervention to prevent reconsolidation.
Where in the brain the fear seats? amygdala?

\paragraph{Multivariate fMRI analysis}
\subparagraph{benzodiapezines:} ansiolitics
\subparagraph{lesion on hippocampus make rats stay longer on open arm}
\paragraph{slide: benzo:} red line; placebo: black line (human analogon)

\subsubsection{Anxiety}
Subjective feeling with physiological responses.

%%---------------------------------------------------------------------------%%
\newpage
\subsection{Memory - Prof. Katharina Henke - 20.03.2017}
One of the most intriguing of the brain's complex functions is the ability to
store information provided by experience and to retrieve much of it at will.

Learning is the name given to the process by which new information is acquired
by the nervous system and is observable through changes in behavior.
Memory refers to the encoding, storage, and retrieval of learned information. 

\subsubsection{Methods and Memory}
To identify areas on the brain that are essential for a task is necessary to
analyze people with brain damage (in a specific area).

\paragraph{fMRI}
Good spatial resolution througout the brain. Measures blood oxigenation
(see BOLD in \ref{bold-fmri}).
The aims of the fMRI study are:
\begin{itemize}
\item to find the structures that support learning/tasks activities/retrieval
activity (detect a task-relevant network of brain structures and functional
specialization of a brain region)
\item to discover silent process (animals and humans)
\item to read the unconscionous mind using fMRI (detect mental process that do
not reflect in behavior)
\item to identify people with real memory problem (it can detect simulation in
patients)
\end{itemize}

\paragraph{PET - Positronen Emission Tomography}

Used on study of learning and memory, measures metabolites and neurotransmitter
concentrations (most common: glucose uptake).
\subparagraph{example of patient - 23 years old with psycological trauma}
No brain damage (structural damage) observed, glucose PET indicates if area in
the brain is working or not - functional amnesia.

\paragraph{Electro encephalography - EEG}

EEG has good temporal resolution, and is used on study of learning and memory
consolidation during sleep, measures excitatory and inhibitory postsynaptic
potentials in neocortex.

\subsubsection{Multistore model}
Three temporal classes of memory are generally accepted.
Sensory memory (up to 1s) is the routine ability to hold ongoing experiences
in mind for fractions of a second. The Short Term Memory~(STM, sometimes
working memory) is the ability to hold information in mind for seconds to
minutes (15-20s to 1 min) once the present moment has passed. The Long Term
Memory~(LTM) entails the retention of information in a more permanent form of
storage for days, weeks or even a lifetime (minutes and longer).

The capacity of working memory very much depends on what the information in
question means to the individual and how readily it can be associated with
information that has already been stored.

\textbf{Forgetting} Were it not for forgetting, our brains would be impossibly
burdened with the welter of useless information that is briefly encoded in our
immediate memory ``buffer''. In fact, the human brain is very good at
forgetting.
Although forgetting is a normal and apparently essential mental process, it can
also be pathological, a condition called \textbf{amnesia}.

\textbf{Amnesia} is defined as the inability to learn new information
(anterograde amnesia) or to retrieve information that has already been acquired
(retrograde amnesia).

\paragraph{Hippocampus}
It is important to establish new declarative memories.
Spatial learning and memory is related with hippocampus. Rats with lesion in
the hippocampus cannot learn to find the platform in a pool. However, it is not
used to recall nondeclarative memory (patient HM case).

episodic memory $\rightarrow$ memory for personal
episodies - autobiographical, what, where, when.
Hippocampus is very vulnerable, i.e., there are lots of diseases by damage on
hippocampus. The larger hippocampal resections, the worse the resulting
learning impairments.

\paragraph{ovulation increases hyppocampus activity as in menstruation (?)}

\paragraph{Long Term Memory}
The LTM is divide in two types: declarative (or explicit) and nondeclarative
(or implicit). The complete division of memory (based on counsciouness) is
shown in Figure~\ref{fig:old-memory-model}.

\begin{figure}[h]
  \centering
  \includegraphics[width=0.5\textwidth]{imgs/old-memory-model.png}
  \caption{Memory model based on counsciouness}
  \label{fig:old-memory-model}
\end{figure}

\subparagraph{Declarative (explicit) memory}
Declarative memory is the storage (and retrieval) of material that is available
to consciousness and can be expressed by language (hence, ``declarative'').

It is divided in two: semantic and episodic memory.

The \textbf{semantic memory} is independent of hippocampus and throught
repetition is possible to learning new things using the neocortex and not
passing throught the hippocampus. The semantic memory holds facts, general
knowledge, no context and refers to the present.
Brain structures: \textbf{lateral temporal lobe} and \textbf{parahippocampal
gyrus}.

The \textbf{episodic memory} holds happenings, temporal and spatial
context, personal information and mental time travel into the past.
Brain structure: hippocampus.

\subparagraph{Nondeclarative (implicit) memory - unconsciouness}
Nondeclarative memory (sometimes referred to as procedural memory) is not
available to consciousness. Such memories involve skills and associations 
that are, by and large, acquired and retrieved at an unconscious level.
Is divided in three memories: procedural, priming and classical conditioning.

\begin{itemize}
\item Procedural: striatum, motor and premotor cortex and cerebellum.
\subitem HM patient still able to ride a bike.
\item Priming: neocortex
\subitem tendency to process some perception in the same way. When you see a
complex picture for the first time you need a longer time to identify things
then in a second/third time.
\item classical conditioning: amygdala and cerebellum
\end{itemize}

\paragraph{Role of hippocampus in episodic memory} 

Hippocampus is speciallized in associate memory. This association can be
spatial, semantic, temporal or sensory.
\subparagraph{anterior part:} semantic and temporal association
\subparagraph{posterior part:} sensory and spatial
Hippocampus is necessary for uncounscious relational encoding and retrieval.
It is higly active during deep sleep. Specific patterns of activity observed in
the hippocampus while rats were awake were observed to “replay” during both REM
sleep and slow wave sleep.

Knowing the precise function of a brain structure allows for a precise
diagnosis of brain function.

\subsubsection{New memory model - Katharina (2010)}
This model divides memory based on processing modes, not consciousness.
Three characteristics: number of learning trials, associations and flexibility.
Figure~\ref{fig:new-memory-model} shows the new memory division.
\begin{itemize}
\item Encoding (number of learning trials): rapid (1 trial) or slow (many
trials)
\item Association \& flexibility: flexible associations, rigid associations, or
no associations (unique item)
\end{itemize}

\begin{figure}[h]
  \centering
  \includegraphics[width=0.5\textwidth]{imgs/new-memory-model.png}
  \caption{Memory model based on processing modes}
  \label{fig:new-memory-model}
\end{figure}

%%---------------------------------------------------------------------------%%
\newpage
\subsection{Body Perception - 27.03.2017}
\subsubsection{Phantomology}
Phantomology (is the name of the second chapter of Summa Technolofiae - 1964)
and defines it as the science of the virtual reality of the body.
The science of the body in the brain, from out-of-limb to out-of-body experiences. 
Phantom  limb  sensation refers to the persistent experience of the postural and
motor aspects of a limb after its physical loss.

Nearly all patients have an illusion that the missing limb is still present,
following the amputation.
Phantom sensations are not limited to amputed limbs. Many types of phantom
sensations are reported, such as:
\begin{itemize}
\item phantom body parts: the feeling of the body part after the removal of the
body part (not necessarily a limb).
\item hemiphantom: the feeling that half of the body is an entity that has its
own life.
\item phantom double: the feeling of have a second body that imitates the
original one.
\item out of body experiences: the feeling of be disconnected with its own body.
\end{itemize}

The phantom limb is studied since 1510 (Paré) but the term ``phantom limb''
is from 1829 (Mitchell).

\textbf{Sensation is in the brain, not in the limb - Descartes}

Even adult brains are capable to reorganize itself. In one experiment with adult
monkeys, the cortical plasticity was identified: the monkeys lose a finger and
the neighbors fingers took the ``empty'' area of the finger on the brain.
In the maladaptive model, the pain is the price for the plasticity, the loss
of input generates a ``invasion'' that causes pain.
In the maintained representation model, the pain generates an increased input 
and then it is maintained the neural representation, in Nature 2013, an article
was published saying that phantom pain is associated with preserved structure
and function in the former area (fMRI activation on phantom hand movement).

It is not very clear what is guided by phantom limb and why some people feel 
phantom limb pain. It is the peripheral nervous system or the central nervous
system that origin the phantom limb sensation?

\paragraph{Phantom pain}
Phantoms might simply be a curiosity, or a provocative clue about higherorder 
somatic sensory processing, were it not for the fact that a substantial number 
of amputees also develop phantom pain. Phantom pain is, in fact, one of the 
more common causes of chronic pain syndromes and is extraordinarily difficult
to treat. Because of the widespread nature of central pain processing, ablation
of the spinothalamic tract, portions of the thalamus, or even primary sensory
cortex does not generally relieve the discomfort felt by these patients.

\paragraph{Phantomology}
There are two types of phantom body parts: after amputation and in congenital limb
deficiency.
Phantom limb can be feel even in congenital limb experiencies.
Three main explanations: i) projection of enhanced motility of finger rudiments;
ii) based on contralateral representation of intact limb; iii) based on innate
schema for hand-mouth coordination. 
One patient has been feeling, as long as she can remember, a complete body.
And she gets aware of hands in reflex movements and uses them to gesture.
Also, she reports preference in posture (for instance in folding her arms) and
says the inverted posture feels ``awckward''. Natural phantom limbs obey the
``lawfulness of limbs'', but amputees can learn to ``execute'' impossible limb
movements.

\subparagraph{fMRI of phantom hand movements}
Task: rhythmic opening/closing of the right phantom hand. Showed bilateral
activation in premotor and parietal cortices, but no M1 and S1 activations.
\textbf{Phantom moving is like imagine moving a limb - bilateral activation on
brain.}
 
Scientific fact: does not mean remembering is not important, also does not mean
``body schema'' is ``inate''.

One neglected phenomenal aspect of phantom limbs is the obstacle shunning: put
a physical object at the same place as the phantom body part. Approximately 50\%
believes the phantom limb is still there even when a pile of books is placed at
the same location of the limb, the others 50\% perceive that the phantom limb is
in his/her head. Experiment: ask for a person with phantom limb sensation to
point out the location of the phantom body part, and then, add a physical object
to the same place. The phantom body part vanishes but the phantom pain remains.
\textbf{Even if the phantom limb sensations disappears the phantom limb pain
can still holds} In some cases, the confrotation of the phantom limb with solid
matter makes the person change the postural body as if the phantom limb changed
its place. Also, there are cases where the kind of matter that is taking the
place of the phantom limb matters. For instance, in one experiment with a man
who lose the anterior part of his right foot, if he put his foot really close
to a wall or other objects, the phantom sensation was there (but with the toes
in different position), however, when the man put his foot behind his left foot,
the sensation was different (as if the toes retracted into the foot).

To move the feet from an obstacle, if we do it in a fast way (brief interstimulus
intervals) the foot appears to pass through the object. However, if we do it with
longer interstimulus intervals, the foot appears to go around the object.
Amputees with shunning show the long apparent motion trajectory with shorter
interstimulus intervals. Shunning is associated with worse prothesis tolerance.

The phantom limb studies is relevant in research because the individual
differences in visual-somesthetic interactions, but also has a clinical
relevance considering the adjustment to a protesis.

\paragraph{Phantom limb without amputation}
Also called spinal phantom limb. The main difference to amputation phantom is
the visual observation of dysfunctional but still present limbs. 

\paragraph{Supernumary}
It is a condition where the affected individual believes and receives sensory
information from limbs of the body that do not actually exist, and never have
existed. One of the patients have the sense of arms protruding from chest.
Every time the patient tried to ``slip into the phantom limb'', she reports
that the phantoms are evade laterally.

You do not need to physically lose a limb to experience phantom limb. Without
the visual system we can simulate the feeling of phantom limb.
Experiment 1: Pinocchio-illusion - with closed eyes, participant touch his/her
nose, biceps is stimulated. Participants perceive arm or/and nose in different
location.
Experiment 2: Rubber hand - Place the rubber hand on a table in front of you
and conceal your real hand behind a cardboard. A second person will, using
identical movements, do the same thing with your hand and the fake hand. After 
a while, participants are convinced that the fake hand is their own hand.

\paragraph{Negative phantom limbs}
It happens when there is a physical limb but there is no ``connection'', the
person do not recognize the existence of the body part, also called xenomelia.

Strong desire for amputation, it happens almost exclusively in men, well-educated,
elevated incidence of non-heterosexuality and non-righthandedness. Xenomelia is
accompanied by reduced response to tactile stimulation below the desired
amputation line, also, is accompanied by reduced thickness and volume in Superior
Parietal Lobe (SPL) and smaller area in Inferior Parietal Lobe, primary
somatosensory (S1) and secondary somatosensory (S2).
Xenomelia: cause or consequence?

\paragraph{Hemiphanton}
Patients with a sensorimotor hemisyndrome often do not perceive/acknowledge
the disorder called \textbf{anosognosia}. They often deny ownership over
paralyzed side, they speak of other person: \textbf{somatoparaphrenia}. This
other person frequently behaves in a hostile way, is disliked by the patient and
aggressively treated: \textbf{misoplegia}.

\paragraph{Whole-body phantom or phantom double or doppelg\"anger}
The phantom double or doppelg\"anger is purely somaesthetic illusionn (it is
felt but not seen). Patients usually point a very precise localization in space,
in a neurological context it is almost exclusively after \textbf{parietal lobe
lesions}. Indirect evidence that ``felt presence'' is another self: vague sense
of familiarity, shared feelings, imitation of body movements.

\subparagraph{Autoscopic phenomena}
It is defined as a visual experience where the subject sees an image of 
him/herself in external space, viewed from within his/her own physical body.
Autoscopic phenomena consist of out-of-body experience (OBE), autoscopic
hallucination (AH), and heautoscopy (HAS). Figure~\ref{fig:autoscopic-phenomena}
shows graphically the difference between the three types.
\begin{itemize}
\item OBE: During an OBE, people seem to be awake and feel that their ``self'',
or center of awareness, is located outside of the physical body and somewhat
elevated. It is from this elevated extrapersonal location that the subjects
experience seeing their body and the world.
\item AH: During an AH, people experience seeing a double of themselves in 
extrapersonal space without the experience of leaving one's body.
\item HAS: The individual experiencing an HAS also has the experience of seeing
a double of himself in extrapersonal space. However, it is difficult for the
subject to decide whether he/she is disembodied or not and whether the self is
localized within the physical body or in the autoscopic body (usually lesions on
left posterior insula).
\end{itemize}

\begin{figure}[h]
  \centering
  \includegraphics[width=0.5\textwidth]{imgs/autoscopic-phenomena.png}
  \caption{Types of autoscopic phenomena. Figure from [\ref{ref:out-of-body}]. }
  \label{fig:autoscopic-phenomena}
\end{figure}

\paragraph{Whole-body phantom experiments}
To induce the whole-body phantoms one can use the following experiments:
\begin{itemize}
\item out-of-hand experience: mirror box after Ramachandran enables movement in a
paralyzed phantom hand by visual observation of the reflection of the still
existing hand.
\item out-of-body experience: feeling of being ``out of body'' induced by
observing own body in fron of oneself. Also, touch observed on other body
simultaneously to touch felt on own body induces feeling of being at the place
of seen touch.
\end{itemize}

phantom limbs can reflect perceptuo-motor memories, innate or empathically
acquired programs and phantom sensations can be induced by intersensory conflicts.

\paragraph{todo} Finger and thustwitt, 2003, Neurosurgery 52.

%%---------------------------------------------------------------------------%%
\newpage
\section{Clinical Neuroscience}

%%---------------------------------------------------------------------------%%
\subsection{Neurology: Ophthalmology, Otology, Epileptology and Parkinson - 03.04.2017}
\subsubsection{Neuro ophtamology}
\paragraph{Clinical eye testing}  
\begin{itemize}
\item moviments
\item smooth pursuit
\item saccadic eye moviment - fast eye moviment
\subitem disorders: velocity, metrics (can ``pass'' the point), latency
\item nystagmus
\subitem primary gaze: indicative of cerebellum loss $\rightarrow$ gaze holding
and rebound
\subitem vestibulo occular reflex: the head moves and the eyes move together
and then, after a while the eyes turn to the object-goal position
\end{itemize}

\paragraph{Clinical balance testing}
\begin{itemize}
\item spontaneous nystagmus: eyes drift to the side of the loss.
\item head impulse test: negative $\rightarrow$ eye stays in position;
positive: $\rightarrow$ eyes follow head and then turn.
\item vertical: ``close'' one eye and the other stays in the same position
\item dynamic: see letters moving the head
\end{itemize}

\paragraph{vertical occular deviation}
\paragraph{dynamic visual acuity}

\paragraph{sensorymotor balance:} romberg test: close eyes and balance is lost.
\paragraph{caloric testing}
\paragraph{bimalleolar vibration sense}

\subsubsection{Epilepsy}
\paragraph{What is epilepsy?}
It is a brain disorder, a chronic condition.

\paragraph{What is a seizure?}
It is a temporary disruption of normal brain function.
\begin{itemize}
\item eye moviment, body tension
\item depending on the anatomical location of the seizure, can be all body or
only an arm for instance
\end{itemize}

\paragraph{EEG}
pyramidal cells in the cortex: sleep and close eyes produce high variations.

\paragraph{Seizures does not mean epilepsy.}
Some seizures can be provocated.

\paragraph{Classification of seizures:}
\begin{itemize}
\item focal: part of the brain
\item generalized: whole brain; always without consciouness.
\end{itemize}

\paragraph{EEG measures synchrony post-synaptic potential}

\paragraph{Hypersynchronization}
spikes and sharp edges (seizure)

\paragraph{generalized seizures}
neurons die

\paragraph{Treatment}
\begin{itemize}
\item surgery: small area
\item pharmacotherapy
\item disease specific treatment
\end{itemize}

\subsubsection{Parkinson}
Parkinson is a syndrome not a disease. A disease can be defined as a health
condition that has a clearly defined reason behind it. A syndrome (from the
Greek word \textit{súndromos} meaning ``run together'') however, may produce
a number of symptoms without an identifiable cause.

\paragraph{Variety of symptons}
The symptons from parkinson are always related with a lack of dopamine.
However, too much dopamine is also a disturb but it is not parkinson.

\begin{itemize}
\item slow movements (bradykinesia)
\item tremor at rest
\item stiffness (rigidity of the extremities and neck)
\item postural instability
\end{itemize}

\paragraph{Parkinson's disease}
The common reason for Parkinson's syndrome. It is the second most common
degenerative disorder of the nervous system (Alzheimer's disease is the leader).

The defects in motor function are due to the progressive loss of dopaminergic
neurons in the substantia nigra pars compacta, a population that projects to
and innervates neurons in the caudate and putamen.
The cause of the progressive deterioration of these dopaminergic neurons is not
known, but genetic investigations are providing clues to the etiology and
pathogeneses. Whereas the majority of cases of Parkinson’s disease are
sporadic, there may be specific forms of susceptibility genes that confer
increased risk of acquiring the disease. Two genes are best studied linked with
Parkinson: \textit{$\alpha$-synuclein} and \textit{parkin}.

\paragraph{L-dopa is the most effective therapy to Parkinson's disease}
Dopamine is like insuline for diabetes, however it is a symptomatic-therapy not
a cause-therapy and manage the quantity of dopamine is not an easy task.
In Parkinson the spatial distribution of the degenerating neurons is largely
restricted to the substancia nigra pars compacta, this way, one therapy for
Parkinson’s disease would be to enhance release of dopamine in the caudate and
putamen.

In contrast to most other neurodegenerative disorders, there is effective
temporary symptomatic treatment for PD consisting of DA replacement with
levodopa or DA agonists and adjunctive medications or surgical approaches. To
achieve a true cure, the underlying mechanisms of neuronal cell death need to
be understood, strategies for enhancing neuronal survival and regrowth need to
be developed, and consideration needs to be made toward the replacement of
cells lost during the degenerative process. If these goals can be obtained, a
full recovery could be achieved.

%%-----------------------------------------------------------------------------------------%%
\newpage
\subsection{Multiple Sclerosis, Neuromuscular, Stroke and Neuropsychology}
- 10.04.2017

\subsubsection{Multiple Sclerosis}

Multiple sclerosis~(MS) is a disease of the central nervous system
characterized by a variety of clinical problems arising from multiple regions
of demyelination and inflammation along axonal pathways, \textbf{demyelination
is necessary to diagnose MS: you need MRI information and additional signs}. MS
as a demyelinating disease is deeply accepted in the clinical literature,
although precisely how the demyelination translates into functional deficits is
poorly understood.

A possible explanation of the human disease is that a genetically susceptible
individual becomes transiently infected (by a minor viral illness, for example)
with a  microorganism that expresses a molecule structurally similar to a
component of myelin. An immune response to this antigen is mounted to attack
the invader, but the failure of the immune system to discriminate between the
foreign protein and itself results in destruction of otherwise normal myelin.

MS commonly manifests between ages 20 and 40. It is the most frequent CNS
disease among young adults. The signs and symptoms of MS are determined by the
location of the affected regions.

Abnormalities are often apparent in the cerebrospinal fluid, which usually
contains an abnormal number of cells associated with inflammation and an
increased content of antibodies (a sign of an altered immune response).

The diagnosis of MS generally relies on the presence of a neurological problem
that remits and then returns at an unrelated site, \textbf{MS is characterized
by relapses manifestation: the neurological problem comes and goes}.

\begin{itemize}
\item distructive disease
\item inflamation and shrink (loss of myelin and axons) on brain
\item enviromental effects: more common above equator line
\item diagnostic depend of space and time
\subitem space: not only one area of CNS lesion
\subitem time: because it comes and goes
\item PPMS: primary progressive multiple scelerosis: CSF
\subitem punctions of CSF: if you find antibodies in the CSF and serum it is a
general inflamation, if you find only in the CSF is MS.
\item scale 0 to 10; 10 means death
\item MS became a treatable disease in 90's
\end{itemize}

\subsubsection{Neuromuscular disorders}
\begin{itemize}
\item diagnosis: medical history, neurological examination, lab tests
\item myotonic reaction: is a symptom of a small handful of certain
neuromuscular disorders characterized by delayed relaxation (prolonged
contraction) of the skeletal muscles after voluntary contraction or electrical
stimulation.
\end{itemize}

\subsubsection{Stroke} 
Any acute neurological sympton is a stroke unless proven otherwise. However,
stroke is not a seizure, and also is not a migraine.

\paragraph{Types of stroke}
There are two types: ischemic and hemorrhagic.

\begin{itemize}
\item ischemic: (80\% of the cases) $\rightarrow$ block of a vessel -
recanalization
\item hemorrhagic: (20\% of the cases) blood pressure control / hematoma
evacuation, reduction of intracranial pressure - cateter
\end{itemize}

\paragraph{cells in penumbra are potentially salvable}
penumbra is an area around the vessel. After 270 mins of a stroke,
recanalization can lead to  hemorrhagia.

\paragraph{prognosis}
death or be in a dependent state (approx. 90\% hemorragic and approx. 40\% of
ischemia)

\subsubsection{Neuropsychological disorders }
\begin{itemize}
\item aphasia (language disorders)
\subitem broca $\rightarrow$ patient do not speak but understand
\subitem wesnick $\rightarrow$ spontaneous speech is fluent but it is
impossible to repeat phrases.
\subitem aprosodia $\rightarrow$ patients speak in a monotone, no matter the
circumstances or meaning of what is said.
\item alexia $\rightarrow$ can write but can not read
\item apraxia $\rightarrow$ can use tools but can not immitate their uses
\item anosognosia $\rightarrow$ no notion of its own deficience
\end{itemize}

%%-----------------------------------------------------------------------------------------%%
\newpage
\subsection{Spinal Cord Injury - Dr. Martin Schubert and Dr. Armin Curt}
25.04.2017

Spinal cord injury~(SCI) can be divided into traumatic and non-traumatic
aetiologies.

\paragraph{Traumatic SCI}
Occurs when an external physical impact acutely damages the spinal cord. The
primary injury damage cells and initiates a complex secondary injury cascade,
which produces the death of neurons and glial cells, ischemia and inflammation.

\paragraph{Non-traumatic SCI}
Occurs when an acute or chronic disease process, such as a tumor, infection or
degenerative disc disease, generates the primary injury.
\\
\\
\\
Regardless of the cause, \textbf{SCI can lead to permanent and severe
neurological deficits}. SCI have devastating physical, social and vocational
consequences for patients and their families. Prevetion is the most important
intervention today. For SCI that cannot be prevented, the development of
effective treatments become crucially important.

The incidence of SCI varies worldwide. The incidence of traumatic SCI is higher
in North America, than Australia or Western Europe. The incidence of
non-traumatic SCI is also higher in Canada than in Australia, for instance.
Traumatic SCI occurs more commonly in males ($\approx$ 80\%) than in females
($\approx$ 20\%). The age profile of individuals with traumatic SCI has a
bimodal distribution, with one peak between 15 and 29 years of age and the
second, smaller but growing, above 50 years.

\subsubsection{Diagnosis}
Fractures on the spinal column are often described by their vertebral level,
but the neurological injury is described by the spinal cord level at which the
nerve roots emerge. Different levels of the spinal cord innervate distinct
muscle groups; in general, SCI results in the partial or complete loss of
function below the level of injury. SCIs generally result in loss of
sensorimotor function, but can also affect the sympathetic nervous system.

\paragraph{Sympathetic Nervous System}
Sympathetic Nervous System~(SNS) is responsible for up- and down-regulating in
many homeostatic mechanisms in living organisms. Fibers from the SNS innervate
tissues in almost every organ system, providing at least some regulatory
function to things as diverse as pupil diameter, gut motility, and urinary
system. It is perhaps best known for mediating the neuronal and hormonal stress
response commonly known as the fight-or-flight response. This response is also
known as sympatho-adrenal response of the body, as the preganglionic
sympathetic fibers that end in the adrenal medulla (but also all other
sympathetic fibers) secrete acetylcholine, which activates the great secretion
of adrenaline (epinephrine) and to a lesser extent noradrenaline
(norepinephrine) from it. Therefore, this response that acts primarily on the
cardiovascular system is mediated directly via impulses transmitted through the
SNS and indirectly via catecholamines secreted from the adrenal medulla.
\\
\\
\\
High thoracic or cervical injuries can lead to severe hypotension and
bradycardia. The loss of innervation to secondary lymphatic organs can induce
secondary immunodeficiency.

\paragraph{Spinal Shock}
It is a temporary clinical state of flaccid paralysis post-SCI, including loss
of motor, sensory, autonomic and reflex function at or below the level of
injury. Spinal shock can affect the accuracy of the initial neurological
examination (used to define the severity of the SCI).

\paragraph{Neurogenic Shock}
It is typified by hypotension, bradycardia, wide pulse pressure and warm pink
extremities. It is most clinically relevant with a neurological level of injury
above T6.

After any traumatic injury, the advanced trauma life support protocols dictate
initial care, which includes airway, breathing and circulation support (along
with immobilization of the spinal cord using a rigid cervical colar and
backboard).

Patients with suspected traumatic SCI undergo neurological examinantion (which
includes a voluntary motor and sensory examination of each limb and a rectal
examination) to asses for neurological dysfunction and radiographic imaging to
look for damage to the vertebrae and/or spinal cord.
Plain X-ray, CT and MRI are the most commonly used radiological tools. X-ray
and CT are useul to detect gross fracture dislocation injuries, MRI is well
suited for evaluate the integrity of soft tissue, such as invertebral discs,
ligaments and the spinal cord and nerve roots. Also, MRI can evaluate for the
presence of oedema and/or haemorrhage.

If the SCI is detected, the grade of the injury is classified according to the
American Spinal Injury Association (ASIA) Impairment Scale, which ranges from A
(most severe) to E (least severe).

\paragraph{SCI Syndromes}
\textbf{Central cord syndrome} is the most common incomplete SCI syndrome (15\%
to 25\% of SCIs). Most commonly diagnosed in elderly patients with pre-existing
cervical spondylosis who after a fall results in cervical hyperextension.
Central cord syndrome is characterized by disproportionate motor impairement of
the upper limbs, rather than the lower limbs.

\textbf{Brown-Sequard syndrome} is most commonly observed in individuals with
penetrating traumatic SCI (as gunshots or knife wounds). It is characterized by
loss of motor function, light touch, proprioception and vibration sensation
ipsilateral to the injury and loss of pain and temperature sensation
contralateral to the injury.

\paragraph{Prognosis}
Neurological recovery in patients with SCI is typically observed within the
first six months after injury (continuous improvements can be seen up to five
years later). In general, thoracic injuries are associated with reduced
potential for motor recovery compared with injuries in the cervical or lumbar
spinal cord.
Patients with ASIA Impairment Scale grade A, are generally predict to have less
than 5\% chance of walking 1 year after the injury. There are some tools that
predict neurological recovery and long-term walking potential considering for
instance age, neurological examination and MRI feature.

\subsubsection{Pathophysiology}
The initial trauma (first injury) causes displacement or dislocation of the
vertebral column, which causes compression or transection of the spinal cord.
This causes a focal region of cell death and blood-spinal cord barrier
disfunction (it initiates a cascade of secondary injury mechanisms). Secondary
injury mechanisms includes vascular changes, inflammation, loss of ionic
homeostasis and oxidative stress (increasing the damage to the spinal cord).
Cell death contributes to the formation of cystic cavities, which are
surrounded by a glial scar (a deposition of astrocytes and molecules that
inhibit neuronal regeneration), which interfere with neuronal regeneration and
functional recovery.

Traumatic SCI is pathophysiologically divided into primary and secondary
injuries. And can also be temporally divided into the acute ($<$ 48 hours),
subacute (48 hours to 14 days), intermediate (14 days to 6 months) and chronic
($>$ 6 months). The pathophysiology of the traumatic spinal cord is shown in
Figure~\ref{fig:SCI-pathophysiology}.

\begin{figure}
  \centering
  \includegraphics[width=0.5\textwidth]{imgs/SCI-timeline.png}
  \caption{Pathophysiology of the traumatic spinal cord}
  \label{fig:SCI-pathophysiology}
\end{figure}

The first injury damage neurons and oligodendrocytes (myelinating cell type of
the CNS), disrupts the vasculature and compromises the blood-spinal cord
barrier. Together, these events immediately initiate a secondary injury
cascade.

\paragraph{Acute phase}
Secondary cellular changes in the acute phase are caused by cell
permeabilization, pro-apoptotic signalling and ischaemic injury due to the
destruction of the microvascular supply of the spinal cord within minutes of
injury. In addition, blood vessel injury can cause severe haemorrhages, which
can expose the cord to an influx of inflammatory cells, cytokines and
vasoactive peptides.

The inflammatory response in the accute and subaccute phases, combined with the
disrupted blood-spinal cord barrier, progressively add to spinal cord swelling,
leading to further mechanical compression of the cord, which can extend for
multiple spinal segments and worsen the injury.

\paragraph{Subacute phase}
In this period, ischaemia and excitotoxicity contribute to a loss of
intracellular and extracellular ionic homeostasis. Ongoing necrosis of neurons
and glia due to ischaemia, inflammation and excitotoxicity releases ATP, DNA
and potassium, which can activate microglial cells. Activated microglia
infiltrate the injury site, where they propagate the inflammatory response and
contribute to ongoing apoptosis of neurons and oligodendrocytes. Phagocytic
inflammatory cells can clear myelini debris at injury site but can also induce
further damage to the spinal cord. High levels of the neurotransmitter
glutamate are released from dying neurons and astrocytes and are poorly
reabsorbed by surviving astrocytes. This causes a receptor overactivation,
which combined with the loss of ATP-dependent ion pump functions and subsequent
resultant sodium dysregulation, can lead to excitotoxic cell death.

The multiple causes of cell death that occur during the acute and subacute
phases of SCI can produce greater damage than the original primary injury and
form the basis for the neuroprotective intervetions.

\paragraph{Intermediate-chronic phase}
The spinal cord lesion evolves to intermediate through to chronic phases that
are marked by attempts at remyelination, vascular reorganization, alterations
in the composition of the extracelular matrix (ECM) and remodelling of neural
circuits.

In humans, cell death in acute phase promotes \textbf{cystic cavities}, which
contain extracellular fluid. These cavities become a formidable barrier to
directed axonal regrown and are a poor substrate for cell migration. In the
acute phase, signalling from activated microglia, astrocytes and macrophages
causes the secretion of proteins that are inhibitory to axonal growth, which
condense with astrocytes to form the \textbf{glial scar}, that potently
restricts axon regeneration. But also, the glial scar isolates the injury site
to reduce the spread of cytoxic mollecules and inflammatory cells.

Even if regenerative efforts are able to overcome spinal cord lesions,
properties of the adult mammalian CNS can still limite neurite regrowth.
Molecules present in myelin are potent inhibitors of axon regeneration, and
several molecules released by degenerating oligodendrocytes can contribute to
failure of regeneration. 
Although severe SCI can destroy substantial portions of the spinal cord white
matter, a surviving subpial rim of demyelinated axons can persist in rodent
models of injury. Endougenous mechanims exists for at least partial
regeneration of the injuried spinal cord. CNS neurons exhibit both anatomical
and synaptic plasticity, which might contribute to ongoing functional recovery
for years after injury. These and other mechanisms can be supplemented by the
neuroprotective and neuroregenerative strategies but barriers to regeneration
still exist, meaning additional therapies to remove or overcome these barriers
are necessary.

\paragraph{Animal models}
Animals models have contributed to our understanding of the pathophysiology of
SCI and have been useful for the preclinical testing of new therapies. The
ideal animal model should anatomically and pathophysiologically resemble humans
SCI, require minimal training, be inexpensive and produce consistent results.
Rat models are the most commonly used for SCI research, however, differences in
size, molecular signalling, anatomy and the recovery potential following SCI
have made direct translation challenging. Large animal models, such as
non-human primates, overcome some of these barriers, but substantial
differences in cost and unique housing requirements make their use less common
and even they are unable to exactly mimic human SCI.

\subsubsection{Management of patients with SCI}
Functional neural tissue is progressively lost during the hours after SCI,
making it crucially important to rapidly diagnose patients and implement
neuroprotective intervations during the acute phase.

The management of patients includes monitoring of haemodynamics and
decompressive surgery to limit the damage to the spinal cord. In addition,
patients can receive a neuroprotective agent (its use is controversial). Long
term management includes monitoring and treating systemic complications of
injury as autonomic dysreflexia, neuropathic pain, pressure sores, respiratory
complications, gastrointestinal and urinary disturbance, etc.

\paragraph{Decompressive Surgery}
Surgical intervention is an essential basis of the acute treatment for patients
with spinal trauma and acute SCI. Surgery typically involves open reduction and
decompression paired with an instrumented fusion to stabilize the spinal column
in an antomical position. Decompressing the spinal cord early after SCI should
help to limit the zone of injury and improve clinical outcomes (some studies
shown better prognosis when the decompression is done within 24 hours than
after 24 hours).

\subsubsection{Complications}
Can be \textbf{local} or \textbf{systemic}. \\

Locals:\\
\paragraph{Neuropathic arthropathy}
Slow progressive destruction of a joint. Can lead to deformity, overlying skin
ulceration and potentially fatal infections. The loss of sensation that is
common after SCI allows repeated microtraumas to go unnoticed, which promotes
bone resorption.

\paragraph{Spasticity}
Is the velocity-dependent increase in muscle tone with exaggerated deep tendon
reflexes that results from injury to upper motor neurons. Can substantially
affect mobilization, activities of daily living and sleep. Treatment may
include physical therapy, systemic pharmacological treatments, local botulinum
toxin injections or surgery.
\\
\\
\\
Systemics:
\paragraph{Cardiovascular}
Compromise of sympathetic flow can lead to hypotension. As a result, 60\% of
the patients experience symptomatic hypotension (dizziness, weakness and
syncope).
Treatment includes the use of lower extremity compression stockings, medical
management (such as the use of hydration, salt tablets etc) and/or peripheral
vasoconstriction.

\paragraph{Disreflexia}
It is an urgent condition that most commonly occurs in patients with injuries
at or above T6. It is caused by the presence of a noxius stimulus, which causes
a reflex overstimulation of spinal sympathetic neurons, leading to vaso
constriction and dangerous acute hypertension. Increase on the blood pressure
can lead to heart attack. Treatment requires upright positioning of the
patient, removal of the triggering stimulus and pharmacological
anti-hypertensives.

\paragraph{Respiratory}
Paralysis of the phrenic nerve, intercostal muscles and/or abdominal muscles
leads to reduced lung capacity, ineffective cough and accelerated fatigue with
respiratory demands. As a consequence, recurrent pneumonia and pleural effusion
happens. Respiratory complications are the leading cause of mortality in
patients with chronic SCI. In individuals with high cervical injuries, or those
with poor respiratory reserve, lifeloong ventilator dependency can be used.

\paragraph{Secondary immunodeficiency}
The disruption of CNS input to immune organs can result in the systemic
dysfunction of macrophages and natural killer cells in a process known as
immune paralysis. This increase susceptibility to infections (pneumonia,
urinary tract or wound infections).

\paragraph{Genitourinary and gastrointestinal}
Dysfunction of the genitourinary and gastrointestinal system increases care
requirements, the risk of infection and can be a source of substantial social
and psycological stress in patients. Injuries at or above L1-L2 interrupt
innervation of the detrusor, or the bladder muscle, and urinary sphincters,
which can cause an inability to empty the bladder, acontractile bladder,
urinary incontinence and recurrent infections. Management includes urethal
catheterization every few hours, surgical creation of a urinary stoma,
injections of botulinum toxin and pharmacological therapies. The neurological
level of injury can also affect sexual function, injuries above T11 can affect
psychological erection or vaginal lubrification and ability to orgasm, even
with preservation of the reflexive erection/vaginal lubrification. SCI can
interrupt the voluntary control of the anal sphincter and or the
parasympathetic innervation of the bowel. Both cases lead to constipation, an
increased risk of infection and stress. Treatments range from dietary fibre
intake, digital rectal stimulation and the use of suppositories.

\paragraph{Pressure sores}
It cause pain, increased care requirements and can be life-threatening if not
promptly treated. Prevention of pressure sores requires daily inspection and
cleaning of the skin, but also a relief of the pressure on each region every
few hours.

\paragraph{Neurogenic heterotopic ossification}
It is a process that form ectopic bone in the connective tissue around joints.
This ossification occurs most commonly on large joints and presets with
localized pain, redness, low-grade fever and increased spasticity.

\paragraph{Neuropathic pain}
It is experienced by up to 40\% of patients with chronic SCI and can have a
substantial effect on the psychological well-being and quality of life of
patients. Neuropathic pain can be treated pharmacologically (antidepressants,
anticonvulsants and/or opioids), surgically (implanted spinal cord stimulator)
or through non-allopathic treatments (acunpucture, massage etc).

%----------------------------------------------------------------------------%%
\newpage
\subsection{Epilepsy - 02.05.2017}
Epilepsy is one of the most common neurological disorders affecting about 8 in
1000 inhabitants (prevalence of epilepsy: approxi. 0.8\%). It is characterized
by recurrent \textbf{seizures}.

In some patients epilepsy seizures may not be the only symptoms. Especially
syndromes like temporal lobe epilepsy can also have \textbf{psychiatric
aspects}, i.e, mood disorders and anxiety disorders can often be associated
with epilepsies while ictal (during the seizure) or interictal (between the
seizures) psychotic disorders are less frequent.

Typical ictal (during the seizure) symptoms are alterations of cousciouness,
motor, sensory and psychic events. After the seizure (postictal) there is a
complete recovery or transient functional deficit as Todd's paralysis, aphasia,
depression, fatigue etc. Between seizures (interictal) there are few symptons
(if any) as depression, cognitive deficits, etc.

It seems likely that abnormal activity generates plastic changes in cortical
circuitry that are critical to the pathogenesis of the epilepsy. The importance
of neuronal plasticity in epilepsy is indicated most clearly by an animal model
of seizure production called kindling. To induce a kindling, a stimulating
electrode is implanted in the brain, often in the amygdala. At the beggining of
the experiment, weak electrical stimulation, in the form of a low-amplitude
train of electrical pulses, has no discernible effect on the animal’s behavior
or on the pattern of electrical activity in the brain. As this weak stimulation
is repeated once a day for several weeks, it begins to produce behavioral and
electrical indications of seizures. By the end of the experiment, the same weak
stimulus that initially had no effect now causes full-blown seizures. This
phenomenon is essentially permanent; even after an interval of a year, the same
weak stimulus will again trigger a seizure.

The behavioral manifestations of epileptic seizures in human patients range
from mild twitching of an extremity to loss of consciousness and uncontrollable
convulsions.

Modern thinking about the causes (and possible cures) of epilepsy has focused
on where seizures originate and the mechanisms that make the affected region
hyperexcitable.

\paragraph{Prejudices}
The most common prejudices are: epilepsy is contagious, is a mental disease,
patients with epilepsy are mentally retarded and every seizure is
life-threatening.

Many names in history had epilepsy such as Alexander, the Great, Julius Cesar,
Dante Alighieri, Alfred Nobel and many others.

\paragraph{Epidemiology}
5 to 10\% of the population have seizures and approximately 0.6 to 1\% of the
population have epilepsy. The first manifestation occurs equally in the
childhood/adolescence, in adulthood or at an older age.
The mortality rate between epilepsy patients is almost three times as high as
healthy individuals in Switzerland. In Switzerland, the mortality rate is
7.8/1000 individuals per year, while for epilepsy patients is 23/1000 per year.

\subsubsection{Epilepsy}
Strictly speaking, epilepsy is not a single disease but a disorder that can
have many different causes. \textbf{Epilepsy is a chronic disease defined by
recurring and unprovoked seizures}.
So there are many types of epilepsies that can be classified e.g. according to
their different etiologies (origins). Using this criterion we can distinguish
between two main classes of epilepsies: \textbf{idiopathic} and
\textbf{symptomatic}.

\paragraph{Idiopathic epilepsies}
Idiopathic epilepsies are genetically determined and cannot be treated
surgically. However, most patients with such syndromes respond very well to
medical treatment.
Although we know, that idiopathic epilepsies usually run in families, the
\textbf{genetics of epilepsies} has only begun to be explored: while some
special syndromes have been identified (e.g. ``autosomal dominant nocturnal
frontal lobe epilepsy'' with mutations on chromosomes 1, 15, and 20) the
genetic mechanisms of idiopathic epilepsies are still unclear.

\paragraph{Symptomatic epilepsies}
Symptomatic epilepsies are caused by identifiable alteration of brain tissue
(tumors, scar, inflammation etc). For patients with symptomatic epilepsies who
do not respond well to antiepileptic drugs the surgical treatment can be
investigated as a possible treatment.

\subsubsection{Seizure}
Seizure are caused by transient functional disturbances of cerebral neurons
leading to excessive and/or synchronous discharges.

There are many different types of seizures whose symptoms depend on the
specific brain regions involved. 
Note that \textbf{seizures are not always a sign of epilepsy}, they can
sometimes also be provoked by high fever, substance abuse, sleep deprivation or
metabolic disorders like diabetes in persons who do not suffer from epilepsy.

Among epileptic seizures we can again distinguish between two main classes:
\textbf{generalized} and \textbf{focal}.

\paragraph{Generalized seizures}
Show signs of involvement of both cerebral hemispheres from beginning onward
and are usually associated with loss of consciousness.

\paragraph{Focal (or partial) seizures}
Show signs of involvement of only one specific brain region – at least
initially. These seizures may or may not be associated with impaired
consciousness.

The seizure threshold for each seizure type is shown in
Figure~\ref{fig:seizure-threshold}.

\begin{figure}
  \centering
  \includegraphics[width=0.5\textwidth]{imgs/seizure-threshold.png}
  \caption{Seizure threshold for different types of seizures}
  \label{fig:seizure-threshold}
\end{figure}

\subsubsection{First aid}
It is necessary do not panic if one sees a person during a seizure. Watch the
person and consult a watch, usualy seizures don't last longer than 2-3 minutes.
If necessary rescue the person out of danger zone and protect his/her head.
Do not restrain the person and do not put anything in the person's mouth.
Observe the person after the seizure, if necessary position he/she in the
lateral recumbent position. If the seizures continues for longer than 5
minutes, call emergency physician.

\subsubsection{Diagnosis}

The diagnostic work-up tries first to establish whether a patient really
suffers from epilepsy. If he or she does, we try to define the specific
epilepsy syndrome: the symptoms and findings that comprise a particular type of
epilepsy defined by its etiology, occurring types of seizures, course and
prognosis etc.

Since the specific symptoms of seizures depend on the functions of the brain
regions involved in the epileptic brain activity, the \textbf{semiology of
focal seizures}, i. e. the signs and symptoms of a seizure and their temporal
sequence can be analyzed with the aim of localizing the seizure origin in the
brain.
Other examinations usually performed during the diagnostic work-up include
checking the patietn history, blood and CSF tests, \textbf{magnetic resonance
imaging (MRI)} of the brain in search of morphological correlates of the
epieptogenic focus and, especially \textbf{electroencephalography}, recordings
of brain electrical activity in search of epileptiform potentials like spikes
or sharp waves.

These examinations are also performed to exclude non-epileptic attacks
indicative of differential diagnoses like syncope, movement disorders,
non-epileptic psychogenic seizures etc., which should not be confused with
epilepsy.

\subsubsection{Treatment}

Epileptic seizures are symptoms of epilepsies, and medical treatment aims at
preventing seizures. Thus, antiepileptic drugs can make a patient seizure-free
but they cannot treat the underlying cause of the epilepsy.

Seizure freedom can be accomplished medically in about 60\% to 70\% of all
epilepsy patients. In those patients, in whom seizures recur in spite of
adequate medical treatment, it is sometimes possible to offer epilepsy surgery,
which can be very successful in certain epilepsy syndromes like e.g. medial
temporal lobe epilepsy.

Presurgical evaluation relies not only on imaging and EEG-studies including
seizure recordings but also on neuropsychological examinations, which can
demonstrate specific cognitive deficits like, for example, memory deficits
induced by epileptogenic activity and/or lesions in specific brain areas.
Therefore, \textbf{neuropsychology in epileptology} can help to localize the
epileptogenic focus, to predict potential risks of epilepsy surgery and to
identify possible cognitive side effects of antiepileptic drugs.

\paragraph{Palliative epilepsy surgery}
Aims at reducing the frequency and/or the severity of seizures not controlled
by other means.

\paragraph{Curative epilepsy surgery}
Aims at removing the epileptogenic zone to establish seizure-freedom without
causing any additional neurological or neuropsychological deficits.
Before it is possible to decide for or against surgical treatment, the chances
and risks of the operation must be evaluated presurgically. In some patients
this also necessitates the implantation of intracranial electrodes to record
seizures directly from within the brain. In addition, intracranial electrodes
can be used for functional electrostimulation mapping of motor, sensory and
cognitive functions to increase the safety of cortical resections.
Thus, in the presurgical evaluation of pharmacoresistant focal epilepsies the
frontiers between neurophysiology and neuropsychology become blurred. This
becomes especially clear in mesial temporal and mesial frontal epilepsies,
which show that \textbf{limbic seizures and functions} tap aspects of
declarative memory and emotional processing.
In the \textbf{temporal lobe epilepsy} the declarative (conscious) memory is
impaired while the procedural (unconscious) memory is intact.

%%---------------------------------------------------------------------------%%
\newpage
\subsection{Depression}

Feeling sad and blue are common and universal moods experienced in response to
transient loss, stress, or disappointment. 
The clinical diagnosis of major depressive disorder~(MDD) involves a persistent
symptom complex that includes some combination of depressed mood, diminished
interest or pleasure, changes in sleep and appetite, excessive guilt, agitation
or retardation, loss of energy, diminished ability to think or concentrate,
thoughts of death, and thoughts of suicide.
The community lifetime prevalence is approximately 12 to 15\%, with the lowest
rates in Asian countries and the highest in the Americas, Europe, and
Australia.

The rates of MDD are two- to threefold higher in women than men.

Different subtypes of major depression have been identified, including
depression with psychotic features and a typical depression.
Depression is an extremely disabling disorder. About 75\% of patients have
recurrent episodes. Ten to 30\% recover incompletely, with persistent residual
symptoms. MDD also complicates the course of cardiovascular illness, diabetes,
hypertension, and other chronic medical conditions. Suicide, accidents, and the
risk of death from heart disease are highest among the depressed.

Three common questionaries for depression:
\begin{itemize}
\item Beck depression inventory
\item Hamd questionary
\item one more ??
\end{itemize}

Self, future, not be able to feel joy.

Health people have positive bias. Realistic people are subdepressed.

%%---------------------------------------------------------------------------%%
\newpage
\subsection{Schizophrenia - 08.05.2017}

Schizophrenia is a catastrophic illness with an onset typically in adolescence 
or early adulthood. It was identified and defined by Eugen Bleuler (Swiss
psychiatrist) in 1911. The combination of significant incapacity, onset early
in life, and chronicity of illness makes schizophrenia a particularly tragic
disorder, occurring in 0.5 to 1\% of the population.

A substancial body of evidence indicates that schizophrenia is associated with
both \textbf{genetic and environmental factors}. Currently, it remains an open
question whether these environmental factors play a necessary role in the
neurodevelopmental abnormalities responsible for psychotic brain disorders or
whether their role is additive or interactive with genetic contributions.

People with schizophrenia are more likely than other to have schizophrenic
relatives. The most compelling evidence is the 50\% concordance rate for
monozygotic twins relative to 15\% concordance for dizygotic twins.

Some risk factors:
\begin{itemize}
\item heritability (DISC1/2, COMT)
\subitem Disc - hippocampus alteration
\subitem Comt - dopamine dysregulation
\item Lower social status
\item Paternal age (father)
\item pregnancy/birth complications and postnatal infections
\end{itemize}

\subsubsection{Characteristics and Symptoms}

The clinical symptoms are divided into two categories: \textbf{positive} and
\textbf{negative}. The positive symptoms represent distortions or exaggerations
of normal cognitive or emotional functions. The negative symptoms reflect a
loss or diminution of normal function.

Some of the positive symptoms are:
\begin{itemize}
\item delusions (false beliefs that don't go away even with evidence that they
aren't true)
\item hallucinations (hearing or seeing things that are not real)
\item incoherence
\item disorganized speech
\item disorganized or bizarre behavior
\item psychomotor disturbance
\end{itemize}

Negative symptoms:
\begin{itemize}
\item attention deficits (attentional impairment)
\item lack of willpower/initiative
\item apathy
\item social withdrawal
\item alogia: poverty of speech or speech devoid of coherent content
\item affective flattening: diminuition in the ability to express emotions
\item anhedonia: inability to experience pleasure
\item avolition: inability to initiate or persist in goal-directed behavior
\end{itemize}

The profound and pervasive cognitive and emotional disturbances that
characterize schizophrenia suggest that it is a serious brain disease affecting
multiple functions and systems (for instance, auditory hallucinations and
disruptions in linguistic expression suggest involvement of the auditory cortex
and perisylvian language regions). Other symptoms, such as delusions or
disorganized behavior, are more difficult to localize to specific regions and
suggest dysfunction distributed across multiple neural systems and circuits.
Negative symptoms may be related to the prefrontal cortex, which mediates
goal-directed behavior and fluency of thought and speech.

* Usually related with religion \\
* Mixing topics - interpolation - can not keep one subject (usually people say
thought is getting louder)\\

\paragraph{Neuropathological alterations}
The most consistently described structural alterations in schizophrenia are
ventricular enlargement and decreases in the volume of temporal lobe
structures, including the hippocampus, amygdala, and entorhinal cortex.
Some structural differences were find in the brain:
\begin{itemize}
\item less activity in Prefrontal cortex: helps people think logically and
organize their thoughts.
\item activity on visual and auditory cortices: the pattern of an
hallucionation is equivalent to when people are really seeing or hearing. For 
the schizophrenic there is almost no way to distiguish them.
\item large basal ganglia: might affect movement patterns.
\item small amygdala: responsible for basic feelings (like fear, lust and
hunger). Schizophrenics usually have few emotion.
\end{itemize}

Also, there are some neurotransmitters differences like:
\begin{itemize}
\item dopamine: among other things, it is responsible for feeling pleasure.
Links the basal ganglia with the prefrontal cortex. In some parts of the brain
the level of dopamine is low (might affect the pleasure state) and in other
parts are high (maybe responsible for hallucionations).
\end{itemize}

\subsubsection{Diagnosis}
Difference between American system and European: symptoms for one month or two
months respectively. It must be clear that any of these symptoms are not due to
depression or to neuroleptic medication.

One of the symptoms for at least one month:
\begin{itemize}
\item thought echo
\item delusions of control, influence (body or limb movements)
\item hallucinatory voices giving a running commentary on the patient's
behavior
\item persistent delusions of other kinds that are culturally inappropriate
(or impossible: e.g. be able to control weather, communication with aliens
from  another world, etc.)
\end{itemize}

Two of the symptoms for at least one month:
\begin{itemize}
\item persistent hallucinations
\item breaks or interpolations in the train of thought (resulting in
incoherence, irrelevant speech or neologisms)
\item catatonic behavior
\item ``negative'' symptons
\end{itemize}

\subsubsection{Therapy}
The therapy of psychotic disorders is based on medication, psychoterapy,
rehabilitation.

\paragraph{Medication}
Antipsychotic drugs are used but they must be with low dosage start (to reduce
side effects), use of tranquilizer (benzodiazepines) during the acute phase.
The treatment in adequate dosage until remission, sometimes, there is a
treatment resistance (need to change to another antipsychotic drug).
In the long term treatment side effects are crucial, as the psycho-social
reintegration (rehabilitation), for this, family therapy has an important role.

\paragraph{Psychoterapy}
The goals of psychoterapy are separated in categories like: diagnostics (``what
is happening to me?''), symptom reduction (``the voices should disappear''),
well-being (``I want to feel better''), optimism (``my future is important, I
want to live a normal life''), work (``I want to complete my training'') and
becoming an adult (``I want to be independent of my parents'').

10\% of patients suicide in the first year of schizophrenia - need to take care
of medications.

In Switzerland (and some other countries) you can't medicate (treat) patients
against their will.

\paragraph{Dopamine in Schizophrenia}
Schizophrenia is associated with abnormalities in the dopaminergic synapses of
the brain. The strongest link between dopamine synapses and schizophrenia comes
from studies of drugs that alleviate the symptoms of schizophrenia.
A large number of antischizophrenic (or neuroleptic) drugs have been found,
most of them belong to two chemical families: phenothiazines (including
chlorpromazine) and butyropherones (including haloperidol). These drugs have
two properties in common: they block postsynaptic dopamine receptors and they
inhibit the release of dopamine from the presynaptic neuron. One interpretation
of these results is that schizophrenia is due to excess activity at dopamine
synapses. Another possibility is that the underlying problem in schizophrenia
is not an excess of dopamine activity at all, but a deficit of glutamate
activity \footnote{Glutamate is the predominant excitatory amino acid released
by neurons in the cerebral cortex that project widely throughout the limbic
system, and dopamine synapses are known to inhibit glutamate release in these
target regions.}.
If glutamate release is reduced in the schizophrenic brain, one way to correct
this deficiency would be to block activity at dopaminergic receptors, relieving
glutamate synapses from inhibition.

%%-----------------------------------------------------------------------------%%
\newpage
\subsection{Addiction Clinics - Dr. Marcus - 15.05.2017}

Drug addiction is a chronic, relapsing disease with obvious medical, social,
and political consequences. Addiction (also called substance dependence) is a
persistent disorder of brain function in which compulsive drug use occurs
despite serious negative consequences for the afflicted individual.

\subsubsection{General Information}
\paragraph{Global Burden of Disease (2010)}
Alcohol and ilicit drug use account for 5.4\% of world's annual disease burden
(impact of a health problem as measured by financial cost, mortality,
morbidity, or other indicators), with tobacco responsible for 3.7\%.

\paragraph{Mortality in Switzerland (2007)}
\begin{enumerate}
\item Tobacco (9.201 deaths)
\item Alcohol (1.993 deaths)
\item Suicid (1.360 deaths)
\item Traffic accidents (384 deaths)
\end{enumerate}

\paragraph{Harm related to the use of drugs}
The harm caused by drugs can be divided in two categories: to the user itself
and to others, the subdivision of each categorie is shown in Figure~\ref{fig:harm-drugs}.

\begin{figure}
  \centering
  \includegraphics[width=0.7\textwidth]{imgs/harm-drugs.png}
  \caption{Different types of harm related to the use of drugs}
  \label{fig:harm-drugs}
\end{figure}

Considering the overall harm caused by drugs (to the user + to others), the
drugs that cause most harm are:
\begin{enumerate}
\item Alcohol (72 pts harm score)
\item Heroin (55 pts harm score)
\item Crack-cocaine (54 pts harm score)
\end{enumerate}
The cocaine itself appears in the fifth position with 27 pts of harm score.

\subsubsection{Dependence Syndrome (ICD-10)}
To consider that exist a \textbf{dependence syndrome, three of more of the
following manifestations should have occurred together for at least one month}
or if persisting for periods of less than one month, should have occurred
together repeatedly within a 12-month period.

\begin{itemize}
\item Strong desire or compulsion to take the substance
\item Impaired capacity to control substance taking behaviour
\item A physiological withdraw state when substance use is reduce or ceased
\item Evidence of tolerance to the effects of the substance (need for
significantly increased amounts of the substance)
\item Preocupation with substance use by alternative pleasures (pleasures or
interests being given up or reduced because the substance use)
\item Persistent substance use despite harmful consequences
\end{itemize}

\paragraph{From first use to dependence} 
The dependence is caused by environmental and genetic factors. A single (first)
use is not enough to become dependent.
However, some drugs need less time to create a dependence. In order of most
addictive drugs: cocaine $\rightarrow$ alcohol/nicotine $\rightarrow$ cannabis.

\subsubsection{Alcohol}

The alcohol abuse/dependece is the mental disorder that shows the highest gap
(92\%) between policy and practice (not only in Europe).

\paragraph{Stigmas of Alcohol, Depression and Schizophrenia}
todo: ``Addiction is a Brain disease, and it matters''

\paragraph{Alcohol risks for health consequences}

Table~\ref{table:alcohol-risks} shows the level of risk that the comsuption of
alcohol represents for the health. For comparison, one bottle of wine (750ml,
12\% vol) corresponds to 70g of alcohol and 7 drinks. In general, 10g of
alcohol is found in 100ml of wine with 12\% vol, or 330ml of bier with 4\% vol,
or 30ml of alcohol beverages with 40\% vol.

\begin{table}
  \begin{tabular}{ p{4cm} | p{5cm} | p{5cm} }
    \hline
    risk & men & women \\ \hline
	\hline
    low & 0-4 drinks per day (40g alcohol concentration) & 0-2 drinks per day
    (20g alcohol concentration) \\ \hline
    medium & 4-6 drinks per day (40-60g alcohol concentration) & 2-4 drinks per
    day (20-40g alcohol concentration) \\ \hline
    high & 6-10 drinks per day (60-100g alcohol concentration) & 4-6 drinks per
    day (40-60g alcohol concentration) \\ \hline
    very high & $>$10 drinks per day ($>$100g alcohol concentration) & $>$6
    drinks per day ($>$60g alcohol concentration) \\ 
    \hline
  \end{tabular}
  \caption{Alcohol risks for health consequences}
  \label{table:alcohol-risks}
\end{table}

\paragraph{Treatment goals}
In the early 19th century started the \textbf{temperance movement} promoting
abstinence and a ``moral'' life, aiming at a society without alcohol and drugs.
In the early 20th century, this movement spread around Europe and became a
popular social movement. Alcohol use was considered a lack of virtue, the
treatment was based on work therapy and pedagogic measures to overcome ``lack
of willpower''.

At the beggining of the treatment, approximately half of the patients with
alcohol abuse, would like to stop drinking completely (abstinence), and the
other half would like to reduce the consumption of alcohol (moderate drinking,
without dependency). However, preferences change over time, after four weeks,
49\% of the patients change from reduction to abstinence as prefered treatment
goal.

Nowaday, current guidelines sugest that is best engange the patient in reduce
the alcohol consumption at first than force an abstinence.

The general \textbf{treatment goals} are preservation or restoration of health
and social integration. Abstinence or a moderate use that does not
substantially affect health and social environment as well as the treatment of
concurrent diseases (psychiatric and somatic) are suited to reach this goal.

Some treatments are pharmacotherapy, psycotherapy and social support.

\subsubsection{Heroin}
It was used as a pain killer. There was an open drug scene in Zurich around
1980/1990. In 1991 was introduced the \textbf{Four Pillar Strategy}: not drug
free, but drug use that is socially compatible.
\begin{itemize}
\item prevention
\item therapy
\item harm reduction
\item regression
\end{itemize}

\paragraph{Opioid agonist treatment}
Opioid agonist therapy (OAT) is an effective treatment for addiction to opioid
drugs such as heroin, oxycodone and others. The therapy involves taking the
opioid agonists methadone (Methadose) or buprenorphine (Suboxone). These
medications work to prevent withdrawal and reduce cravings for opioid drugs.
People who are addicted to opioid drugs can take OAT to help stabilize their
lives and to reduce the harms related to their drug use.
Methadone and buprenorphine are long-acting opioid drugs that are used to
replace the shorter-acting opioids the person is addicted to. Long-acting means
that the drug acts more slowly in the body, for a longer period of time. By
acting slowly, it prevents withdrawal for 24 to 36 hours without causing a
person to get high. OAT also helps to reduce or eliminate cravings for opioid
drugs. Treatment works best when combined with other types of support, such as
individual or group counselling. 

This treatment has good evidence of efficacy resulting in reduction of heroin/
cocaine use, reduction of mortality, improviment in quality of live, increase
in treatment retention.

In 2015, 33k people died of opioid in USA.

\subsubsection{Cocaine}
Cocaine is still an issue in Zurich, top 3 cities in europe
(first place: Antwerp, Belgium, second place: Amsterdam, Holland).

\paragraph{Patient interview}
cocaine addicted, the interview was in German, without translation.

\subsection{Addiction in Society - Prof. Boris Quednow - 15.05.2017}

The vast majority of the people who regularly consume psychoactive drugs are
NOT addicted nor will they ever become addicts.

\subsubsection{Psychoactive drug consumption}

The American Psychiatric Association defines addiction in terms of both
physical dependence and psychological dependence (in which an individual
continues the drug-taking behavior despite obviously maladaptive consequences).

In addition to a compulsion to take the agent of abuse, a major feature of
addiction for many drugs is a constellation of negative physiological and
emotional features, referred to as “withdrawal syndrome,” that occur when the
drug is not taken. 

Although positive reinforcement is clearly necessary for the development of
drug use, it may fall short in explaining the development of compulsive use.
So, what drives the drug consumption in some people and why/how do some of them
become addicted? What factors distinguish drug use from abuse and dependence or
addiction?

\paragraph{Opponent-process theory}
According to \textbf{opponent-process theory}, drug addiction is the result of
an emotional pairing of pleasure and the emotional symptoms associated with
withdrawal. At the beginning of drug or any substance use, there are high
levels of pleasure and low levels of withdrawal. Over time, however, as the
levels of pleasure from using the drug decrease, the levels of withdrawal
symptoms increase, thus providing motivation to keep using the drug despite a
lack of pleasure from it.

\paragraph{Incentive-sensitization model}
According to \textbf{incentive-sensitization model}, the motivation to abuse
substances must be a force a lot stronger than merely liking something.
Everybody will have things they like but it won't change their behavior in the
same way that an addicted will. Liking is not a good explanation for addition.
Instead addicts have developed a powerful motivation that is called incentive
salience. Incentive salience is an intense type of wanting because the brain
develops a strong association between a stimuli and a reward (pathological
reward learning). This association develops subconsciously but it can soon
start to influence outward behavior. This compulsion to get more of the stimuli
can be incredibly strong. Thus, addicteds may no longer even like the substance
but they still feel compelled to use it.

This way, \textbf{``want'' vs ``like'' dissociates over time}. The wanting
sensation is mediated by dopamine and the liking sensation is mediated by
endogenous opioids.


\subsubsection{Addiction as pathological learning and memory}
The transition from occasional to compulsive drug use and the persistent
vulnerability to relapse are due to neuroadaptations in brain circuits
implicated in reward, memory, drive, and control.

Figure~\ref{fig:circuits-drug-addiction} shows a schematic relation between the
circuits underlying addiction. Reward (nucleus accumbens and ventral pallidum),
memory (amygdala and hippocampus), control (pre frontal cortex and anterior
cingulate gyrus) and drive (orbital frontal cortex and subcallosal cortex). In
a nonaddicted brain, these circuits are balanced, this results in proper
inhibitory control and decision making. However, during addiction, the enhanced
expectation value of the drug in the reward, drive, and memory circuits
overcomes the control circuit, this favors a positive-feedback loop initiated
by the consumption of the drug and perpetuated by the enhanced activation of
the drive and memory circuits. 

\begin{figure}
  \centering
  \includegraphics[width=0.5\textwidth]{imgs/circuits-drug-addiction.png}
  \caption{Circuits underlying addiction}
  \label{fig:circuits-drug-addiction}
\end{figure}

\subsubsection{Stages of addiction}
Drug addiction, also known as substance dependence, is a chronically relapsing
disorder characterized by (1) compulsion to seek and take the drug, (2) loss of
control in limiting intake, and (3) emergence of a negative emotional state
when access to the drug is prevented.
Drug-taking behavior progresses from impulsivity to compulsivity in a
three-stage cycle: binge/intoxication, withdrawal/negative affect and
preoccupation/anticipation. Positive reinforcement (pleasure/gratification) is
more closely associated with impulse control disorders. Negative reinforcement
(relief of anxiety or relief of stress) is more closely associated with
compulsive disorders.

\paragraph{Stage I}
The stage I is the binge/intoxication stage. It is marked by a positive
reinforcement by the drug, rewarding effects mediated by dopamine and
endogenous opioids (VTA to Nacc), the stimulus-response habit learning is
enhanced, and associative learning of context cues happens.

\paragraph{Stage II}
The stage II is the withdrawal/negative effect stage. In this stage there is a
negative emotional state involving the extended amygdala. This can induce
stress and anxiety-like effects, it projects to hypothalamus and brain stem.
The endogenous opioids decrease, and then the negative reinforcement happens
(drug seeking to avoid withdrawal).

\paragraph{Stage III}
The stage III is the preoccupation/anticipation (craving) state. It is a state
with high vulnerability to relapse (even after prolonged abstinence) where
conditioned drug-associated cues or stress can elicit strong drug craving,
which in turn makes relapse more likely. Disrupted PFC funcion is crucial for
this stage.

Approximately 10 to 20\%  of drug users become addicted at some point during
their lifetime. Among the factors leading to addiction, there are
biological/gens and environmental factors, some are listed in
Figure~\ref{fig:drug-addiction-factors}.

\begin{figure}
  \centering
  \includegraphics[width=0.7\textwidth]{imgs/drug-addiction-factors.png}
  \caption{Factors leading to drug addiction}
  \label{fig:drug-addiction-factors}
\end{figure}

\subsubsection{Molecular target of drugs}
There are three classes of drugs accordign to their mechanism of action.

\begin{itemize}
\item I - binds to G-protein - metabotropic receptors - Opioids, Cannabinoids,
Hallucinogens
\item II - binds to ionotropic receptors - Nicotine, Alcohol, Benzodiazepines
\item III - interacts with monoamine transporters (GABA receptors) - Cocaine,
Amphetamine
\end{itemize}

All addictive drugs increase dopamine levels (in animal models). Addiction
liability and reinforcing effects are mainly mediated by i) mesolimbic
projection (VTA - ventral tegmental area to NAc - Nucleus Accumbens), ii)
mesostriatal projection (substancia nigra to dorsal striatum) and iii)
mesocortical projection (VTA to PFC).

\paragraph{Dopamine and reward: pharmacokinectic profile of drug}
Drugs with fast uptake leads to a fast drug effect (reward), more intense high
and stronger reinforcing effects.
The reward elicits approach and consumatory behavior, increases frequency and
intensity of behavior leading to reward a maintain learned behavior and induces
subjective feelings of pleasure. But, when the reward is predict but doesn't
happens, dopaminergic neurons code the discrepancy between the prediction and
occurrence of rewards, that is crucial for approach behavior. If the drug do
not ``come'' the brain seeks the drug $\rightarrow$ absence of dopamine.

The dopamine release facilitates learning. Drugs elicit more dopamine release
for greater duration than natural resources of reward (food, sex, social
interaction). Upon repeated administration tolerance does not develop to drug-
induced dopamine release, but natural rewards are weaken.

By blocking reuptake or enhancing release, cocaine and amphetamine increase
the synaptic availability of dopamine, norepinephrine and serotonin. However,
the acute reinforcing effects of these drugs depend critically on dopamine.
Low doses of dopamine receptor antagonists injected either systemically or
centrally into the nucleus accumbens, amygdala, or bed nucleus of the stria
terminalis block cocaine and amphetamine self-administration in rats.

\subsubsection{Perpetuation of drug addiction}
Why do addicts continue to take drugs if reward is mediated by dopamine release
in the mesolimbic pathway and addicts exhibit attenuated drug-induced when
dopamine increases?

Because of the conditioning. Association of environmental cues with the drug
may underlie intense desire for drug. The dopamine increase by conditioned cues
are larger than those produced by drugs itself. The use of drugs is compulsive
when exposed to drug cues.

\paragraph{Serotonin and addiction}
Without dopamine, one can still get addicted if there is serotonin in the body.
Cocaine and other psychostimulants increase serotonin activity in the brain.
Serotonin depletion in PFC reduces the reinforcing properties of cocaine in
conditioned place preference~(CPP) in rats. \textbf{A functional serotonin
system is required for establishing psychostimulant consumption and related
behaviors in animals}.

%%---------------------------------------------------------------------------%%
\newpage
\subsection{Neurosurgery - Jorn Fierstra - 22.05.2017}
\paragraph{Highly Specialized Medicine}
Applying and mastering the most modern and complex techniques and cutting edge
knowledge currently available.

Specialized in surgeries for brain and spinal pathologies (most brain surgery).
Brain: Neurovascular (Aneurysms), Neuro-oncology (Tumors), Epilepsy,
Hydrocephalus, Deep Brain Stimulation, Trauma, Infectious diseases.
Spine: Spinal canal stenosis, Herniated discs, Neurovascular, Neuro-Oncology.

UZH/ETH is an outstanding place.

\subsubsection{Brain pathologies}
\paragraph{Aneurism}
An aneurysm is an arterial condition in which the wall of an artery weakens,
creating a bulge, or distension of the artery. An aneurysm can occur in
important arteries such as those supplying blood to the brain, and the aorta.
The bulge in an artery forms as a result of a weakening of the artery wall,
which allows the blood pressure to distend the artery wall wider than usual. A
cerebral aneurysm is usually saccular. This shape also accounts for most cases
of ruptured brain aneurysms. Ruptured cerebral aneurysms are the most common
cause of a type of hemorrhagic stroke. This type of stroke is less common than
ischemic strokes - strokes caused by a blocked artery rather than an internal
bleed.

Most aneurysms do not themselves cause any symptoms, however, a large aneurysm
may obstruct circulation to other tissues. An aneurysm can also contribute to
the formation of blood clots that then obstruct smaller blood vessels,
potentially causing ischemic stroke or other serious problems; this is known as
thromboembolism. Otherwise, aneurysms tend to produce symptoms only when there
are complications such as rupture. Brain aneurysm can lead to hemorrhage; a
symptom of the stroke bleed is a sudden extreme headache.

When an aneurism is found, it is necessary determine if it must be treated.
Many people do not have any symptom and can live with it. Approximatelly 50\%
of aneurism will result in bleeding, this way some people do not need surgery.

\paragraph{Brain tumor} 30 cases per year
\paragraph{1st image} tumor from dura
\paragraph{3rd image} glioma (butterfly)
\paragraph{4th image} surgery, microscope, neural navigation
\paragraph{5th image} tumor that do not ``appear'', glowing tumor

\paragraph{Bypass}
Cerebral bypass surgery is performed to restore, or ``revascularize'', blood
flow to the brain. A cerebral bypass is the brain's equivalent of a coronary
bypass in the heart. The surgery connects a blood vessel from outside the brain
to a vessel inside the brain to reroute blood flow around a damaged or blocked
artery. The goal of bypass surgery is to restore blood supply to the brain and
prevent strokes.

\paragraph{Elana - Excimer Laser Assisted Non Occlusive Anastomosis}
The only proven non-occlusive bypass technique in Neurosurgery, only technique
available to make a bypass in the brain without temporary occlusion of the
blood-flow. This can significantly diminish the risk of stroke. The Elana
procedure is designed to make traditional bypass procedures safer and easier.

\subsubsection{Spine Pathologies}
\paragraph{Neck Injuries and Disorders}
\paragraph{Tumor spinal cord} 

\paragraph{Intraoperative MRI}
Intraoperative magnetic resonance imaging (iMRI) refers to an operating room
configuration that enables surgeons to image the patient via an MRI scanner
while the patient is undergoing surgery, particularly brain surgery. iMRI
reduces the risk of damaging critical parts of the brain and helps to confirm
that the surgery was successful or if additional resection is needed before the
patient’s head is closed and the surgery completed.

\subsubsection{Neurosurgery challenges} 

\begin{itemize}
\item Understanding the origin of neurological disease
\item advance treatement and thereby outcome of patients with neurological
disease
\item advance current diagnostic and therapeutic modalities available to us
\end{itemize}

It is necessary to have better imaging methods that auxiliate to figure out the
precise area to be removed in a surgery, for instance.

\subsubsection{Novel MRI - clinical neuroscience research}
An emerging technique for mapping cerebrovascular re-activity (CVR) is blood
oxygen level-dependent MRI (BOLD MRI) of changes in cerebral blood flow (CBF)
during manipulation of end-tidal partial pressure of carbondioxide. It uses a
3 Tesla MRI together with RespirAct device that change CO\textsubscript{2} and
O\textsubscript{2}.

Respiract: changes in CO\textsubscript{2} to contract/dilate veins in brain.
RespirAct™ is an investigative tool that's designed to control end-tidal gas
concentrations in the lungs and blood. Failure of the blood vessels to dilate
or contract at the necessary time can starve tissue for oxygen. RespirAct™ is
involved in researching diagnostic procedures that measure vascular reactivity,
i.e., the ability of the blood vessels to regulate blood flow in reaction to
varying circumstances which affect blood supply and demand. We're currently
investigating the control of C0\textsubscript{2} and 0\textsubscript{2} for a
number of applications. 

Neurovascular uncoupling
finger tapping - image effects
sensitive to threshold

%%---------------------------------------------------------------------------%%
%%%%%%%%%%%%%%%%%%%%%%%%%%%%%%%%%%%%%%%%%%%%%%%%%%%%%%%%%%%%%%%%%%%%%%%%%%%%%%%
%%---------------------------------------------------------------------------%%
\newpage
\section{Previous exams}

Note: this answers were provided by students and were not verified by a teacher. Use them at your own risk.

\subsubsection{2005}
\begin{enumerate}
\item \paragraph{NeuroImmunology - A student in a major Neurobiology lab attempts to generate and purify recombinant rabbit myelin basic protein (MBP).  He uses a eukaryotic expression system.  Suddenly, his cell-cultures become contaminated by bacteria.  He has already spent a lot of time on his project and does not want to tell his supervisor about his mishap.  The student decides to ignore the contamination and purify his protein.  He further decides to assess the level of contamination by injecting the protein into mice.  He reckons that if they developed septic shock syndrome, he would have to tell his boss and start from scratch.  If the mice were fine, he would not tell his boss about his blunder and go ahead as planned. After injecting his protein, the mice do not develop septic shock and the student is ecstatic.  However, after one week, the mice develop hind-limp paralysis and a student from the Neuroimmunology lab next door tells him that his mice appear to suffer from EAE.  The student now wonders whether this is to be expected after MBP injection or not.  You can surely help:}
\subparagraph{Why did the mice develop this autoimmune disease?  Please name step-by-step events leading to CNS-inflammation.}
\subparagraph{Does the contamination have anything to do with the mice developing EAE?  Explain the four mechanisms of peripheral tolerance and provide examples!  Hint: the narrative above provides one of the four examples you need.}
\subparagraph{Explain why we have auto-aggressive T cells and the events occurring in the thymus “usually” lead to central tolerance.  What is positive and ngegative selection?}

\item \paragraph{Imaging - Abstract of a paper (Goel V. and Dolan RJ: The functional anatomy of humor: segregating cognitive and affective componnents.}
\subparagraph{What is shown in figure 1 (statistical map) - explain how it is generated.}
\subparagraph{what is shown in figure 2 (modulation in ROI) - explain.}
\subparagraph{How do you interpret the activation in ventromedial prefrontal cortex?}
\item \paragraph{Psychiatry II - Explain pathogenesis of Alzheimer and reveal therapeutic approaches.}
\item \paragraph{Different aspects of MS are listed.Weight the importance of the different aspects and their impact for treatment} See answer \ref{question:MS-aspects}.
\item \paragraph{Neuroinformatics - What are the organizing principles of neural systems? Explain on examples of retinal stereopsis and why the brain uses them.}
\item \paragraph{Sleep research - Why is sleep divided into REM and non-REM? Why is slow-wave-activity meaningful in non-REM-sleep?}
\end{enumerate}

\subsubsection{2006}
\begin{enumerate}
\item \paragraph{Imaging - Abstract from Nature: A general mechanism for perceptual decision-making in the human brain (2004). Figure 1 (behavioral data) and Figure 2 (fMRI data) were given.}
\subparagraph{explain behavioral results}
\subparagraph{explain fMRI results}
\subparagraph{what is the signal measured with fMRI?}
\subparagraph{What is a statistical map and to what extent does it reflect neuronal activity?}
\subparagraph{is the conclusion drawn from this study supported by data?}

\item \paragraph{Disease models - What are the arguments in favour and against the protein-only hypothesis of prion propagation?}
\item \paragraph{Brain and behavior - Compare and contrast two research strategies in the study of schizophrenia with an emphasis on how they may complement each other}
\item \paragraph{Sleep and circadian rhythm}
\subparagraph{Define terms: sleep homeostasis; circadian sleep-wake rhythm}
\subparagraph{Describe: their main physiological markers; their practical manifestations}
\item \paragraph{Neuroinformatics}
\subparagraph{differences in organising principle between electronic and neural computation}
\subparagraph{ilustrate neural computation principles by specific examples (e.g. retina), explain functional utility.}
\item \paragraph{Multiple Sclerosis}
\subparagraph{Where are the predilection sites in MS lesion? (Where are Plaques most frequently? Reasons?}
\subparagraph{Which cellular or histological features are typical for MS plaques?}
\subparagraph{What is the role of the BBB (blood brain barrier) in MS pathogenesis and by which techniques can it be elucidated?}
\subparagraph{Describe mechanisms of demyelination and axonal damage and mechanisms of remyelination and axonal repair.}
\subparagraph{Which forms of MS do you know and how are they characterized?}
\subparagraph{What is the prevalence rate in Switerland and how are such figures determined?}
\subparagraph{Main clinical features (typical symptoms) of chronic MS.} See answer \ref{question:MS-symptoms}.
\subparagraph{Forms of therapy}
\subparagraph{In what respect are therapies effective and in which respect less so?}
\end{enumerate}

\subsubsection{2013}
\begin{enumerate}
\item \paragraph{FMRI}
\item \paragraph{Microglia}
\item \paragraph{Circadian Rhythms}
\item \paragraph{Motor Learning}
\item \paragraph{Parkinson}
\item \paragraph{Multiple Sclerosis}
\end{enumerate}

\subsubsection{2011}
\begin{enumerate}
\item \paragraph{ fMRI is routinely used to study the neural processes underlying behavior. Please describe all the procedures necessary for conducting and correctly interpreting an fMRI study, covering the following areas:}
\subparagraph{Data acquisition (what signals are measured in fMRI, and how?)}
\subparagraph{Data analysis (which sequential routines are necessary to detect signal changes in fMRI images, using software packages such as SPM?)}
\subparagraph{Results interpretation (what inferences about neuronal activity can be drawn from fMRI results?}

See answer \ref{question:fMRI-complete}.

\item \paragraph{Please list the different types of long-term memory you know of. Describe their properties in humans, group them according to involved brain structures and give examples of behavioral tests that allow to model these memory types in rodents}

See answer \ref{question:long-term-memory}.

\item \paragraph{Sleep regulation in physiological short and long sleepers: Explain the most important principles how sleep and wakefulness are physiologically regulated and how sleep-wake regulation may differ between habitual short and long sleepers.}

See answer \ref{question:sleep-physiological}.

\item \paragraph{Robotic tools have played a significant role in the investigation of human motor
learning.}
\subparagraph{Describe the role of internal models in human motor control and how such
models are acquired}
\subparagraph{Identify three unique features of robotic systems that make them valuable tools to investigate human motor learning.}
\subparagraph{Discuss how these unique features could be applied to clinical assessment and therapy of sensorimotor impairments} answer here.

\item \paragraph{The term ``frontotemporal dementias'' subsumes a heterogeneous group of disorders:}
\subparagraph{Please describe the clinical presentations of patients with frontotemporal dementia (major clinical syndromes, and characteristic features).}
\subparagraph{Which genes/gene loci have been associated with frontotemporal dementia?}
\subparagraph{Please summarize which major molecular subgroups of frontotemporal dementias can be defined and briefly discuss current knowledge and/or hypotheses on underlying pathomechanisms in the two most common subgroups.}

\item \paragraph{The maintenance of central and peripheral tolerance is the reason that autoimmune diseases are relatively rare. Please answer the following questions:}
\subparagraph{How does central T cell tolerance work (which organ performs T cell education, what is negative and positive selection)?}
\subparagraph{What are the mechanisms of peripheral tolerance? Remember, we discussed four of them. Please shortly recapitulate.}
\subparagraph{Why does the inflammation in an MS-lesion subside after a while? What mechanisms can dampen an ongoing immune response (if you do not know, speculate!)}

\end{enumerate}

%%%%%%%%%%%%%%%%%%%%%%%%%%%%%%%%%%%%%%%%%%%%%%%%%%%%%%%%%%%%%%%%%%%%%%%%%%%%%%%
\newpage
\subsection{All Question - topics}
%%---------------------------------------------------------------------------%%
%%%%%%%%%%%%%%%%%%%%%%%%%%%%%%%%%%%%%%%%%%%%%%%%%%%%%%%%%%%%%%%%%%%%%%%%%%%%%%%
%%---------------------------------------------------------------------------%%
\subsubsection{Methods}
\begin{enumerate}
\item \paragraph{fMRI is routinely used to study the neural processes
underlying behavior. Please describe all the procedures necessary for
conducting and correctly interpreting an fMRI study, covering the following
areas:}\label{question:fMRI-complete}

fMRI (functional Magnetic Resonance Imaging) is a non invasive technique that
employs principles of magnetic resonance that are sensitive to blood
oxygenation and are fast to acquire.

\subparagraph{Data acquisition (what signals are measured in fMRI, and how?)}

To acquire data it is necessary follow three steps:
\begin{enumerate}
\item Place an object (brain) in a strong magnetic field: protons in the body
have spins with a specific orientation and frequency. When the body is inside
an MRI scanner, the protons align with the direction of the magnetic field.
\item Apply radio waves: radio frequences pulses with the appropriate frequence
(Larmor frequency) change the orientation of the spins as the protons absorb
the energy. When the pulse is turned off, the protons return to their original
orientations (this process is called relaxation), and during relaxation the
protons emit energy in the form of radio waves.
\item Measure radio waves emitted by the object (brain):  two measures can be
acquired - T1 longitudinal and T2 transverse. T1 is how quicly the protons
realign with the magnetic field and accurately distinguish different types of
tissue, T2 is how quicly the protons emit energy when recovering equilibrium
and relate changes in MR-signal to an experimental manipulation.
\end{enumerate}

One of the most commons signal used to relate changes in the MR-signal to the
experimental manipulation is the BOLD (Blood Oxygenation Level Dependent)
signal. This signal measures inhomogeneities in the magnetic field (T2) due to
changes in the level of O\textsubscript{2} in the blood. This way, fMRI
measures neural activity indirectly. The oxygenated blood is non magnetic
while the deoxygenated blood is magnetic. When a specific region of the cortex
increases its activity in response to a task, this region consumes the oxygen
leading to an initial drop in oxygenated hemoglobine and an increase in local
carbon dioxide and deoxygenated hemoglobine.

\subparagraph{Data analysis (which sequential routines are necessary to detect
signal changes in fMRI images, using software packages such as SPM?)}
The data analysis using software such as SPM allows standardised detection of
activity changes in each voxel, and for this three macro steps are necessary:
\begin{enumerate}
\item Preprocessing: in the preprocessing phase we can realign the images to
fix small head movements, normalise the data to increase sensitivity with more
subjects, to extrapolate findings to the whole population, to make the results
comparable among different studies, also, we can smooth the image to increase
signal to noise, impreve inter-subject average.
\item Model estimation: calculates parameter for instance, from GLM of voxel
timeseries.
\item Contrasts and SPM: does the statistical inference.
\end{enumerate}

\subparagraph{Results interpretation (what inferences about neuronal activity
can be drawn from fMRI results?}

The fMRI results (using BOLD signal) are not an absolut measure, the results
can differ from session to session due to differences in the scanner, in the
subject, etc. This way, BOLD signals need to be compared between different
conditions within the same experiment to infer BOLD changes, for example, we
acquire signal for the task P and also signals from a control without the task
P. The difference between the two acquisition will be the result of the task P.
This is an approach that assumes a ``pure insertion'' theory, where cognitive
(and neural) process can be added to others withou changing them and the change
in behavior (and in brain activity) reflects only the added process. Also, the
fMRI results are correlative results, i.e., they can show that signals from a
brain region co-occur with a task of interest but cannot show that a region is
necessary for that function.

\item \paragraph{What are the physiological correlations of fMRI signal? How
does the fMRI signals correlate with neuronal activities?}

The physical correlation of fMRI is neural activity, resulting in an initial
(about 0.5-2s) ‘undershoot’ of the proportion of oxygenated hemoglobin, due to
the consumption of oxygen for the neural activity. This leads to a reduction of
the BOLD (blood oxygen level dependant) signal. Neural activity seems to
mediate vasodilation (maybe through to release of NO), leading to an increase
of blood flow (after 2-10s), resulting in an increase of the BOLD signal due to
the better blood supply. Studies comparing BOLD data with EEG data have shown
that the BOLD signal rather reflects the information uptake and processing by
neurons than their spiking output measured by EEG.

\item \paragraph{Functional magnetic resonance imaging~(fMRI) is one of the
most widely-used techniques for non-invasive studies of human brain function.
fMRI exploits the so-called “blood oxygen level dependent” (BOLD) contrast.}

\subparagraph{Please discuss the biophysical basis of BOLD: how is BOLD linked
to changes in regional magnetic field inhomogeneities and regional blood flow
during neuronal activation?}

BOLD measures inhomogeneities in the magnetic field due to changes in the level
of O\textsubscript{2} in the blood. This way, fMRI measures neural activity
indirectly. The oxygenated blood is non magnetic while the deoxygenated blood
is magnetic. When a specific region of the cortex increases its activity in
response to a task, this region consumes the oxygen leading to an initial drop
in oxygenated hemoglobine and an increase in local carbon dioxide and
deoxygenated hemoglobine.

\subparagraph{Please discuss what aspects of neuronal activity are most closely
respected by BOLD and which biochemical processes link neuronal activity to 
vasodilation and vasoconstriction.}

Neurovascular coupling refers to the relationship between local neural activity
and subsequent changes in cerebral blood flow (CBF). The BOLD signal is based
on this coupling and measures neural activity indirectly.
The cerebral metabolism depends on a constant supply of both glucose and
oxygen. A continuous supply of these two energy substrates is maintained by
cerebral blood flow~(CBF), which delivers glucose and oxygen to neural tissue
through the complex web of blood vessels in the brain’s vascular system. During
neural activity, increases in oxygen and glucose consumption are followed by an
increase in CBF.
Action potentials at pre-synaptic cell leads to neurotransmitter release (e.g.,
glutamate, GABA) that open ion-channels on post-synaptic cell. Re-uptake of
glutamate and pump ions out of cell again uses energy and oxygen, this triggers
blood vessel dilation and decrease ratio of deoxygenated/oxygenated blood, i.e,
decrease in magnetism, increasing the BOLD signal strength. A neuronal activity
inhibition will lead to a vasoconstriction correspond to a decrease in blood
oxygenation, and a decrease in the BOLD signal.

\subparagraph{Please summarize the basic principles of statistical parametric
mapping, the standard approach to analyzing fMRI images.}

The SPM identify functionally specialized brain responses and is the most
prevalent approach to characterizing functional anatomy and disease-related
changes.
The basic pipeline includes three steps:
\begin{itemize}
\item preprocessing: images are spatially aligned to each other to correct for
the effect of subject movement during scanning (realignment/motion correction),
then spatially normalised into a standard space and smoothed.
\item modelling: parametric statistical models are applied at each voxel (a
volume element, the three-dimensional extension of a pixel in 2D) of the data,
using a general linear model (GLM) to describe the data in terms of
experimental effects, confounding effects and residual variability.
\item statistical inference: classical statistical inference is used to test
hypotheses that are expressed in terms of GLM parameters. This results in an
image in which the voxel values are statistics: this is a statistical
parametric map (SPM).
\end{itemize}

\item \paragraph{What is the neurological basis for BOLD fMRI?}

One of the most commons signal used to relate changes in the MR-signal to the
experimental manipulation is the BOLD~(Blood Oxygenation Level Dependent)
signal. This signal measures inhomogeneities in the magnetic field (T2) due to
changes in the level of O\textsubscript{2} in the blood. This way, fMRI
measures neural activity indirectly. The oxygenated blood is non magnetic while
the deoxygenated blood is magnetic. When a specific region of the cortex
increases its activity in response to a task, this region consumes the oxygen
leading to an initial drop in oxygenated hemoglobine and an increase in local
carbon dioxide and deoxygenated hemoglobine.

\end{enumerate}

%%---------------------------------------------------------------------------%%
%%%%%%%%%%%%%%%%%%%%%%%%%%%%%%%%%%%%%%%%%%%%%%%%%%%%%%%%%%%%%%%%%%%%%%%%%%%%%%%
%%---------------------------------------------------------------------------%%
\subsubsection{Perception and Attention}
\begin{enumerate}
\item 
\end{enumerate}

%%---------------------------------------------------------------------------%%
%%%%%%%%%%%%%%%%%%%%%%%%%%%%%%%%%%%%%%%%%%%%%%%%%%%%%%%%%%%%%%%%%%%%%%%%%%%%%%%
%%---------------------------------------------------------------------------%%
\subsubsection{Decision Making}
\begin{enumerate}
\item \paragraph{Describe the process of decision making and the areas of the
brain with significant BOLD signal during the process}

The decision-making process is the ensemble of computations performed in the
brain in order to take an action in response to external stimuli.
It can be schematized as:

REPRESENTATION $\rightarrow$ VALUTATION $\rightarrow$ ACTION SELECTION
$\rightarrow$ ACTION PERFORM AND EVALUATION $\rightarrow$ LEARNING

It is strongly related to the perception process, due to the necessity to get a
proper representation of the outer world before taking an action. Important
stage in this process is the learning phase, which influences how all the other
steps are performed. Complex meta-analyses of the brain during experiments show
that the areas most involved are:

\begin{itemize}
\item vmPFC - ventral medial prefrontal cortex
\item dlPFC - dorsal lateral prefrontal cortex
\item PCC - posterior circular cortex
\end{itemize}

Several experiments show also relationships with striatum and amygdala, in
particular referring to the learning phase, and OFC~(occipital frontal cortex)
and ACC~(anterior cingulate cortex) referring to willingness to face some kind
of cost in action selection.

\item \paragraph{Describe an experimental setup to study the decision process
in humans}

Evaluation of the decision process in humans can be done using the BDM
auctions. The experiment structure is:

\begin{itemize}
\item Keep away the subject from food, to increase the attractiveness of the
foods
\subitem set the subjects in an “hungry state”, so we ensure they have interest
n getting the food the bid for.
\item Show them food (100 trials, 50 items) and ask them to bid to get it
\item Force them to wait 30 mins and provide them with food
\subitem the wait prevent them from thinking to eat immediately after the
experiment is done.
\end{itemize}

The core of the experiment is the auction, in which subjects have to bid to get
the food they see. The bids can be \$0,\$1, \$2 or \$3. A random number P is
selected for each object, being B the subject bid, if B >= P, the food is won
and P is paid, otherwise the food is not won and 0 is paid.
The bid becomes what in economics is called WTP (Willingness To Pay). In order
to get a control group, half of the trials use the free bid system previously
described while the other half uses a forced bid system, where the subject
can’t decide the amount of its bid.

Experimental results show an increase in the BOLD signal in the vmPFC and in
the dlPFC for the free bid trials that doesn’t occur in the forced bid trials,
indicating that these areas are involved into the active decision process.

\item \paragraph{Describe an experimental setup to study the decision process
in primates}

In order to study the decision-making process in primates, the following
experimental setup is prepared:

\begin{itemize}
\item Monkey prevented from drinking
\item Two groups of coloured dots are shown
\item The monkey has to select one of the two groups
\item Reward based on monkey selection is provided
\end{itemize}

Each group represent a different kind of liquid (e.g. water and juice) and the
number of dots represent the amount (e.g. 4 drops vs 1 drop). What the monkey
learns is the relative value of the liquids against each other. Analyses of
BOLD signal of the brain during this task show how vmPFC and dlPFC are
activated during the task. Further, it is possible to perform single neuron
measurement in the monkey OFC (occipital frontal cortex) and see direct
relationship with the task performed and a single neuron activity.

\item \paragraph{Describe an experimental setup to study aversive decision
process in humans}

In order to verify if aversive decision process involves different brain areas,
the BDM auctions can be modified replacing ``desirable'' food with
``undesirable'' ones (e.g. children food), also, intead of keep the subjects
away from food, instruct them to eat at most 2 hours before the test, to
decrease the attractiveness of the food. The resulting BOLD signal is similar
to standard BDM experiment, showing that same brain areas are involved.

\item \paragraph{Comment about complex decision process (multiple factors to be
evaluated) and differences in rewards offered}

Extra experiments have been done to verify differences in activated brain areas
for complex choices (e.g. evaluating color+shape+regularity), showing again a
strong involvement of the vmPFC. It is interesting to note, however, that when
we change the kind of reward (e.g. money vs food) the areas of the vmPFC that
are activated are slightly different, indicating a different processing
according to the kind of reward the subject is going to get.

\item \paragraph{Describe an experimental setup to study the decision process
when deciding for someone else}

The experimental setup to verify if we decide differently if we or someone else
is going to be rewarded is the following:

\begin{itemize}
\item Subjects are coupled. Partners are chosen such that they often make
opposite choices.
\item The subject is informed about who is going to get the reward, if himself
or the coupled subject
\item A dual choice between small, short term reward and large, long term
reward is shown
\end{itemize}

Due to the coupling rule, subjects have to model what they expect the partner
is going to do. Measurements during this experiment show how there is no real
difference in brain activation when deciding for ourselves or for someone else,
but there is a relevant difference when it gets to take a decision or model the
partner decision scheme.

\item \paragraph{Describe an experimental setup to study the action selection
process on rodents}

The experimental setup to evaluate the decision-making process in rodent is
based on a small T structure with two rewards, one large and the other small,
at its ends.
To get the large reward there is some sort of price the mice have to pay: it
could be a delay (e.g. 15 seconds) or a physical effort (e.g. climb a wall).
Placed in this setup with a properly tune reward size, the mice show a
preference of the larger reward, accepting the ``cost'' of the delay/effort.
Afterwards, the mices are divided in 3 groups:

\begin{itemize}
\item Control group (surgery without brain lesions)
\item OFC group (lesion to occipital frontal cortex)
\item ACC group (lesion to anterior cingulate cortex)
\end{itemize}

The test is the repeated and the results show:

\begin{itemize}
\item no behaviour difference for the control group
\item decreased willingness of facing a delay in the OFC group (with a minor
degrade of physical willingness too)
\item decreased willingness of facing a physical effort in the ACC group
\end{itemize}

This indicates an active role of these two areas in the decision process. It is
interesting to notice that despite the damaged brain, the mice can afterwards
learn again their preference for the higher reward, indicating how the learning
process is able to somehow compensate the physical injuries.

\item \paragraph{Describe the sequential sampling process and an experimental
setup to verify its hypothesis}

The sequential sampling process hypothesis describes the decision-making as an
accumulation of evidences, until a certain threshold of certainty is reached
and an action is taken. In order to verify this, the following experimental
setup is prepared:

\begin{itemize}
\item The subject is placed in front of randomly moving dots
\item Some dots are set to move only to right or to left
\item As soon as the subject identifies the direction of movement, it takes a
decision and select the side
\item The delay of the decision is measured
\end{itemize}

This experiment tries to show how faster the decision becomes when the number
of coherent dots is higher, showing that more information lead to a faster
accumulation of evidence and then decision, modelling the process with a drift
diffusion model.

\item \paragraph{Describe what is it meant with conditioning and what brain
areas are involved, bringing examples to support your thesis}

There are two kinds of conditioning: classical (or Pavlovian) and operant (or
instrumental).

Classical conditioning is related to the learn of unconscious or automatic
answer to some external stimuli, while operant conditioning is a conscious
behaviour in which a response is learned due to a reward system (procedure also
called “Reinforcement Learning”).

The first one can be demonstrated in the classical Pavlov’s dog experiment.
Classical conditioning is usually related to amygdala and the case of H.M.
shows it isn’t related to hippocampus. Similarly, the case of S.M. showed the
importance of amygdala: her lesioned brain was unable to learn to react to fear
stimuli.

\item \paragraph{Give a model for learning processes}

A model for learning process is the prediction-error model. The concept behind
it is that every time the brain predicts the outcome of an action, it compares
the prediction with the actual outcome and updates its internal prediction
model. It is described as $\Delta V = \alpha \beta ( \lambda - \Sigma V)$,
where:
\begin{itemize}
\item $\Delta V$ is the change in learning (prediction error update)
\item $\alpha$ and $\beta$ are learning rates
\item $\lambda$ is the amount of unpredicted outcome
\item $\Sigma V$ is the amount learned (accumulated learning)
\end{itemize}

This model sees the learning as an accumulation of evidences that leads to a
minimization of the prediction error. The prediction is updated at each time
step. In humans, the learning phase is often correlated to Striatum/OFC
activity.

\end{enumerate}

%%---------------------------------------------------------------------------%%
%%%%%%%%%%%%%%%%%%%%%%%%%%%%%%%%%%%%%%%%%%%%%%%%%%%%%%%%%%%%%%%%%%%%%%%%%%%%%%%
%%---------------------------------------------------------------------------%%
\subsubsection{Emotions}
\begin{enumerate}
\item \paragraph{How, on a conceptual level, do fear conditioning and
extinction work?}
Fear conditioning is a behavioral paradigm in which organisms learn to predict
aversive events. It is a form of learning in which an uncoditional stimulus
(e.g. an electrical shock) is associated with a conditional (neutral) stimulus
(e.g., a tone), resulting in the expression of fear responses to the originally
neutral stimulus. Eventually, the neutral stimulus alone can elicit the state
of fear. Conditioning stimuli and unconditioning stimuli inputs converge in
amygdala. Amygdala neurons show long term potential (LTP) that facilitates CS
conditioning.

Fear extinction works the same way as fear conditioning but, instead of
generate a response (fear), the response is extinct. This can be obtained by
present the conditioned stimulus alone, so that it no longer predicts the
coming of the unconditioned stimulus, and the conditioned response gradually
stops. However, the complete memory is not ``removed''. Extinction builds a
competing memory.

\item \paragraph{How is it implemented in a human / rodent brain? What are some
empirical findings?}

It is necessary a protein synthesis to keep the memory. If you present, after a
CS conditioning, the tone with a protein synthesis inhibitor, we observe a
disruption in memory and in the paired condition.

\end{enumerate}

%%---------------------------------------------------------------------------%%
%%%%%%%%%%%%%%%%%%%%%%%%%%%%%%%%%%%%%%%%%%%%%%%%%%%%%%%%%%%%%%%%%%%%%%%%%%%%%%%
%%---------------------------------------------------------------------------%%
\subsubsection{Memory}
\begin{enumerate}
\item \paragraph{Please list the different types of long-term memory you know
of. Describe their properties in humans, group them according to involved brain
structures and give examples of behavioral tests that allow to model these
memory types in rodents.}\label{question:long-term-memory}

Long term memory is the retention of information in a more permanent form of
storage (days, weeks, or even lifetime). The LTM is divided in two types:
declarative (or explicit) and nondeclarative (or implicit).
The declarative memory is that one available to consciouness and it is divided
in two types: semantic and episodic. The nondeclarative memory is not available
to consciouness and it is divided in four types: procedural, priming, classical
conditioning and habituation/sensitization.
Semantic memory refer to facts and general knowledge while episodic memory
refer to personal memories. Semantic memory is relaterd with lateral temporal
lobe and parahippocampal gyrus. Episodic memory is related with hippocampus.

The procedural memory refers to skills and habits, and it is related with
striatum, premotor cortex and cerebelum. The priming memory refers to the
tendency to process some perception in the same way, it is related with the
neocortex. The classical conditioning memory refers to the associations and it
is related with amygdala and cerebelum. Lastly, the habituation/sensitization
refers to ignore or pay more attention to determined situations, and it is
related with the reflex pathways.

In rats, we can access the classical conditioning memory using the classical
pavlonian conditioning, this way, we can observe the associated behavior.

\item \paragraph{List techniques used to investigate memory working principles}
Most used techniques to investigate memory are fMRI, PET and
Electroencephalography~(EEG).

fMRI is used to find the structures that support learning/tasks activities,
also used to detect mental process that do not reflect in behavior, can be
used to identify people with real memory problems (detect simulations).

PET is used on study of learning and memory, measures metabolites and
neurotransmitter concentrations.

EEG is used on study of learning and memory consolidation during sleep. It
measures excitatory and inhibitory postsynaptic potentials.

\item \paragraph{Describe the multistore model}

The multistore model is a memory model that take into account the time that a
memory is stored. It is divide in three classes:
\begin{itemize}
\item sensory memory, it is up to 1s, used for routine experiences
\item short term memory (aka, working memory), it is up to 1 minute. It is the
ability to hold information once the present moment has passed.
\item long term memory, over 1 minute and longer.
\end{itemize}

It is importante to notice that long term memory is not permanent: forgetting
is a key process necessary to keep the amount of information stored under
control and avoid memorizing useless concepts and data. Although forgetting is
a normal and essential mental process, it can be pathological, a condition
called amnesia.

\item \paragraph{Describe HM disease and symptoms}
HM was a patient suffering of seizures preventing him from having a regular
life. For this reason, he was subject of a brain surgery, with the removal of
hippocampus. Despite the success of the treatment (he didn’t have any more
seizures) it suffered of a serious retrograde and antegrade amnesia: he was
unable to remember events immediately before the surgery and he lost the
ability to remember new events. This clinical case has been a fundamental step
in understanding the role of hippocampus in episodic memory, which is defined
as the memory allowing us to remember specific events (“episodes”) of our life.
Also, during the analysis of HM impairment, it came out that he was still able
to acquire other kinds of knowledge (e.g. to perform some kind of drawing task)
but without the ability to remember he learned it (e.g. being surprised by how
easy a task was at first attempt, when he actually trained on it several times
before).

\item \paragraph{Provide a scheme of existing long-term memories, with a brief
description of each one and which brain areas are involved in each of them,
based on consciousness memory model}

Long term memory is the retention of information in a more permanent form of
storage (days, weeks, or even lifetime). The LTM is divided in two types:
declarative (or explicit) and nondeclarative (or implicit).
The declarative memory is that one available to consciouness and it is divided
in two types: semantic and episodic. The nondeclarative memory is not available
to consciouness and it is divided in four types: procedural, priming, classical
conditioning and habituation/sensitization.

Semantic memory refer to facts and general knowledge while episodic memory
refer to personal memories. Semantic memory is relaterd with lateral temporal
lobe and parahippocampal gyrus. Episodic memory is related with hippocampus.

The procedural memory refers to skills and habits, and it is related with
striatum, premotor cortex and cerebelum. The priming memory refers to the
tendency to process some perception in the same way, it is related with the
neocortex. The classical conditioning memory refers to the associations and it
is related with amygdala and cerebelum. Lastly, the habituation/sensitization
refers to ignore or pay more attention to determined situations, and it is
related with the reflex pathways.

\item \paragraph{Describe the role of hippocampus in unconscious memory
retrieval and sleep}

HM case demonstrated that hippocampus plays a central role in episodic memory.
Several experiments show how hippocampus become active in particular during REM
sleep, working on the consolidation of the memory and how it is important for
associative memory.

\item \paragraph{Provide a scheme of memory based on processing modes}

The model based on processing modes is divided in three classes:
\begin{itemize}
\item Rapid encoding of flexible associations: responsible for episodic
memories, related with hippocampus and neocortex.
\item slow encoding of rigid associations: procedural memory, classical
conditioning and semantic memory, related with basal ganglia, cerebellum and
neocortex.
\item rapid encoding of single or unitilized items: familiarity and priming,
related with parahippocampal gyrus and neocortex
\end{itemize}

\end{enumerate}

%%---------------------------------------------------------------------------%%
%%%%%%%%%%%%%%%%%%%%%%%%%%%%%%%%%%%%%%%%%%%%%%%%%%%%%%%%%%%%%%%%%%%%%%%%%%%%%%%
%%---------------------------------------------------------------------------%%
\subsubsection{Body Perception}
\begin{enumerate}
\item \paragraph{Define phantomology}
Phantomology is the science of the virtual reality of the body. It is the
science of the body in the brain, from out-of-limb to out-of-body experiences.

\item \paragraph{What is phantom limb? What types of phantom sensations exist?}
Phantom limb refer to a sensation/persistend experiences of the postural and
motor aspects of a limb after its physical loss.
There are four groups of phantom sensations:
\begin{itemize}
\item phantom body parts: the feeling of the body part without the body part
itself. Can occur after amputation or even in congenital limb deficiency.
\item hemiphantom: the feeling that half of the body is an entity that has its
own life.
\item phantom double: the feeling of have a second body that imitates the
original one.
\item out of body experiences: the feeling of be disconnected with its own
body.
\end{itemize}

Also, even without the loss of the limb, one can have phantom sensation, it is
called spinal phantom limb.

\item \paragraph{What is phantom pain? Describe some theories about why it
happens.}
Phantom pain is the feeling of a pain in the body part that it is not attached
to the body. It is not well clear why phantom pain happens. One theory
(maladaptive model) says that the pain is the price for the plasticity of the
brain. The loss of inputs in a determined area generates an "invasion" from
neighbors areas that causes pain. Another theory (maintained representation
model) states that the representation of the phantom in the brain is maintaned
give the pain as input.

\item \paragraph{Why some people with congenital limb deficiency experience
phantom sensation?}
It is not well know why this happens, but there are three main explanations.
\begin{itemize}
\item projection of enhanced motility of rudiments (if some part of the limb is
there but not enterily)
\item contralateral representation of intact limb
\item innate body schema
\end{itemize}

\item \paragraph{What is the obstacle shunning?}
It is when a physical object is placed at the same location as the phantom body
part. Approximately 50\% of the patients believe the phantom limb is still
there, and the other half realize that the phantom limb is in his/her head. In
some cases, the confrontation of the phantom limb with an obstacle makes the
person change the postural body as if it is necessary to move the phantom limb
of place. However, it is interesting to notice that even if the phantom
sensation disappears, the phantom pain can still holds.
Shunning is associated with worse prothesis tolerance.

\item \paragraph{Explain what is supernumary and negative phantom limbs.}
Supernumary is a condition where the individuo believes and receives sensory
information from limbs that do not exist and never have existed. This can be
simulated by the pinochio illusion or by rubber hand experiment. In the
pinochio ilusion, the participant close his/her eyes and touchs his/her nose
and have the biceps stimulated. The participant will perceive arm and/or nose
in different location. In the rubber hand experiment, a rubber hand is placed
in front of the participant and his/her real hand is hided by a cardboard. A
second person will, using identical movements, do the same thing with the fake
and real hand. After a while, participants are convinced that the fake hand is
their own hand.

Negative phantom limbs happens when the limb exists but the person do not
recognize the existence of the body part. This phenomenon is also called
xenomelia. This sensation is accompanied by reduced response to tactile
stimulation in the ``unconnected'' limb. Also is related with reduced thickness
and volume in Superior Parietal Lobe and smaller area in Inferior Parietal
Lobe, S1 and S2 areas.

\item \paragraph{What is whole-body phantom? How can one induce it?}
The whole-body phantom is purely somaesthetic illusion, i.e, one can feel but
cannot see. Patients usually point a precise localization in space and it
happens almost exclusively after Parietal Lobe Lesions.
The visual experience of wholy-body phantom is called autoscopic phenomena,
represents a dissociation between the image of him/herself and the body. It is
divided in three groups:

\begin{itemize}
\item Out-of-body experience: people feel their self outside the physical body
and somewhat elevated.
\item Autoscopic Hallucination: people experience seeing a double of
themselves, without ``leave'' the original body.
\item Heautoscopy: people experience seem a double of themselves. However, the
subject cannot point out where the self is localized.
\end{itemize}

To induce the whole-body phantom we can use the out-of-body experience: observe
the own body in front of oneself (for example, using a head mountad display)
and simultaneously touch the real and fake bodies. After a while, the
participant will felt that he/her is at the place of seen touch.
\end{enumerate}

%%---------------------------------------------------------------------------%%
%%%%%%%%%%%%%%%%%%%%%%%%%%%%%%%%%%%%%%%%%%%%%%%%%%%%%%%%%%%%%%%%%%%%%%%%%%%%%%%
%%---------------------------------------------------------------------------%%
\subsubsection{Neurology}
\textbf{Ophtalmology}\\
\begin{enumerate}
\item \paragraph{How can we use saccadic eye information to diagnose
disorders?}
Saccadic disorders may produce abnormal latency to initiate eye movements,
abnormal speed of eye movements (generally slow), or abnormal accuracy of eye
movements (hypometria or hypermetria). The specific type of abnormality relates
to the pattern of neural activity delivered to the ocular motoneurons.

\item \paragraph{Explain nystagmus.}
Nystagmus is a condition of involuntary eye movement, acquired in infancy or
later in life, that may result in reduced or limited vision. Due to the
involuntary movement of the eye, it has been called "dancing eyes". Can be
related with a loss of vestibulo-occular reflex, when the head moves, the eyes
move together and then, after a while, the eyes turn to the object-goal position.
In spontaneous nystagmus, eyes drift to the side of the loss.

\item \paragraph{What is the Romberg's test?}
Romberg test is a test to identify loss of balance. The Romberg test is based
on the premisse that a person needs two of three of the following senses to
maintain balance: proprioception, vestibular function and vision. If a person
loses the balance when close the eyes, then exists a disorder on proprioception
or on the vestibular system.

\end{enumerate}
\textbf{Otology}\\
\begin{enumerate}

\item \paragraph{What is otology?}
Otology is the study of the anatomy and disease of the ear. It is related with
hearing and vestibular systems. Neurotology is a focused study of diseases of
the inner ear.

\item \paragraph{Describe some tests used in neurotology.}
One common test used in neurotology is the caloric test.

In the caloric test, the external auditory canal is irrigated with cold or warm
water or air. The difference in temperature between the body and the injected
substance should mimics a head turn. If the water is warm (44$^{\circ}$C or
above) it mimics a head turn to the ipsilateral side, and thus both eyes must
turn toward the contralateral ear. If the water is cold (30$^{\circ}$C) it
mimics a head turn to the contralateral side, and thus both eyes must turn
toward the ipsilateral ear. Absent reactive eye movement suggests vestibular
weakness of the side being stimulated.

\end{enumerate}

\textbf{Parkinson}\\
\begin{enumerate}
\item \paragraph{What is Parkinson? Is it a syndrome or a disease?}
Parkinson is the syndrome of having a combination of slowness of movement,
stiffness, tremor at rest and postural instability. Many related diseases can
cause the parkinson syndrome. The most common reason is the Parkinson disease.
Parkinson disease is the second most common degenerative disorder of the
nervous system.

\item \paragraph{What does cause the parkinson disease?}
The symptoms of parkinson are due to the progressive loss of dopaminergic
neurons in the substancia nigra pars compacta, which projects and innervates
neurons in the caudate and putamen. The cause of the deterioration of these
dopaminergic neurons is not known. Some genes are been studied and linked with
Parkinson: $\alpha-synuclein$ and \textit{parkin}.

\item \paragraph{What are the treatments for parkinson?}
Since the symptoms are caused by loss of dopamine, the most common treatment is
medication to increase dopamine in the caudate and putamen. However, this
treatment is a symptomatic-therapy, not a cause-therapy.
To achieve a true cure the mechanisms of neuronal cell death need to be
understood and strategies for enhancing neuronal survival and regrowth need to
be developed.

\end{enumerate}

\textbf{Neuromuscular disorders}\\
\begin{enumerate}
\item \paragraph{What are neuromuscular disorders?}
Neuromuscular disorders affect the nerves that control the voluntary muscles.
When the neurons that control the muscles become unhealthy or die,
communication between the nervous system and muscles breaks down. As a result,
the muscles weaken and waste away.

\item \paragraph{How neuromuscular disorders can be diagnosed?}
The diagnostic of neuromuscular disorders can use medical history, neurological
examination and laboratory tests.
A common test is the Nerve Conduction Studies~(NCS), it is a measurement of the
speed of conduction of an electrical impulse through a nerve. NCS can determine
nerve damage and destruction.
During the test, the nerve is stimulated, usually with surface electrode
patches attached to the skin. Two electrodes are placed on the skin over the
nerve. One electrode stimulates the nerve with a very mild electrical impulse
and the other electrode records it. The resulting electrical activity is
recorded by another electrode. The nerve conduction velocity (speed) is then
calculated by measuring the distance between electrodes and the time it takes
for electrical impulses to travel between electrodes.
Another test that may be performed is electromyography (EMG). An EMG measures
the electrical activity in muscles and is often performed at the same time as
NCS. Both procedures help to detect the presence, location, and extent of
diseases that damage the nerves and muscles.

\item \paragraph{What is myotonic reaction?}
Myotonia is a symptom of certain neuromuscular disorders characterized by
delayed relaxation (prolonged contraction) of the skeletal muscles after
voluntary contraction or electrical stimulation.
\end{enumerate}

\textbf{Stroke}\\
\begin{enumerate}
\item \paragraph{What is a stroke? Types and treatments.}
A stroke, also known as a cerebrovascular accident, is a medical emergency. Any
acute neurological symptom is a stroke unless proven otherwise. Strokes happen
when blood flow to the brain stops.
Within minutes, brain cells begin to die. There are two kinds of stroke. The
more common kind (80\% of the cases), called ischemic stroke, is caused by a
blood clot that blocks a blood vessel in the brain. The other kind, called
hemorrhagic stroke, is caused by a blood vessel that breaks (changing in blood
pressure control) and bleeds into the brain.
The symptoms of a stroke usually are related with trouble walking, speaking and
understanding, as well as paralysis or numbness of the face, arm or leg.
For ischemic stroke a treatment is the recanalization of the vessel or use of a
catheter. For hemorrhagic stroke an option is the use of a cliping.

\item \paragraph{What is a penumbra in a ischemic stroke?}
The penumbra is an area around the direct affected area. It is an area of
moderate reduction of the blood flow. Cells in penumbra are potentially
salvable. However it needs to be treated as soon as possible. After 270mins,
recanalization of a ischemic stroke can lead to hemorrhagia.

\item \paragraph{Constraint Induced Movement Therapy - limitations and
describe.}

The term Constraint-Induced Movement Therapy (CIMT) describes a package of
interventions designed to decrease the impact of a stroke on the upper-limb
function of some stroke survivors. It is a behavioural approach to
neurorehabilitation based on "Learned-Nonuse".
CIMT is typically performed for individuals following a stroke as between
30-66\% of the survivors will experience some functional loss in their impaired
limb. Furthermore, CIMT has also been performed for individuals with other
injuries and even for multiple sclerosis. The aim of CIMT is to improve and
increase the use of the more affected extremity while restricting the use of
the less affected arm.
The three major components of CIMT include;
\begin{itemize}
\item Repetitive, structured, practice intensive therapy in the more affected
arm
\item Restraint of the less affected arm
\item Application of a package of behavioural techniques that transfers gains
from the clinical setting to the real world, thus CIMT uses operant training
techniques applied in the context of rehabilitation medicine.
\end{itemize}

The limitations of CIMT come from the way the therapy is proposed. Patients
are expected to wear a mitt or sling to restrict movement of the unimpaired arm
for 90\% of waking hours, which is very restrictive. Also, Therapy sessions of
6 hours per day for 2 weeks is very intensive and increases patient and
caregiver burden. Given these challenges, some patients and therapists have
expressed concerns about the therapy.

\end{enumerate}

\textbf{Neuropsychological disorders}\\
\begin{enumerate}
\item \paragraph{Cite language disorders that you know.}
Aphasia are language disorders that affects a person's ability to communicate.
Some types of aphasia are:
\begin{itemize}
\item Broca's aphasia: a person with Broca's aphasia knows what he or she wants
to say, but is unable to accurately produce the correct word or sentence.
Expressing language in the form of speech and writing will be severely reduced.
\item Wernicke's aphasia: a persons with Wernicke's aphasia can produce many
words and they often speak using grammatically correct sentences with normal
rate and prosody. However, often what they say doesn’t make a lot of sense or
they pepper their sentences with non-existent or irrelevant words.
\item aprosodia: a person with aprosodia has an inability to properly convey or
interpret emotional prosody. Prosody in language refers to the ranges of
rhythm, pitch, stress, intonation, etc. 
\item alexia: a person with alexia loses the ability to read or understand
words, sentences, or, in some cases, even letters. It is also called visual
aphasia or word blindness. Interestingly, the person still have the ability to
write.
\end{itemize}

\item \paragraph{What is apraxia?}
Apraxia is a motor disorder caused by damage to the brain (specifically the
posterior parietal cortex), in which the individual has difficulty with the
motor planning to perform tasks or movements when asked, provided that the
request or command is understood and he/she is willing to perform the task.
A person can use a tool but cannot imitate their use when asked.

\item \paragraph{What is anosognosia?}
Anosognosia is a deficit of self-awareness, a condition in which a person with
some disability seems unaware of its existence. It is a neurological disorder
and distinguishes from denial, which is a psychological defense mechanism.
 
\end{enumerate}

%%---------------------------------------------------------------------------%%
%%%%%%%%%%%%%%%%%%%%%%%%%%%%%%%%%%%%%%%%%%%%%%%%%%%%%%%%%%%%%%%%%%%%%%%%%%%%%%%
%%---------------------------------------------------------------------------%%
\subsubsection{Multiple Sclerosis}
\begin{enumerate}
\item \paragraph{In which way may the pathological and pathophysiological
disturbance in MS help to explain the clinical course of the disease?}

Multiple sclerosis is an inflammatory demyelinating disease of the CNS in which
activated immune cells invade the central nervous system and cause
inflammation, neurodegeneration and tissue damage. Multiple sclerosis can be
pathologically defined as the presence of distributed glial scars (or
sclerosis) in the central nervous system disseminated in time and space.
Currently it is unknown what the primary cause of MS is. Current models can be
divided into two groups: Inside-out and Outside-in. In the first ones, it is
hypothesized that a problem in the CNS (like an inflamation or viral illness)
produces an immune response that destroys myelin and finally breaks the BBB. In
the second models, an external factor produces BBB leaks, immune cells enter
the CNS, and destroys myelin and axons. Progression of MS is linked to
increasing neurologic dysfunction which by current understanding reflects
neurodegenerative processes.

\item \paragraph{Different aspects of MS are listed. Weight the importance of
the different aspects and their impact for treatment.}
\label{question:MS-aspects}

There is no cure for multiple sclerosis. Treatment typically focuses on
speeding recovery from attacks, slowing the progression of the disease and
managing MS symptoms. Some people have such mild symptoms that no treatment is
necessary.
Some treatments that can be used in MS are:
\begin{itemize}
\item Physical therapy: a physical or occupational therapist can teach the
patient stretching and strengthening exercises and show him/her how to use
devices to make it easier to perform daily tasks. Also, together with the use
of a mobility aid when necessary can also help manage leg weakness and other
gait problems often associated with MS.
\item Muscle relaxants: the patient may experience painful or uncontrollable
muscle stiffness or spasms, particularly in legs. Muscle relaxants such as
baclofen (Lioresal) and tizanidine (Zanaflex) may help.
\item Medications to reduce fatigue
\item Other medications: medications also may be prescribed for depression,
pain, sexual dysfunction, and bladder or bowel control problems that are
associated with MS.
\end{itemize}

\item \paragraph{Main clinical features (typical symptoms) of chronic MS?}
\label{question:MS-symptoms}

Classic MS symptoms are:
\begin{itemize}
\item Sensory loss (ie, paresthesias): usually an early complaint
\item Motor symptons: muscle cramping secondary to spasticity
\item Autonomic symptoms: bladder, bowel, and sexual dysfunction
\item Optic neuritis: loss of vision (or loss of color vision) in the affected
eye and pain on movement of the eye. 
\item Eye symptoms: including diplopia on lateral gaze
\item Heat intolerance
\item Constitutional symptoms: fatigue and dizziness; fatigue must be
differentiated from depression (which may, however, coexist), lack of sleep,
and exertional exhaustion due to disability
\item Pain
\item Subjective cognitive difficulties: with regard to attention span,
concentration, memory, and judgment
\item Depression
\end{itemize}

\item \paragraph{What are treatment and therapies for MS?}
There is no cure for MS, but some treatment and therapies can slow down the
advance of the disease and reduce the symptoms.
Experiencing a relapse is discouraging, so ending it as quickly as possible
benefits both the body and the mind. Inflammation is a key feature of MS
relapses, and it can lead to many of the secondary symptoms. Corticosteroids
and interferons are often used to ease inflammation and reduce the severity of
attacks. Interferons dampen the overactive immune system. Also some therapies
can help with the symptoms as: physical therapy (can help manage leg weakness
and other gait problems), speech therapy (for speech difficulties, such as long
pauses or slurred language, but also for swallowing) or injections of botulinic
toxin (alleviating bladder and bowel problems).

\item \paragraph{How does the blood brain barrier (BBB) function in MS?}
The blood-brain barrier (BBB) is a cellular and metabolic barrier located at
the capillaries in the brain that alters permeability, restricting the passage
of some chemical substances and microscopic objects from the bloodstream into
the neural tissue, while allowing other substances to pass into the brain. For
example, the blood-brain barrier restricts the passage of bacteria, red-blood
cells, and certain toxins and water-soluble molecules into the brain while
allowing the passage of oxygen, glucose and amino acids. Normally, a person's
nervous system would be inaccessible for the white blood cells due to the
blood-brain barrier. However, it has been shown using MRI that, when a person
is undergoing an MS ``attack'', the BBB has broken down in a section of the
brain or spinal cord, allowing white blood cells called T lymphocytes to cross
over and destroy the myelin. 

\item \paragraph{What is a BBB test in MS?}
CSF shows oligoclonal bands and elevated IgG synthesis in over 80\% of patients.
Evoked potentials that may show problems with sensory systems that are not
known from clinical exam. However, none of these (except serial MRIs) can show
whether lesions are separated in space and time.

\item \paragraph{In which points are treatments satisfactory?  In which points not so?}

\item \paragraph{What is the prevalence / incidence in Switzerland?}
The prevalence is 110 per 100.000 inhabitants. The average incidence in Europe
is 108 per 100.000 inhabitants.

\item \paragraph{What is the prevalence rate in Switzerland and how are such figures determined?}
The prevalence is 110 per 100.000 inhabitants. The average incidence in Europe
is 108 per 100.000 inhabitants. The prevalence rate is defined as the number of people that have the disease at a specific point in time.

\item \paragraph{Describe mechanisms of demyelination and axonal damage and mechanisms of remyelination and axonal repair?}
The early stages of MS involve relapsing-remitting where patient experience
demyelination associated with loss of function (i.e., vision and gait), which
is usually regained following remyelination. In the progressive stages of MS,
irreversible functional deficit occurs which has been associated with
progressive loss of axons and neurons.
Demyelination in multiple sclerosis occurs by two major mechanisms: primary
injury to the myelin sheath with relative preservation of oligodendrocytes, or
direct injury to oligodendrocytes (oligodendrogliopathy). Following
demyelination of the axon there is either (i) remyelination with axonal
preservation, or (ii) axonal death. Remyelination can take place by either
surviving mature oligodendrocytes, or more likely by the recruitment of
progenitor oligodendrocytes that mature into myelinating cells. Axonal death
occurs as a result of either the inflammatory response, in particular MHC Class
I restricted CD8 positive T cells, or by the failure of local neurotrophic
support as a result of injury to glial cells.
The majority of white matter tract axons were entirely enwrapped by myelin;
therefore, it is likely that axons cannot obtain proper nutrient support from
their external environment on their own and require metabolic support from
glial cells. Following demyelination, there is an overall increased demand for
ATP. The energy required to maintain the intra-axonal ionic balance and the
resting membrane potential by the Na+/K+ ATPase pump were increased, due to
redistribution and an overall increase in the number of Na+ channels.
Increased metabolism requirements for demyelinated axons may make these axons
more susceptible to death through disease mechanisms such as inflammation.

\item \paragraph{Where are the predilection sites in MS lesion (where are the plaques most frequently)? Reasons?}
Plaques are randomly distributed. They have a predilection for the 
periventricular white matter, optic nerves, and spinal cord but spare no part
of the CNS. The predilection for these sites may be explained in part by the
distribution of venules where blood–brain barrier breakdown takes place.

\item \paragraph{What cellular or histological features are typical for MS
plaques?}

Long-lasting chronic MS lesions are characterized by a relatively uniform
appearance of extensive primary demyelination with variable axonal loss,
gliosis and minor infiltration by lymphocytes and macrophages/microglia. Early
MS lesions show a profound inflammatory reaction consisting of T cells, a
variable but low number of B cells and massive infiltration by foamy
macrophages at the sites of demyelination.


\item \paragraph{What is the role of the BBB in MS pathogenesis and by which
techniques can it be elucidated?}

The presence of a blood–brain barrier (BBB) restricts the movement of soluble
mediators and leukocytes from the periphery to the CNS. Despite the presence of
this tightly regulated BBB, leukocyte entry into the CNS is an early event in
multiple sclerosis (MS), an inflammatory disorder characterized by the
formation of multifocal lesions in the brain and spinal cord. Acute MS lesions,
featuring areas of demyelination, axonal loss and immune cell infiltrates,
display BBB disruption. The BBB breakdown is a crucial and obligatory step in 
the paathogenesis of MS.

\item \paragraph{Which forms of MS do you know and how are they characterized?}
There are four types of MS:
\begin{itemize}
\item Relapsing-Remitting MS (RRMS): this is the most common form of multiple
sclerosis. About 85\% of people with MS are initially diagnosed with RRMS.
People with RRMS have temporary periods called relapses when new symptoms
appear.
\item Secondary-Progressive MS (SPMS): in SPMS, symptoms worsen more steadily
over time, with or without the occurrence of relapses and remissions. Most
people who are diagnosed with RRMS will transition to SPMS at some point.
\item Primary-Progressive MS (PPMS): this type of MS is not very common,
occurring in about 10\% of people with MS. PPMS is characterized by slowly
worsening symptoms from the beginning, with no relapses or remissions.
\item Progressive-Relapsing MS (PRMS): a rare form of MS (5\%), PRMS is
characterized by a steadily worsening disease state from the beginning, with
acute relapses but no remissions, with or without recovery.
\end{itemize}

\item \paragraph{Why does the inflammation in an MS-lesion subside after a while?
What mechanisms can dampen an ongoing immune response (if you do not know,
speculate!)?}

In MS, subsequent to the demyelination and degeneration, the remyelination and
neurogenesis are stimulated and progenitor cells migrate into damage sites.
Remyelination is the default spontaneous process in which demyelinated axons
are ensheathed with new myelin sheaths.
Remyelination is not only found in inactive lesions but can also be observed in
lesions with ongoing demyelinating activity. However, repair processes are
especially characteristic of the early disease phase. Remyelination may even
begin within a month or two after active demyelination. In contrast,
remyelination in late stage MS appears sparse and restricted to borders of
inactive lesions. Thus, myelin repair, termed remyelination, occurs in acute
inflammatory lesions in MS and is associated with functional recovery and
clinical remittances. The new myelin sheath may either act as a protective
physical barrier to damaging inflammatory molecules, or restore trophic support
to the axon.

\item \paragraph{How MS can be diagnosed?}
To diagnose MS is necessary find demyelination. MRI information and additional
signs can be used. Also, the presence of antibodies in the CSF and its absence
in the serum indicates MS.

\end{enumerate}


%%---------------------------------------------------------------------------%%
%%%%%%%%%%%%%%%%%%%%%%%%%%%%%%%%%%%%%%%%%%%%%%%%%%%%%%%%%%%%%%%%%%%%%%%%%%%%%%%
%%---------------------------------------------------------------------------%%
\subsubsection{Spinal Cord Injury}
\begin{enumerate}
\item \paragraph{What are the two main aetiologies of SCI?}
SCI can be traumatic or non-traumatic.
The traumatic SCI occurs when an external physical impact acutely damages the
spinal cord as primary injury. In the non-traumatic SCI, a chronic disease
process, such as a tumor or infection generates the primary injury.
In both cases, the primary injury damage cells and initiates a complex secondary
injury cascade, which produces death of neurons and glial cells, ischemia and
inflammation.

\item \paragraph{How is SCI diagnosed?}
A suspection of SCI includes identification of an injury (or a process that
generates a primary injury) and the presence of some of the symptoms: acute
head, neck, or back pain; decline of feeling in the extremities; loss of
control over part of the body; urinary or bowel problems; walking difficulty;
pain or pressure bands in the chest area; difficulty breathing; head or spine
lumps.

Patients with suspected SCI undergo neurological examination (including
voluntary motor and sensory examination of each limb and a rectal examination)
and radiographic imaging to look for damages. The images techniques commonly
used are X-ray, CR and MRI. X-ray and CT are useful to detect gross fractures
and MRI is used to evaluate soft tissues, such as discs, ligaments and nerve
roots.

\item \paragraph{Why SCI can sometimes cause urinary problems?}
Regulatory functions as pupil diameter, gut motility, and urinary system are
controlled by the sympathetic nervous system. Sympathetic pathways start in the
spinal cord. Some SCI can interrupt this pathways and then generates a variety
of side effects as urinary problemas, gastrointestinal problems, and even
sexual function.

\item \paragraph{Cite two SCI syndromes and its characteristics.}
\begin{itemize}
\item Central cord syndrome: most commonly diagnosed in elderly patients. It is
characterized by disporportionate motor impairement of the upper limbs rather
than the lower limbs.
\item Brown-Sequard syndrome: most commonly observed in individuals with
penetrating traumatic SCI. It is characterized by loss of motor function,
proprioception and vibration sensation ipsilateral to the injury, and loss of
pain and temperature sensation contralateral to the injury.
\end{itemize}

\item \paragraph{How a SCI develops from the primary injury?}
The primary injury causes displacement or dislocation of the vertebral column,
which causes compression or transection of the spinal cord. This causes a focal
region of cell death and blood-spinal cord barrier disfunction, which causes a 
a cascade of secondary injury mechanisms. This secondary injury mechanisms
includes vascular changes, inflammation, loss of ionic homeostasis and
oxidative stress (increasing the damage to the spinal cord). Cell death
contributes to the formation of cystic cavities, which are surrounded by a
glial scar (a deposition of astrocytes and molecules that inhibit neuronal
regeneration), which interfere with neuronal regeneration and functional
recovery.

\item \paragraph{Describe the temporal phases of the SCI.}
SCI can be temporally divided into the acute ($<$ 48 hours), subacute (48 hours
to 14 days), intermediate (14 days to 6 months) and chronic ($>$ 6 months)
phases.

In the acute phase blood vessel injury can cause severe hemorrhages, which can
expose the cord to an influx of inflammatory cells. The inflammatory response
combined with the disrupted blood-spinal cord barrier, progressively creates a
swelling, leading to further mechanical compression of the cord, which can
extend for multiple spinal segments and worsen the injury. Cell death in acute
phase promotes cystic cavities. These cavities become a barrier to directed
axonal regrown and are a poor substrate for cell migration.

In the subacute phase ischemia and excitotoxicity contribute to a loss of
homeostasis, and ongoing necrosis of neurons and glia releases ATP and
potassium, which can activate microglial cells. Activated microglial cells
infiltrate the injury site, where they propagate the inflammatory response.
Phagocytic inflammatory cells can clear myelini debris at injury site but can
also induce further damage to the spinal cord.

The multiple causes of cell death that occur during the acute and subacute
phases of SCI can produce greater damage than the original primary injury and
form the basis for the neuroprotective intervetions.

Lastly, the chronic phase is marked by attempts at remyelination, vascular
reorganization, alterations in the composition of the extracelular matrix~(ECM)
and remodelling of neural circuits.

\item \paragraph{How animal models can be applied to study of SCI?}
Animals models are useful for the preclinical testing of new therapies.
The ideal animal model should anatomically and pathophysiologically resemble
humans SCI, require minimal training, be inexpensive and produce consistent
results. Rat models are the most commonly used for SCI research, however,
differences in size, molecular signalling, anatomy and the recovery potential
following SCI have made direct translation challenging.
Large animal models, such as non-human primates, overcome some of these
barriers, but substantial differences in cost and unique housing requirements
make their use less common and even they are unable to exactly mimic human SCI.

\item \paragraph{Cite some complications of SCI.}
Complications can be local or systemics. Examples of local complications are
neuropathic arthropathy (slow progressive destruction of a joint) and
spasticity (can affect mobilization, activities of daily living and sleep).
Some systemics complications are cardiovascular, respiratory (leading cause of
mortality in patients with chronic SCI), secondary immunodeficiency,
genitourinary and gastrointestinal problems, pressure sores etc.

\end{enumerate}

%%---------------------------------------------------------------------------%%
%%%%%%%%%%%%%%%%%%%%%%%%%%%%%%%%%%%%%%%%%%%%%%%%%%%%%%%%%%%%%%%%%%%%%%%%%%%%%%%
%%---------------------------------------------------------------------------%%
\subsubsection{Epilepsy}
\begin{enumerate}

\item \paragraph{What is epilepsy?}
It is a disorder in which nerve cell activity in the brain is disturbed,
causing seizures.

\item \paragraph{What is a seizure?}
A seizure happens because of abnormal electrical activity in the brain. It is a
temporary disruption of normal brain functions. A seizure usually change the
eyes moviment and the body tension. Seizures does not mean epilepsy. In
epilepsy the seizures are unprovoked.
Seizures are caused by transient functional disturbances of cerebral neurons
leading to excessive and/or synchronous discharges.

\item \paragraph{How do seizures are classified?}
The seizures can be classified in focal or generalized.
The focal seizures are disturbs that starts in a specific part of the brain.
These seizures can often be subtle or unusual and may go unnoticed or be
mistaken for anything from intoxication to daydreaming.
The generalized seizures are the result of abnormal activity in both
hemispheres and usually associated with unconsciouness.

\item \paragraph{What is EEG for test seizures? How it works?}
An electroencephalogram~(EEG) is a test that measures and records the
electrical activity of the brain. Electrodes are attached to the patient's
head and the brain's electrical activity is recorded. EEG measures synchrony
post-synaptic potential. A hypersynchronization indicates neuron participation
in the generation and propagation of the seizure. Spikes and sharp waves are
an epileptiform of brain electrical activity.

\item \paragraph{What are the treatments for epilepsy?}
The are some treatments for epilepsy. The treatment aims at preventing
seizures, and can make a patient seizure-free but cannot treat the underlying
cause of the epilepsie. The most common is pharmacotherapy, where drugs can
control the seizures. If the area on brain that cause the seizures is small,
well defined and do not control speech, sight, moviment or hearing, surgery can
be a treatment option or even a palliative option, just to reduce the frequency
and/or severity of seizures. More than that, depending on the type of the
seizure, an specific treatment can be used as diets or nerve stimulation.

\item \paragraph{Compare idiopathic vs symptomatic epilepsies.}
Idiopathic epilepsies are genetically determined and cannot be treated
surgically. However, most patients with such syndromes respond very well to
medical treatment.
On the other hand, symptomatic epilepsies are caused by identifiable alteration
of brain tissue (tumors, scars, inflammation etc). Patients with symptomatic
epilepsies who do not respond well to antiepileptic drugs, can see the surgical
treatment as a possibility.

\end{enumerate}

%%---------------------------------------------------------------------------%%
%%%%%%%%%%%%%%%%%%%%%%%%%%%%%%%%%%%%%%%%%%%%%%%%%%%%%%%%%%%%%%%%%%%%%%%%%%%%%%%
%%---------------------------------------------------------------------------%%
\subsubsection{Depression}
\begin{enumerate}
\item \paragraph{Every sadness is a depressive state? How depression can be
diagnosed?}

Feeling sad is a common mood in response to loss, stress or disappointment.
The Major Depressive Disorder (or depression) involves persistents symptoms
that includes a combination of depressed mood, diminished interest or pleasure,
changes in sleep and appetite, loss of energy, thought of death and suicide,
etc.

\item \paragraph{Please describe briefly:}
\subparagraph{Some conceptional problems in studying depression from a
neuroscientific point of view} 

Depression is a disorder of subjective feeling. It is very difficult to
evaluate the disease scientifically, or even create animal models that are
adequate to depression because they can't talk to tell their feelings. The
definition of psychiatry depends on social values and personal evaluation of
suffering, but not depends on the organic disorder. No reliable objective
markers like genetic defects or metabolic disfunctions exist to analise the
depression in a objective way.

\subparagraph{Some changes of the neurobiology system in a depression state}
\label{question:neurobiology-depression}
\begin{itemize}
\item Change of HPA-axis (hypothalamus-pituitary-adrenal): the prominent
mechanism by which the brain reacts to acute and chronic stress is the
activation of HPA-axis that rises cortisol levels. Hypothalamus secretes CRH
(corticotropin-releasing hormon) leading the pituitary gland (hypophysis) to
secrets adrenalcorticotropin (ACTH). The ACTH interacts with receptors in the
adrenal gland stimulating the production and release of cortisol. Cortisol is
the adrenal glucocorticoid stress hormone secreted in humans and other
primates.
\item Reduction of growth hormone: causes sleep disorders, disturbances of
appetite regulation and functioning of limbic system.
\end{itemize}

\end{enumerate}

%%---------------------------------------------------------------------------%%
%%%%%%%%%%%%%%%%%%%%%%%%%%%%%%%%%%%%%%%%%%%%%%%%%%%%%%%%%%%%%%%%%%%%%%%%%%%%%%%
%%---------------------------------------------------------------------------%%
\subsubsection{Schizophrenia}
\begin{enumerate}
\item \paragraph{Please describe shortly a neurobiological model of
schizophrenia}\label{question:neurobiology-schizophrenia}
Imaging studies shows some consistently structural alteration in the brain of
schizophrenic patients, for instance, less activity in prefrontal cortex,
activity on visual and auditory cortices, differences in size of basal ganglia,
hyppocampus and amygdala. Also, some differences in dopamine was found.
Each one of this differences can be used to justify the symptoms of
schizophrenia.
The prefrontal cortex helps people to think logically and organize their
thoughts, so, less activity in this area can be related with the common symptom
of disorganized speech/behavior. Also difference in size of basal ganglia,
amygdala and hippocampus can affect the movement patterns and basic feelings,
(causing inability to express emotions and attention deficits, for instance).
More than that, the variation in dopamine can be related with apathy and
inability to experience pleasure.

\item \paragraph{Describe two different approaches in schizophrenia research.
How do they contrast/complement?}

\item \paragraph{Positive symptoms, how are they linked to dopamine?}
The positive symptoms represent distortions or exaggerations of normal
cognitive or emotional functions. Some positive symptoms of schizophrenia are
dellusions, hallucinations, disorganized speech and psychomotor disturbance.
They can be linked with an excess of dopamine in some parts of the brain. This
excess in dopamine can activate some brain areas as visual or auditory cortex
leading to hallucionation for instance.

\item \paragraph{Negative symptoms, how are they linked to dopamine?}
The negative symptoms reflect a loss or diminution of normal cognitive and
emotional functions. Some negative symptoms of schizophrenia are attention
deficits, lack of initiative, apathy, social withdrawal, diminuition in the
ability to express emotion and inability to experience pleasure. These
symptoms can be associated with a lack of dopamine. Since dopamine is
responsible for feeling pleasure and reward-motivated behavior, a lack of
dopamine can lead to loss of emotions, for instance.

\item \paragraph{How can dopamine release be influenced (pharmacologically, by
the environment)?}

Some estimulants (pharmacologics) can change the dopamine level by inhibit
dopamine uptake, resulting in increased dopamine concentration. An animal's
social environment and stress level actually can change the number of D2
receptors in the brain and change the proportion of the nucleus accumbens that
is dedicated to pleasure. Thus, the environment can alter how pleasurable a
given reward can be. 

\item \paragraph{How to diagnose schizophrenia?}
There is no definitive test to diagnose schizophrenia, but if some symptoms
persist for at least one month and are clearly not due to depression or
neuroleptic medication, it can be schizophrenia.
The symptoms that must persist are: delusions of control, hallucinatory voices giving a running commentary on the person's behavior, delusions that are culturally inappropriate or that are impossible, incoherence of speech, or one of the negative symptoms.

\item \paragraph{What are the treatments for schizophrenia?}
The therapy treatment is based on medication, psycotherapy and rehabilitation.
Some antipsychotic drugs can be used but with low dosage start (to reduce side
effects). Tranquilizers can be used during acute phases. In the long term,
control the treatment side effects is crucial to the rehabilitation
(reintegration) of the patient in society. Psycotherapy helps the patient to 
deal with his/her thoughts and behavior, learning more about the illness and
learning how to differentiate what is real and what is not.

\item \paragraph{How dopamine is related with schizophrenia?}
The strongest link between dopamine and schizophrenia comes from studies of
drugs that alleviate the symptoms of schizophrenia. A large number of
antischizophrenic drugs have two properties: block postsynaptic dopamine
receptors and inhibit the release of dopamine from presynaptic neurons.
One interpretation of these results is that schizophrenia is due to an excess
of dopamine. Another possibility is that the problem is not an excess of
dopamine, but a deficit of glutamate. If glutamate release is reduced, one way
to correct this deficiency would be block activity at dopaminergic receptors,
relieving glutamate synapses from inhibition.

\end{enumerate}

%%---------------------------------------------------------------------------%%
%%%%%%%%%%%%%%%%%%%%%%%%%%%%%%%%%%%%%%%%%%%%%%%%%%%%%%%%%%%%%%%%%%%%%%%%%%%%%%%
%%---------------------------------------------------------------------------%%
\subsubsection{Psychiatric disorders - general}
\begin{enumerate}
\item \paragraph{What is the neurobiological system in schizophrenia and depression?}
See answers \ref{question:neurobiology-schizophrenia} from Schizophrenia and
\ref{question:neurobiology-depression} from Depression.

\item \paragraph{Animal models of behaviour allow us to investigate the
symptoms of psychiatric disorders such as depression and schizophrenia. Discuss
statements, giving examples of some specific model.}

Modeling of human neuropsychiatric disorders in animals is extremely
challenging given the subjective nature of many key symptoms, the lack of
biomarkers and objective diagnostic tests, and the early state of the relevant
neurobiology and genetics. Many of the symptoms used to establish psychiatric
diagnoses in humans (e.g., hallucinations, delusions, sadness, guilt) cannot be
convincingly ascertained in animals. When there are reasonable correlates in
animals, (eg., abnormal social behavior, motivation, working memory, emotion,
and executive function), the correspondence may only be approximate.

Also, you always have to ensure that the animal model is valid. There are
different aspects of validity that have to be guaranteed:
\begin{itemize}
\item Face validity: the quantifiable behavior and physiology in the animal
model have to be similar to the symptoms in the investigated human illness.
\item Construct validity: the quantifiable behavior and physiology in the
animal model must be a result of the same central state as in the human
patient. Theoretical rationale.
\item Predictive validity: close correspondence between drug actions on
behavior and physiology of the animal model and in the human patient.
\item Inter-laboratory validity
\item Inter-species validity
\end{itemize}

There are three major clusters of symptoms in Schizophrenia: positives,
negatives and cognitive. Positive symptoms include hallucinations and
delusions, experiences that are not characteristic of normal mental life.
Negative symptoms represent deficits in normal functions such as impoverished
speech, asocial behavior, and diminished motivation. Cognitive symptoms include
deficits in working memory and conscious control of behavior.
One common symptom is the attentional impairment, this happens because the
schizophrenic patient has an inability to focus on important stimulus. A
specific test, Latent Inhibition~(LI), was proposed. In the LI, a neutral
stimulus is presented much before the unconditioned stimulus. Healthy people
take too longer to learn the correlation between the stimulus. However, a
schizophrenic patient learns the relation rapidily (as the stimulus were
presented paired).
In animals: Rats were present with a tone followed by a schock, but not paired.
Rats are not able to learn to avoid the schock this way. However, rats treated
with amphetamine learn to avoid the box where they got shocked, due to their
impairement on LI.

In depression, the symptoms are related with diminished interest, loss of
energy, diminished ability to think or concentrate. The most common animal
model for depression is the \textbf{learned helplessness}: animals are exposed
to negative stimuli and don't get the possibility to escape. This leads to the
``learned helplessness'' symptom, especially, if the animals are very young,
which means, they give up very quickly and are not able to escape unwanted
situations. Learned helplessness can be measured by the escape behavior in a
two-way avoidance test. In this test, animals are placed in a shuttle box and
exposed to a foot shock. They are allowed to escape to the save compartment of
the shuttle box. If they get conditioned for the shock with a tone, starting
shortly before the shock, animals learn to escape already at the presentation
of the tone. ``Helpless'' animals are not good in escaping compared to controls.
\textbf{Chronic mild stress}: Animals are chronically exposed to mild stress
like food/water deprivation for some hours, not enough space, over night
illumination etc. The loss of pleasure (anhedonia) is measured with the
ICSS~(Intra-Cranial Self-Stimulation), the PRS (progressive reward schedule) or
the sucrose preference test.
Early life stress: Pups are stressed by separating them from mother for several
hours per day etc, this lead to anhedonia (inability to experience pleasure).

\item \paragraph{Describe methods for measuring motivation, attention and
memory in rodents and/or primates. In which neuropsychiatric disease are these
behavioural processes disrupted?}

Motivation, attention and memory are commonly affected by neuropsychiatric
disorders. Depression is the disorder most related with changes in motivation,
however, schizophrenia and sometimes parkinson patients also present lack of
motivation. Attention is mainly impaired in schizophrenia, but also in
depression. Memory is affected by Alzheimer and some cases in Parkinson, and also commonly in aging.

Measuring motivation:
Learned Helplessness Test: animals are exposed to negative stimuli and don't
get the possibility to escape. This leads to the ``learned helplessness''
symptom, especially, if the animals are very young, which means, they give up
very quickly and are not able to escape unwanted situations. Learned
helplessness can be measured by the escape behavior in a two-way avoidance
test. In this test, animals are placed in a shuttle box and exposed to a foot
shock. They are allowed to escape to the save compartment of the shuttle box.
If they get conditioned for the shock with a tone, starting shortly before the
shock, animals learn to escape already at the presentation of the tone.
``Helpless'' animals are not good in escaping compared to controls.

Measuring attention:
One method to measure attention in rodents is the Multiple Choice Serial
Reaction Time Test. In this test, a rodent is positioned in front, for instace,
of five doors in which one of them will be briefly illuminated. The rodent
needs to poke the right door in order to receive the food. This test address
the attention of the animal.

Measuring memory:
Spatial memory - Morris Water Maze Test: In this task, a tank is filled with
milky water with a platform concealed somewhere beneath the surface. When a rat
is placed in the tank, he will swim around until he finds the platform. When
this task is repeated many times, he learns to use cues from outside the tank
and can find the platform almost immediately from any spot in the tank. He can
do this because he has created a spatial map of the tank and its surroundings.
If the rat's hippocampus is lesioned after he learns the task, there will be no
effect; if it is lesioned before (when he is naive), however, he is unable to
learn it.
Working memory - Radial Arm Maze Test: In this test, a mouse is placed in the
middle of a radial maze that contains food in some arms. The working memory is
assessed when the mouse visits each arm only once. 

\item \paragraph{The term ``frontotemporal dementias'' subsumes a heterogeneous
group of disorders:}

\subparagraph{Please describe the clinical presentations of patients with
frontotemporal dementia (major clinical syndromes, and characteristic
features).}

Frontotemporal dementia~(FTD) is a term for a diverse group of uncommon
disorders that affect the frontal and temporal lobes (areas generally
associated with personality, behavior and language). In FTD portions of these
lobes shrink. Signs and symptoms vary, depending upon the portion of the brain
affected.

The disorders grouped into FTD fall into three subtypes:
\begin{itemize}
\item Behavior variant frontotemporal dementia: characterized by prominent
changes in personality, interpersonal relationships and conduct.
\item Primary progressive aphasia: affects language skills, speaking, writing
and comprehension. It can be semantic (individuals lose the ability to
understand or formulate words in a spoken sentence) or agrammatic (a person's
speaking is very hesitant, labored or ungrammatical).
\item Disturbances of motor (movement or muscle) function: produce changes in
muscle or motor functions with or without behavior or language problems.
\end{itemize}
The most common signs and symptoms of frontotemporal dementia involve extreme
changes in behavior and personality.

\subparagraph{Which genes/gene loci have been associated with frontotemporal
dementia?}

Mutations in three genes \textbf{MAPT}, \textbf{GRN}, and \textbf{C9orf72} have
been associated with FTD. 

\subparagraph{Please summarize which major molecular subgroups of
frontotemporal dementias can be defined and briefly discuss current knowledge
and/or hypotheses on underlying pathomechanisms in the two most common
subgroups.}

FTD is characterized by abnormal protein deposits and almost all cases can now
be classified into three major molecular subgroups based on specific
accumulating proteins with the most common being FTD-tau and FTD-TDP
(accounting for ~40\% and 50\%, respectively) and FTD-FET (accounting for
~5–10\%).

Tau is a microtubule-associated protein that promotes microtubule assembly and
stability which are crucial for maintaining neuronal integrity and axoplasmic
transport. Abnormal intracellular accumulation of hyperphosphorylated tau is a
characteristic feature of a number of neurodegenerative disorders
(tauopathies), including several conditions that may present as FTD (FTLD-tau).

TDP-43 was identified as the protein that accumulates in the vast majority of
cases of FTLD with tau-negative, ubiquitin-positive inclusions (then referred
to as FTLD-U), and in most cases of amyotrophic lateral sclerosis~(ALS).

\item \paragraph{Describe the major diagnostic procedures and tools in the
differential diagnosis of dementias.}

No single test can diagnose dementia, so it is necessary to run a number of
tests that can help pinpoint the problem. Some of them are:
\begin{itemize}
\item Cognitive and neuropsychological tests: to evaluate the cognitive
function. A number of tests measure memory, orientation, reasoning and
judgment, language skills, and attention.
\item Neurological evaluation: to evaluate memory, language, visual perception,
attention, problem-solving, movement, senses, balance, reflexes and other
areas.
\item Brain scans: to check for evidence of stroke or bleeding or tumor or
hydrocephalus.Also to check patterns of brain activity.
\item Laboratory tests: blood tests can detect physical problems that can
affect brain function, such as vitamin B-12 deficiency or an underactive
thyroid gland. Sometimes the spinal fluid is examined for infection,
inflammation or markers of some degenerative diseases.
\end{itemize}

\item \paragraph{Please give a detailed account of the process by which prions,
upon entering the body, reach the CNS.}

The term ``prion'' is derived from proteinacious infectious particle. This
small infectious particle is a disease-causing form of a protein called
cellular prion protein (PrPc). Prion diseases are contracted by exposure to
prion infectivity. There are many ways to the prions enter the body, for
example by direct brain exposures (on neurosurgery), by oral or parenteral
uptake etc.
After replication and accumulation in lymphoid organs, prions invade the
nervous system through sympathetic and parasympathetic nerves.

\item \paragraph{What is the significance of the blood brain barrier~(BBB) in
infectious disease of the CNS?}
The blood–brain barrier~(BBB), composed mainly of specialized brain
microvascular endothelial cells, maintains biochemical homeostasis in the CNS
by regulating the passage of nutrients, molecules and cells from the blood to
the brain.
The restrictive nature of the BBB provides an obstacle for drug delivery to the
CNS, and, thus, major efforts have been made to generate methods to modulate or
bypass the BBB for delivery of therapeutics. Loss of some, or most, of these
barrier properties during neurological diseases including stroke, MS, brain
traumas, and neurodegenerative disorders, is a major component of the pathology
and progression of these diseases. BBB dysfunction can lead to ion
dysregulation, altered signaling homeostasis, as well as the entry of immune
cells and molecules into the CNS, processes that lead to neuronal dysfunction
and degeneration.
Although the BBB protects against CNS-directed inflammation by restricting
immune cell access to the brain, it can also regulate the local inflammatory
response by expressing proinflammatory molecules that promote the recruitment
of peripheral immune cells into the CNS.
\end{enumerate}

%%---------------------------------------------------------------------------%%
%%%%%%%%%%%%%%%%%%%%%%%%%%%%%%%%%%%%%%%%%%%%%%%%%%%%%%%%%%%%%%%%%%%%%%%%%%%%%%%
%%---------------------------------------------------------------------------%%
\subsubsection{Addiction}
\begin{enumerate}
\item \paragraph{What are the primary causes of mortality in Switzerland
(2007)?}
\begin{itemize}
\item Tobacco
\item Alcohol
\item Suicid
\item Traffic accidents
\end{itemize}

\item \paragraph{What drugs cause the most overall harm? What are the types of
harm?}
\begin{itemize}
\item Alcohol
\item Heroin
\item Cocain
\end{itemize}
The harm caused by drugs is divided in two categories: to the user itself and
to the others. In both categories there are physical, psychological and social
harm.

\item \paragraph{What defines a dependence syndrome?}
To consider the existence of a dependence syndrome, it necessary at least three
of the following symptoms together for at least one month (or in a small period
but repeatedly within a year).

\begin{itemize}
\item strong desire or compulsion to take the substance
\item impaired capacity to control substance taking behavior
\item physiological withdrawal state when substance use is reduced or ceased
\item evidence of tolerance to the effects of the substance
\item preocupation with substance use by alternative pleasures
\item persistent substance use despite harmful consequences
\end{itemize}

Addiction is the physical and psychological dependence, in which an individual continues the drug-taking behavior despite obviously harmful consequences.

\item \paragraph{What is/are the cause(s) of the dependency?}
Dependence is caused by environmental and genetic factors. A single use is not
enough to create a dependency, however some drugs need less time to create it.

\item \paragraph{Cite treatments for dependency and its goals.}
The general treatment goal are preservation and restoration of health and
social integration. To achieve this goal, treatment of concurrent diseases and
abstinece or moderate use (without dependence) are suited.

\item \paragraph{What is the Four Pillar Strategy?}
It was a policy to reduce the number of drug users in Zurich. The strategy was
based in a not drug free society but in drug use that is socially compatible.
The four pillars are: prevention, therapy, harm reduction and regression.

\item \paragraph{What is the Opioid Agonist Treatment?}
It is an effective treatment for addiction to opioid drugs. The therapy
involves taking the opioid agonists. These medications work to prevent
withdrawal and to reduce cravings for opioid drugs. People addicted to opioids
can take OAT to help stabilize their lives and to reduce harms related to their
drug use. Usually the agonists are long-acting opioids that replace the
short-acting opioids the person is addicted to. By acting slowly, it prevents
withdrawal for at most 36h without causing a person to get high, also it helps
to reduce or eliminate cravings for opioids drugs.

\item \paragraph{Explain how addiction occurs.}
Positive reinforcement is clearly necessary for the development of drug use,
but it is not sufficient to explain the development of compulsive use. There
are two theories to explain how the dependence occurs: the opponent-process
theory and the incentive-sensitization model.
In the opponent-process theory, drug addiction is the result of emotional
pairing of pleasure and emotional symptoms of withdrawal. At the beggining of
drug use there are high levels of pleasure and low levels of withdrawal. Over
time, as the level of pleasure decrease, the levels of withdrawal increase,
thus providing motivation to keep using the drug, despite a lack of pleasure
from it.
The incentive-sensitization model considers the motivation to abuse substances
must be more stronger than just like it. It states that addicts developed a
motivation called incentive salience. Incentive salience is an intense type of
wanting because the brain develops a strong association between a stimuli and a
reward. This association develops subconsciously but it can influence outward
behavior. Thus, addicteds may no longer even like the substance but they still
feel compelled to use it (want sensation is mediated by dopamine and like
sensation is mediated by endogenous opioids).

In any case, addiction is a pathological learning and memory, that involves
reward (nucleus accumbens and ventral pallidum), memory (amygdala and
hippocampus), control (prefrontal cortex and anterior cingulate gyrus) and
drive (orbital frontal cortex and subcallosal cortex). During addiction, the
enhanced expectation value of the drug in the reward, drive and memory circuits
overcomes the control circuit, favoring a positive-feedback loop initiated by
the comsumption and enhanced by activation of the drive and memory circuits.

\item \paragraph{What are the stages of addiction?}
The addiction is characterized by three stages: compulsion to seek and take the
drug (binge/intoxication), loss of control in limiting intake
(withdrawal/negative effects) and emergence of a negative emotional state when
access to the drug is prevented (preocupation/anticipation).
The first stage (binge/intoxication) is marked by a positive reinforcement by
the drug rewarding effects mediated by dopamine and endogenous opioids (VTA to
Nacc), the stimulus-response habit learning is enhanced and associative
learning of context cues happens.
In the second stage (withdrawal/negative effects) there is a negative emotional
state involving the extended amygdala. This induce stress and anxiety-like
effects, it projects to hypothalamus and brian stem. The endogenous opioids
decrease and then the negative reinforcement happens.
The third stage (preocupation/anticipation) represents high vulnerability to
relapse (even after prolonged abstinence). Conditioned drug-associated cues or
stress can elicit strong craving. Disrupted PFC is crucial for this stage.

\item \paragraph{What are the molecular target of drugs?}
There are three class of drugs according to their mechanism of action.
There are drugs that binds with G-protein (metabotropic receptors) as opioids,
cannabioids and hallucinogens. Three are drugs that binds to ionotropic
receptors like nicotine and alcohol. And there are drugs that interacts with
transporters (GABA receptors) like cocaine and amphetamines.

\item \paragraph{What is the relation between addiction and dopamine?}
The dopamine release facilitates learning. Drugs elicit more dopamine release
for greater duration than natural resources of reward (food, sex, social
interaction). Upon repeated administration tolerance does not develop to
drug-induced dopamine release, but natural rewards are weaken.
By blocking reuptake or enhancing release, cocaine and amphetamine increase the
synaptic availability of dopamine, norepinephrine and serotonin. However, the
acute reinforcing effects of these drugs depend critically on dopamine. Low
doses of dopamine receptor antagonists injected either systemically or
centrally into the nucleus accumbens, amygdala, or bed nucleus of the stria
terminalis block cocaine and amphetamine self-administration in rats.
However, without dopamine, one can still get addicted if there is serotonin in
the body. Cocaine and other psychostimulants increase serotonin activity in the
brain.

\end{enumerate} 

%%---------------------------------------------------------------------------%%
%%%%%%%%%%%%%%%%%%%%%%%%%%%%%%%%%%%%%%%%%%%%%%%%%%%%%%%%%%%%%%%%%%%%%%%%%%%%%%%
%%---------------------------------------------------------------------------%%
\subsubsection{Neurosurgery}
\begin{enumerate}
\item \paragraph{What is an aneurism? How it can be treated?}
An aneurysm is an arterial condition in which the wall of an artery weakens,
creating a bulge, or distension of the artery. Most aneurysms do not themselves
cause any symptoms, however, a large aneurysm may obstruct circulation to other
tissues. An aneurysm can also contribute to the formation of blood clots and
then obstruct smaller blood vessels. When an aneurism is found, it is necessary
determine if it must be treated. Many people do not have any symptom and can
live with it. Approximatelly 50\% of aneurism will result in bleeding, this way
some people do not need surgery.

\item \paragraph{How a bypass surgery is performed?}
Cerebral bypass surgery is performed to restore blood flow to the brain. The
surgery connects a blood vessel from outside the brain to a vessel inside the
brain to reroute blood flow around a damaged or blocked artery. The goal of
bypass surgery is to restore blood supply to the brain and prevent strokes.
Elana is the only technique available to make a bypass in the brain without
temporary occlusion of the blood-flow.

\item \paragraph{What are neurosurgery challenges and research?}
Some challenges are:
\begin{itemize}
\item understanding the origin of neurological disease
\item advance in treatments and outcomes of patients
\item advance in diagnosis and therapies
\item better imaging methods to figure out precise areas to remove in a surgery
\end{itemize}

One tool used in neurosurgery ir the fMRI that is based on neurovascular
coupling. However, in some cases (brain tumor, for instance) this coupling
doesn't hold and then, new techniques are necessary.
The respiract for instance, is an tool that controls the amount of
CO\textsubscript{2} consumed, and thus, be able to measure vascular reactivity.

\end{enumerate}

%%---------------------------------------------------------------------------%%
%%%%%%%%%%%%%%%%%%%%%%%%%%%%%%%%%%%%%%%%%%%%%%%%%%%%%%%%%%%%%%%%%%%%%%%%%%%%%%%
%%---------------------------------------------------------------------------%%
\subsubsection{Motor Learning and robot-assisted neurorehabilitation}
\begin{enumerate}
\item \paragraph{Robotic tools have played a significant role in the investigation of human motor learning.}

\subparagraph{Describe the role of internal models in human motor control and
how such models are acquired}
No answer.

\subparagraph{Identify three unique features of robotic systems that make them
valuable tools to investigate human motor learning.}
No answer.

\subparagraph{Discuss how these unique features could be applied to clinical
assessment and therapy of sensorimotor impairments} 
No answer.

\item \paragraph{Properties of motor learning - name them.  How is this used in therapy?}
No answer.

\item \paragraph{What is one advantage and one disadvantage of robot-assisted
rehabilitation?}
Some advantages:
\begin{itemize}
\item Intensive high repetitive tasks can be performed by robots with high
precision and endurance.
\item robotics tools increases training intensity
\item can be combined with engaging social and gaming features
\item enables a higher patient throughput by allowing the treatment of several
patients by one therapist in parallel
\item can facilitate guided motor training remotely through telerehabilitation
\item robots can even provide tactile feedback that corrects the impaired
movements
\item data collected during the robot training sessions can be quantified with
ease to complement the subjective and qualitative observation of clinicians
\end{itemize}

Some disadvantages:
\begin{itemize}
\item the acceptance by patients and physicians is still low
\item cannot provide some sensory inputs, such as temperature, touch, stroking
and psychological support.
\item robotic training is less flexible than hands-on therapy
\item it is necessary a specific robot for each type of therapy, i.e., the
robot used in shoulder therapy cannot be used for wrist therapy, for instance
\end{itemize}

\end{enumerate}

%%---------------------------------------------------------------------------%%
%%%%%%%%%%%%%%%%%%%%%%%%%%%%%%%%%%%%%%%%%%%%%%%%%%%%%%%%%%%%%%%%%%%%%%%%%%%%%%%
%%---------------------------------------------------------------------------%%
\subsubsection{Neuroimmunology}
This topic was not covered in 2017.
\begin{enumerate}
\item \paragraph{The maintenance of central and peripheral tolerance is the
reason that autoimmune diseases are relatively rare. Please answer the following
questions:}

\subparagraph{How does central T cell tolerance work (which organ performs T
cell education, what is negative and positive selection)?}

The tolerance of T cells begins as soon as a T-cell receptor is formed and
expressed on the cell surface of a T-cell progenitor in the thymus. Tolerance
mechanisms that operate in the thymus before the maturation and circulation of
T cells are referred to as ``central tolerance''. However, not all antigens
that T cells need to be tolerant of are expressed in the thymus, and thus
central tolerance mechanisms alone are insufficient. Fortunately, additional
tolerance mechanisms exist that restrain the numbers and or function of T cells
that are reactive to developmental or food antigens, which are not thymically
expressed. These mechanisms act on mature circulating T cells and are referred
to as ``peripheral tolerance''.
The thymus is the organ necessary to performe cell education. Thymocytes enter
the cortex and undergo T cell receptor (TCR) gene rearrangement and display
both CD4 and CD8. These cells interact with cortical thymic epithelial cells
(CTEC) and undergo apoptosis unless they receive a survival signal generated
via TCR/self-peptide—MHC interaction. Positively selected thymocytes progress
to single-positive CD4 or CD8 cells and enter the medulla. High-avidity
TCR/self-peptide—MHC ligation in the presence of medullary thymic epithelial
cell (MTEC) or dendritic cell (DC) co-stimulatory molecules in the medulla lead
to negative selection of self-reactive thymocytes.

\subparagraph{What are the mechanisms of peripheral tolerance? Remember, we
discussed four of them. Please shortly recapitulate.}

No answer.

\end{enumerate}

%%---------------------------------------------------------------------------%%
%%%%%%%%%%%%%%%%%%%%%%%%%%%%%%%%%%%%%%%%%%%%%%%%%%%%%%%%%%%%%%%%%%%%%%%%%%%%%%%
%%---------------------------------------------------------------------------%%
\subsubsection{Circadian Rhythm}
This topic was not covered in 2017.
\begin{enumerate}
\item \paragraph{Circadian pacemakers, entrainment, Zeitgeber, phas-response
curve}

Pacemaker: SCN (superchiasmatic nucleus of hypothalamus)
entrainment means that the ‘inner clock’, located in the SCN, is flexible in
the way that it can adapt the phase (example: time-zone flights) and the
frequency (example: bunker experiments, where one ‘day’ lasts 25 hours) of the
circadian clock.
Entrainment: via light, signalling from the eye to the SCN (possible
photoreceptor: Melanopsin?). In the SCN, per transcription is activated upon
light signal.
Phase-response curve: depending on the circadian time, when a light pulse is
presented, the phase of the circadian clock is shifted forward or backward. If
the light pulse is presented shortly before the active period has started, then
the phase is advanced and if the pulse is given shortly after the active period
has ended, the phase is delayed (in humans). There is one time point during
night when the phase shift swiches from delayed to advanced.

\item \paragraph{Which physiological and endocrine variables in human are
frequently used a phase-marker of circadian rhythm}

endocrine: melatonin, adrenal gland (adrenalin, cortison); GHRH (Growth
hormone releasing hormon)
Physiological: body temperature; activity (via activity monitor);
alpha-activity in the waking EEG

\item \paragraph{What is the evidence that SCN is a circadian pacemaker}

lesion method $\rightarrow$ arythmicity
in vitro culture of a single SCN neuron    
SCN transplant reserves the rhythm
in vitro SCN

\item \paragraph{Which genes (gene?) are (is) involvedin generation of
circadian rhythm}

in mammals: Bmal1 is rhythmically expressed by the SCN. Clock and bmal1(basic
helix-loop-helix transcription factor family) build heterodimers. These
heterodimers bind to E-boxes of enhancers of the per and cry gene. Per and Cry
proteins dimerize outside the nucleus and are phosphorylated. The dimers
re-enter the nucleus and downregulate transcription of  clock and bmal1
(negative feedback-loop). Situation is even more complex, also containing
positive feedback-loops...)
in fruit fly Cyc/Clk heterodimers activate per/tim gene transcription. Negative
feedback-loop: Per/Tim heterodimers inhibit activation of their own genes via
Cyc/Clk.
\end{enumerate}

%%---------------------------------------------------------------------------%%
%%%%%%%%%%%%%%%%%%%%%%%%%%%%%%%%%%%%%%%%%%%%%%%%%%%%%%%%%%%%%%%%%%%%%%%%%%%%%%%
%%---------------------------------------------------------------------------%%
\subsubsection{Sleep}
This topic was not covered in 2017.
\begin{enumerate}
\item \paragraph{Sleep regulation in physiological short and long sleepers:
Explain the most important principles how sleep and wakefulness are
physiologically regulated and how sleep-wake regulation may differ between
habitual short and long sleepers.}\label{question:sleep-physiological}
No answer.

\item \paragraph{Characteristics of sleep in mammals: Do they apply to
invertebrates?}
behavioural:
Sleeping site 
Quiescence
Body posture
Elevated arousal threshold
Rapid state reversibility
Physiological:
Altered EEG
Reduced muscle tone
Reduced heart rate
Reduced respiration
Reduced body temperature
Regulatory:
Compensatory response to sleep deficit or excess sleep

\item \paragraph{Non-REM-REM sleep}

REM: rapid eye movement. EEG low amplitude, mixed frequency (more similar to
wake than to deep sleep EEG). Most prominent in the morning hours.
non-REM: is subdivided into four substages 1-4 in human, deep sleep consists
of stages 3 and 4. In deep sleep, he EEG contains prominent slow waves (0.5-4.5
Hz, high amplitude).
Cycles: REM sleep occurs every 90-100 mins during sleep (ultradian oscillator
origins in the Pons). General term to describe cyclic alternation between REM
and non-REM sleep. Healthy people usually start with stage 1, then 2, 3, 4, 2,
REM, 2, 3, 4, REM etc.

\item \paragraph{Sleep homeostasis and marker of sleep homeostasis on the sleep
EEG}

homeostasis has been defined as the coordinated physiological processes wich
maintain most of the steady states in the organism; sleep homeostasis refers to
the sleep need in dependance of the time spent awake. Sleep need rises
exponentially during wake and declines exponentially during sleep. According to
2-process model of sleep regulation, sleep need is additionally dependant on
circadian time.
NREM-sleep is controlled thalamocortically.
Marker of sleep homeostasis: slow-wave activity (power of slow waves rises in
recovery sleep after sleep deprivation according to the 2-process model)

\item \paragraph{Endogenous sleep-promoting components: comments}

SCN (superchiasmatic nucleus of hypothalamus)
clock genes: transcriptional/translational process
melatonin: built during sleep
Thalamus: control of NREM-sleep
Pons: Regulation of REM-sleep
Potential homeostatic sleep-promoting agents (Experiment: if CSF from a sleep
deprived animal is transferred to a rested animal, the rested animal becomes
tired $\rightarrow$ there must be an agent in the CSF that accumulates during
wakefulness and makes tired): adenosine, Interleukin-1b, TNFa, GHRH,
prostaglandin.

\item \paragraph{Role of thalamus-correlated rhythm in sleep: comments}

Thalamus controls the NREM sleep rhythmus $\rightarrow$ EEG
activation/desactivation
\end{enumerate}

%%---------------------------------------------------------------------------%%
%%%%%%%%%%%%%%%%%%%%%%%%%%%%%%%%%%%%%%%%%%%%%%%%%%%%%%%%%%%%%%%%%%%%%%%%%%%%%%%
%%---------------------------------------------------------------------------%%
\subsubsection{Models of Computation}
This topic was not covered in 2017.

\begin{enumerate}
\item \paragraph{Roles of automate as models for computation}

The computational process in neurons can be investigated in neuroinformatics
via automate models (as compared to the structural process via neuroscience).

\item \paragraph{Automate suitable models for describing the operation of
neurons and networks of neurons}

Models for automates are transferable to neurons/neuronal networks:
\begin{itemize}
\item Feedforward processor: input $\rightarrow$ blackbox $\rightarrow$ output
(without memory). In neuron: input correlates to the dentritic input (sum of
input signals x weights), output correlates to the axonal output (fire or not
fire)
\item Finite State Machine: input $\rightarrow$ black-box (which remembers the
state in which it is, memory) $\rightarrow$ output. In neuron: neurons also
can feedback information to build up memory (this model accounts only for
short-time memory: seconds). Feedback occurs, when axonal output is networked
to dendrites of the very same neuron.
\item Turing Machine: input $\rightarrow$ black-box containing unbounded memory
$\rightarrow$ output.
\item Church-Turing-Thesis: this machine is able to compute all possible
computations. Philosophic question: is the brain’s memory unbounded?
\item Universal Turing Machine: can simulate the computational process of any
Turing Machine, when it knows the protocol of this machine, thus it can also
simulate the computational process of a neuron… But the protocol is not known
(e.g.: a bee can do computations leading to very various and complex behavior,
there is no computational model that could do that with the limited recourses
of a few thousand neurons that a bee needs to accomplish it).
\end{itemize}
	
However, neuronal networks have different weights for the 10 exp 14
axons/dentrite connections. This is not possible to be determined genetically
(not enough resources), but dependant on the microenvironment of each neuron in
the developmental process. Moreover, they can adapt to the environment by
changing those weights or even establishing new connections between axons and
dentrites. Synaptic release is additionally very versatile, can be modulated
chemically, and be inhibitory or excitatory etc.

\item \paragraph{What is the impact of W.I. (?) and what information do we get
from it?}

No answer.

\item \paragraph{ (a) Describe the properties of a finite state machine. (b) Is
a single neuron kind of  a finite state machine? Explain your answer. (c) What
kinds of model (artificial neurons) do you know about?}

No answer.
\end{enumerate}

%%---------------------------------------------------------------------------%%
%%%%%%%%%%%%%%%%%%%%%%%%%%%%%%%%%%%%%%%%%%%%%%%%%%%%%%%%%%%%%%%%%%%%%%%%%%%%%%%
%%---------------------------------------------------------------------------%%
\newpage

\section{References}
The pictures used in this summary are from the class slide sets or internet,
and belong to their respective owners. In the context of this summary they are
used for educational purposes only.

\subsection{Cognitive Neuroscience}
\begin{enumerate}
\renewcommand{\labelenumi}{\textbf{\theenumi}}
\item Christian C. Ruff and Scott A. Huettel, Chapter 6 - \textbf{Experimental
Methods in Cognitive Neuroscience}, In Neuroeconomics (Second Edition), edited
by Paul W. Glimcher and Ernst Fehr, Academic Press, San Diego, 2014, Pages
77-108, ISBN 9780124160088,
\url{http://dx.doi.org/10.1016/B978-0-12-416008-8.00006-1}
\item Olaf Blanke and Christine Mohr, \textbf{Out-of-body experience,
heautoscopy, and autoscopic hallucination of neurological origin}, In Brain
Research Reviews, vol 50, 2005, Pages 184 - 199,
\url{http://www.sciencedirect.com/science/article/pii/S0165017305000792}
\label{ref:out-of-body}
\item Neuroscience, Purves et.al, chapter 28 - \textbf{Emotions}
\item Neuroscience, Purves et.al, chapter 30 - \textbf{Memory}
\item Fundamental Neuroscience, 4th edition, chapter 48 - \textbf{Learning and
Memory}
\end{enumerate}

\end{document}